\documentclass{IEEEtran}
\usepackage{tikz}
\usepackage{pgfplots}
\usepackage{lipsum}
\usepackage[flushleft]{threeparttable}
\usepackage{booktabs}
\usepackage{amsmath}


\title{Difference in Sine Waves According to Amplitude, Frequency and Initial Phase.}
\author{João Pedro Battistella Nadas}


\begin{document}
\maketitle
\section{Introduction}
Sine waves are cool.
They can be used in a myriad of engineering and mathematics applications, such as fourier transforms, telecommunications, and many others. 
In this paper we aim to show the differences of sine waves when they have different frequencies, amplitudes and initial phases.

\section{System Model}
A sine wave is expressed as
\begin{equation}
    V(t) = A \sin(2\pi f t + \theta_0),
\end{equation}
where $V(t)$ is the value of the sine at a given time $t$, $A$ is the amplitude, $f$ is the frequency, in Hz, and $\theta_0$ is the initial phase, in radians.

In order to represent the parameters of different sine curves, we add a subscript with a number to each of the variables above. This way, if we are talking about the first sine wave parameters we would use $A_1$, $f_1$, and $\theta_{0,1}$ to represent its amplitude, frequency and initial phase, respectively.

\section{Proposed Approach}
We propose to show 2 different sine waves, and visualize the difference in frequency and amplitude in a graphical manner.

\section{Results}

We have built a Matlab simulation to show this differences. Table~\ref{tb:params} contains the simulation parameters.

\begin{table}[h!]
    \centering
\caption{Simulation Parameters}
\label{tb:params}
\setlength{\tabcolsep}{25pt}
\def\arraystretch{1.5}
\begin{threeparttable}
    \begin{tabular}{@{}lr@{}}
        \toprule
        \textbf{Parameter} & \textbf{Value}                                                       \\
        \midrule
        Simulation Time~($t$)                             & 100 ms                                \\
        Sampling Frequency~($f_\text{s}$)                 & 44000 Hz~\cite{Roy2002}               \\
        Amplitude Sine 1~($A_1$)                          & 2 V                                   \\
        Amplitude Sine 2~($A_2$)                          & 1.5 V                                 \\
        Frequency Sine 1~($f_1$)                          & 60 Hz                                 \\
        Frequency Sine 2~($f_2$)                          & 80 Hz                                 \\
        Initial Phase Sine 1~($\theta_{0,1}$)             & 0 radians                             \\
        Initial Phase Sine 2~($\theta_{0,2}$)             & 1.57 radians                          \\
        \bottomrule
    \end{tabular}
\end{threeparttable}

\end{table}

As we can see in Fig.~\ref{fig:sine_waves}, the two sine waves are different, depending on $f$, $A$ and $\theta_0$.

\begin{figure}[t!]
    \centering
    \resizebox{\columnwidth}{!}{% This file was created by matlab2tikz.
%
%The latest updates can be retrieved from
%  http://www.mathworks.com/matlabcentral/fileexchange/22022-matlab2tikz-matlab2tikz
%where you can also make suggestions and rate matlab2tikz.
%
\definecolor{mycolor1}{rgb}{0.00000,0.44700,0.74100}%
\definecolor{mycolor2}{rgb}{0.85000,0.32500,0.09800}%
%
\begin{tikzpicture}

\begin{axis}[%
width=4.521in,
height=3.566in,
at={(0.758in,0.481in)},
scale only axis,
xmin=0,
xmax=100,
xlabel style={font=\color{white!15!black}},
xlabel={Time (ms)},
ymin=-2,
ymax=2,
ylabel style={font=\color{white!15!black}},
ylabel={Value $V(t$) (V)},
axis background/.style={fill=white},
xmajorgrids,
ymajorgrids,
legend style={legend cell align=left, align=left, draw=white!15!black}
]
\addplot [color=mycolor1, dashed, line width=2.0pt, forget plot]
  table[row sep=crcr]{%
0	0\\
0.0227272727272727	0.0171357502701698\\
0.0454545454545454	0.0342702426077983\\
0.0681818181818182	0.0514022191726887\\
0.0909090909090909	0.0685304223093268\\
0.113636363636364	0.0856535946392049\\
0.136363636363636	0.102770479153126\\
0.159090909090909	0.119879819303481\\
0.181818181818182	0.136980359096491\\
0.204545454545455	0.15407084318441\\
0.227272727272727	0.171150016957679\\
0.25	0.188216626637029\\
0.272727272727273	0.205269419365516\\
0.295454545454545	0.222307143300498\\
0.318181818181818	0.239328547705532\\
0.340909090909091	0.256332383042185\\
0.363636363636364	0.273317401061771\\
0.386363636363636	0.290282354896977\\
0.409090909090909	0.307225999153399\\
0.431818181818182	0.324147090000966\\
0.454545454545455	0.341044385265248\\
0.477272727272727	0.357916644518646\\
0.5	0.374762629171449\\
0.522727272727273	0.391581102562763\\
0.545454545454545	0.408370830051286\\
0.568181818181818	0.425130579105953\\
0.590909090909091	0.441859119396408\\
0.613636363636364	0.458555222883325\\
0.636363636363636	0.47521766390856\\
0.659090909090909	0.491845219285123\\
0.681818181818182	0.508436668386974\\
0.704545454545455	0.524990793238629\\
0.727272727272727	0.54150637860457\\
0.75	0.557982212078458\\
0.772727272727273	0.574417084172133\\
0.795454545454545	0.5908097884044\\
0.818181818181818	0.607159121389604\\
0.840909090909091	0.623463882925964\\
0.863636363636364	0.639722876083682\\
0.886363636363636	0.655934907292807\\
0.909090909090909	0.67209878643086\\
0.931818181818182	0.688213326910195\\
0.954545454545454	0.70427734576511\\
0.977272727272727	0.720289663738685\\
1	0.736249105369356\\
1.02272727272727	0.752154499077199\\
1.04545454545455	0.768004677249943\\
1.06818181818182	0.783798476328676\\
1.09090909090909	0.799534736893267\\
1.11363636363636	0.815212303747481\\
1.13636363636364	0.830830026003773\\
1.15909090909091	0.846386757167783\\
1.18181818181818	0.861881355222498\\
1.20454545454545	0.877312682712082\\
1.22727272727273	0.892679606825384\\
1.25	0.907980999479094\\
1.27272727272727	0.923215737400552\\
1.29545454545455	0.938382702210215\\
1.31818181818182	0.95348078050375\\
1.34090909090909	0.968508863933772\\
1.36363636363636	0.983465849291207\\
1.38636363636364	0.998350638586281\\
1.40909090909091	1.01316213912912\\
1.43181818181818	1.02789926360995\\
1.45454545454545	1.04256093017897\\
1.47727272727273	1.05714606252568\\
1.5	1.07165358995799\\
1.52272727272727	1.08608244748075\\
1.54545454545455	1.10043157587395\\
1.56818181818182	1.11469992177051\\
1.59090909090909	1.12888643773354\\
1.61363636363636	1.1429900823333\\
1.63636363636364	1.15700982022362\\
1.65909090909091	1.1709446222179\\
1.68181818181818	1.18479346536467\\
1.70454545454545	1.19855533302269\\
1.72727272727273	1.21222921493558\\
1.75	1.22581410730595\\
1.77272727272727	1.23930901286916\\
1.79545454545455	1.25271294096646\\
1.81818181818182	1.26602490761774\\
1.84090909090909	1.27924393559378\\
1.86363636363636	1.29236905448795\\
1.88636363636364	1.30539930078749\\
1.90909090909091	1.31833371794421\\
1.93181818181818	1.33117135644474\\
1.95454545454545	1.34391127388018\\
1.97727272727273	1.35655253501534\\
2	1.36909421185738\\
2.02272727272727	1.3815353837239\\
2.04545454545454	1.39387513731059\\
2.06818181818182	1.4061125667582\\
2.09090909090909	1.41824677371909\\
2.11363636363636	1.43027686742319\\
2.13636363636364	1.44220196474334\\
2.15909090909091	1.45402119026015\\
2.18181818181818	1.46573367632626\\
2.20454545454545	1.47733856313006\\
2.22727272727273	1.48883499875874\\
2.25	1.50022213926092\\
2.27272727272727	1.51149914870852\\
2.29545454545455	1.52266519925817\\
2.31818181818182	1.533719471212\\
2.34090909090909	1.54466115307774\\
2.36363636363636	1.5554894416284\\
2.38636363636364	1.56620354196111\\
2.40909090909091	1.57680266755559\\
2.43181818181818	1.5872860403318\\
2.45454545454545	1.59765289070713\\
2.47727272727273	1.60790245765282\\
2.5	1.61803398874989\\
2.52272727272727	1.62804674024436\\
2.54545454545454	1.63793997710183\\
2.56818181818182	1.64771297306143\\
2.59090909090909	1.65736501068918\\
2.61363636363636	1.66689538143064\\
2.63636363636364	1.67630338566287\\
2.65909090909091	1.68558833274589\\
2.68181818181818	1.69474954107328\\
2.70454545454545	1.7037863381223\\
2.72727272727273	1.71269806050318\\
2.75	1.72148405400789\\
2.77272727272727	1.73014367365812\\
2.79545454545455	1.73867628375266\\
2.81818181818182	1.74708125791404\\
2.84090909090909	1.75535797913451\\
2.86363636363636	1.76350583982137\\
2.88636363636364	1.77152424184154\\
2.90909090909091	1.77941259656547\\
2.93181818181818	1.78717032491037\\
2.95454545454545	1.79479685738272\\
2.97727272727273	1.80229163412004\\
3	1.80965410493204\\
3.02272727272727	1.81688372934098\\
3.04545454545455	1.82397997662134\\
3.06818181818182	1.83094232583881\\
3.09090909090909	1.83777026588851\\
3.11363636363636	1.8444632955325\\
3.13636363636364	1.85102092343661\\
3.15909090909091	1.85744266820648\\
3.18181818181818	1.8637280584229\\
3.20454545454545	1.86987663267645\\
3.22727272727273	1.8758879396013\\
3.25	1.88176153790845\\
3.27272727272727	1.88749699641802\\
3.29545454545455	1.89309389409098\\
3.31818181818182	1.89855182006\\
3.34090909090909	1.90387037365967\\
3.36363636363636	1.90904916445585\\
3.38636363636364	1.91408781227439\\
3.40909090909091	1.91898594722899\\
3.43181818181818	1.92374320974841\\
3.45454545454545	1.92835925060279\\
3.47727272727273	1.93283373092936\\
3.5	1.93716632225726\\
3.52272727272727	1.94135670653171\\
3.54545454545455	1.94540457613728\\
3.56818181818182	1.94930963392057\\
3.59090909090909	1.95307159321193\\
3.61363636363636	1.95669017784656\\
3.63636363636364	1.96016512218479\\
3.65909090909091	1.96349617113152\\
3.68181818181818	1.96668308015502\\
3.70454545454545	1.96972561530483\\
3.72727272727273	1.97262355322896\\
3.75	1.97537668119028\\
3.77272727272727	1.97798479708211\\
3.79545454545455	1.98044770944313\\
3.81818181818182	1.98276523747134\\
3.84090909090909	1.9849372110374\\
3.86363636363636	1.9869634706971\\
3.88636363636364	1.98884386770306\\
3.90909090909091	1.99057826401565\\
3.93181818181818	1.99216653231312\\
3.95454545454545	1.99360855600098\\
3.97727272727273	1.99490422922051\\
4	1.99605345685654\\
4.02272727272727	1.99705615454448\\
4.04545454545454	1.99791224867646\\
4.06818181818182	1.99862167640675\\
4.09090909090909	1.99918438565638\\
4.11363636363636	1.99960033511696\\
4.13636363636364	1.9998694942537\\
4.15909090909091	1.99999184330769\\
4.18181818181818	1.9999673732973\\
4.20454545454545	1.99979608601887\\
4.22727272727273	1.99947799404656\\
4.25	1.99901312073146\\
4.27272727272727	1.99840150019984\\
4.29545454545454	1.99764317735066\\
4.31818181818182	1.99673820785227\\
4.34090909090909	1.99568665813833\\
4.36363636363636	1.99448860540293\\
4.38636363636364	1.99314413759491\\
4.40909090909091	1.99165335341144\\
4.43181818181818	1.99001636229072\\
4.45454545454545	1.988233284404\\
4.47727272727273	1.98630425064673\\
4.5	1.98422940262896\\
4.52272727272727	1.98200889266491\\
4.54545454545454	1.97964288376187\\
4.56818181818182	1.97713154960813\\
4.59090909090909	1.97447507456033\\
4.61363636363636	1.97167365362984\\
4.63636363636364	1.9687274924685\\
4.65909090909091	1.96563680735351\\
4.68181818181818	1.96240182517155\\
4.70454545454545	1.9590227834021\\
4.72727272727273	1.95549993010006\\
4.75	1.95183352387749\\
4.77272727272727	1.94802383388467\\
4.79545454545455	1.94407113979028\\
4.81818181818182	1.93997573176093\\
4.84090909090909	1.93573791043985\\
4.86363636363636	1.93135798692478\\
4.88636363636364	1.92683628274516\\
4.90909090909091	1.92217312983854\\
4.93181818181818	1.91736887052619\\
4.95454545454545	1.91242385748797\\
4.97727272727273	1.90733845373647\\
5	1.90211303259031\\
5.02272727272727	1.89674797764677\\
5.04545454545454	1.89124368275365\\
5.06818181818182	1.8856005519803\\
5.09090909090909	1.87981899958799\\
5.11363636363636	1.87389944999952\\
5.13636363636364	1.86784233776803\\
5.15909090909091	1.8616481075451\\
5.18181818181818	1.85531721404814\\
5.20454545454545	1.84885012202696\\
5.22727272727273	1.8422473062297\\
5.25	1.83550925136796\\
5.27272727272727	1.82863645208122\\
5.29545454545454	1.8216294129005\\
5.31818181818182	1.81448864821138\\
5.34090909090909	1.80721468221618\\
5.36363636363636	1.79980804889554\\
5.38636363636364	1.79226929196914\\
5.40909090909091	1.78459896485588\\
5.43181818181818	1.77679763063316\\
5.45454545454545	1.76886586199563\\
5.47727272727273	1.76080424121308\\
5.5	1.75261336008773\\
5.52272727272727	1.74429381991079\\
5.54545454545454	1.7358462314183\\
5.56818181818182	1.72727121474631\\
5.59090909090909	1.71856939938536\\
5.61363636363636	1.70974142413423\\
5.63636363636364	1.70078793705311\\
5.65909090909091	1.69170959541598\\
5.68181818181818	1.68250706566236\\
5.70454545454545	1.67318102334842\\
5.72727272727273	1.66373215309735\\
5.75	1.65416114854912\\
5.77272727272727	1.64446871230957\\
5.79545454545455	1.63465555589881\\
5.81818181818182	1.624722399699\\
5.84090909090909	1.61466997290146\\
5.86363636363636	1.60449901345315\\
5.88636363636364	1.59421026800249\\
5.90909090909091	1.58380449184455\\
5.93181818181818	1.57328244886561\\
5.95454545454545	1.56264491148706\\
5.97727272727273	1.55189266060874\\
6	1.54102648555158\\
6.02272727272727	1.53004718399965\\
6.04545454545454	1.51895556194163\\
6.06818181818182	1.50775243361162\\
6.09090909090909	1.4964386214294\\
6.11363636363636	1.48501495594001\\
6.13636363636364	1.47348227575281\\
6.15909090909091	1.46184142747991\\
6.18181818181818	1.45009326567401\\
6.20454545454545	1.43823865276571\\
6.22727272727273	1.42627845900014\\
6.25	1.4142135623731\\
6.27272727272727	1.40204484856662\\
6.29545454545454	1.38977321088395\\
6.31818181818182	1.37739955018392\\
6.34090909090909	1.3649247748149\\
6.36363636363636	1.35234980054804\\
6.38636363636364	1.33967555051008\\
6.40909090909091	1.32690295511558\\
6.43181818181818	1.31403295199863\\
6.45454545454545	1.30106648594397\\
6.47727272727273	1.28800450881771\\
6.5	1.27484797949738\\
6.52272727272727	1.26159786380158\\
6.54545454545454	1.24825513441908\\
6.56818181818182	1.23482077083739\\
6.59090909090909	1.22129575927088\\
6.61363636363636	1.20768109258837\\
6.63636363636364	1.19397777024025\\
6.65909090909091	1.1801867981851\\
6.68181818181818	1.16630918881586\\
6.70454545454545	1.15234596088551\\
6.72727272727273	1.13829813943224\\
6.75	1.12416675570426\\
6.77272727272727	1.10995284708408\\
6.79545454545454	1.09565745701233\\
6.81818181818182	1.0812816349112\\
6.84090909090909	1.06682643610735\\
6.86363636363636	1.0522929217545\\
6.88636363636364	1.03768215875549\\
6.90909090909091	1.02299521968397\\
6.93181818181818	1.00823318270566\\
6.95454545454545	0.993397131499223\\
6.97727272727273	0.978488155176665\\
7	0.963507348203431\\
7.02272727272727	0.948455810318038\\
7.04545454545454	0.933334646451348\\
7.06818181818182	0.918144966645453\\
7.09090909090909	0.902887885972189\\
7.11363636363636	0.887564524451278\\
7.13636363636364	0.872176006968111\\
7.15909090909091	0.856723463191163\\
7.18181818181818	0.84120802748907\\
7.20454545454545	0.825630838847356\\
7.22727272727273	0.809993040784811\\
7.25	0.794295781269562\\
7.27272727272727	0.778540212634783\\
7.29545454545454	0.762727491494116\\
7.31818181818182	0.746858778656757\\
7.34090909090909	0.73093523904224\\
7.36363636363636	0.714958041594929\\
7.38636363636364	0.698928359198197\\
7.40909090909091	0.682847368588329\\
7.43181818181818	0.666716250268137\\
7.45454545454545	0.6505361884203\\
7.47727272727273	0.634308370820434\\
7.5	0.618033988749895\\
7.52272727272727	0.601714236908332\\
7.54545454545454	0.585350313325975\\
7.56818181818182	0.568943419275702\\
7.59090909090909	0.552494759184841\\
7.61363636363636	0.536005540546758\\
7.63636363636364	0.519476973832217\\
7.65909090909091	0.502910272400517\\
7.68181818181818	0.48630665241042\\
7.70454545454545	0.469667332730872\\
7.72727272727273	0.45299353485153\\
7.75	0.436286482793086\\
7.77272727272727	0.419547403017419\\
7.79545454545454	0.402777524337557\\
7.81818181818182	0.385978077827467\\
7.84090909090909	0.369150296731691\\
7.86363636363636	0.352295416374804\\
7.88636363636364	0.335414674070733\\
7.90909090909091	0.31850930903193\\
7.93181818181818	0.301580562278394\\
7.95454545454545	0.28462967654657\\
7.97727272727273	0.26765789619813\\
8	0.250666467128608\\
8.02272727272727	0.233656636675953\\
8.04545454545454	0.21662965352895\\
8.06818181818182	0.199586767635566\\
8.09090909090909	0.182529230111185\\
8.11363636363636	0.165458293146762\\
8.13636363636364	0.148375209916905\\
8.15909090909091	0.131281234487877\\
8.18181818181818	0.114177621725537\\
8.20454545454545	0.0970656272032141\\
8.22727272727273	0.079946507109547\\
8.25	0.0628215181562582\\
8.27272727272727	0.0456919174858994\\
8.29545454545454	0.028558962579575\\
8.31818181818182	0.0114239111646166\\
8.34090909090909	-0.00571197887773458\\
8.36363636363636	-0.0228474496046792\\
8.38636363636364	-0.0399812431041973\\
8.40909090909091	-0.0571121015873917\\
8.43181818181818	-0.0742387674808273\\
8.45454545454546	-0.0913599835188463\\
8.47727272727273	-0.108474492835861\\
8.5	-0.125581039058626\\
8.52272727272727	-0.142678366398464\\
8.54545454545454	-0.159765219743454\\
8.56818181818182	-0.17684034475057\\
8.59090909090909	-0.193902487937762\\
8.61363636363636	-0.210950396775971\\
8.63636363636364	-0.227982819781081\\
8.65909090909091	-0.244998506605786\\
8.68181818181818	-0.261996208131384\\
8.70454545454546	-0.278974676559469\\
8.72727272727273	-0.295932665503532\\
8.75	-0.312868930080461\\
8.77272727272727	-0.329782227001926\\
8.79545454545454	-0.346671314665648\\
8.81818181818182	-0.363534953246546\\
8.84090909090909	-0.380371904787753\\
8.86363636363636	-0.39718093329149\\
8.88636363636364	-0.413960804809809\\
8.90909090909091	-0.430710287535163\\
8.93181818181818	-0.44742815189085\\
8.95454545454546	-0.464113170621263\\
8.97727272727273	-0.480764118881984\\
9	-0.497379774329708\\
9.02272727272727	-0.513958917211968\\
9.04545454545454	-0.530500330456676\\
9.06818181818182	-0.547002799761469\\
9.09090909090909	-0.563465113682859\\
9.11363636363636	-0.579886063725151\\
9.13636363636363	-0.596264444429168\\
9.15909090909091	-0.612599053460743\\
9.18181818181818	-0.628888691698976\\
9.20454545454546	-0.645132163324268\\
9.22727272727273	-0.661328275906101\\
9.25	-0.677475840490582\\
9.27272727272727	-0.693573671687712\\
9.29545454545454	-0.709620587758413\\
9.31818181818182	-0.725615410701281\\
9.34090909090909	-0.741556966339057\\
9.36363636363636	-0.757444084404826\\
9.38636363636363	-0.773275598627925\\
9.40909090909091	-0.789050346819565\\
9.43181818181818	-0.804767170958136\\
9.45454545454545	-0.820424917274222\\
9.47727272727273	-0.836022436335309\\
9.5	-0.851558583130145\\
9.52272727272727	-0.867032217152812\\
9.54545454545454	-0.882442202486441\\
9.56818181818182	-0.897787407886607\\
9.59090909090909	-0.913066706864365\\
9.61363636363636	-0.928278977768947\\
9.63636363636363	-0.943423103870112\\
9.65909090909091	-0.958497973440113\\
9.68181818181818	-0.973502479835312\\
9.70454545454545	-0.988435521577422\\
9.72727272727273	-1.00329600243436\\
9.75	-1.01808283150074\\
9.77272727272727	-1.03279492327792\\
9.79545454545454	-1.04743119775373\\
9.81818181818182	-1.06199058048171\\
9.84090909090909	-1.07647200266005\\
9.86363636363636	-1.09087440120996\\
9.88636363636364	-1.10519671885381\\
9.90909090909091	-1.11943790419265\\
9.93181818181818	-1.13359691178349\\
9.95454545454545	-1.14767270221595\\
9.97727272727273	-1.16166424218864\\
10	-1.17557050458495\\
10.0227272727273	-1.1893904685485\\
10.0454545454545	-1.20312311955808\\
10.0681818181818	-1.21676744950207\\
10.0909090909091	-1.23032245675253\\
10.1136363636364	-1.24378714623864\\
10.1363636363636	-1.25716052951983\\
10.1590909090909	-1.27044162485827\\
10.1818181818182	-1.283629457291\\
10.2045454545455	-1.29672305870146\\
10.2272727272727	-1.30972146789057\\
10.25	-1.3226237306473\\
10.2727272727273	-1.33542889981872\\
10.2954545454545	-1.34813603537952\\
10.3181818181818	-1.36074420450101\\
10.3409090909091	-1.37325248161963\\
10.3636363636364	-1.38565994850485\\
10.3863636363636	-1.39796569432662\\
10.4090909090909	-1.41016881572221\\
10.4318181818182	-1.42226841686254\\
10.4545454545455	-1.43426360951793\\
10.4772727272727	-1.44615351312329\\
10.5	-1.45793725484282\\
10.5227272727273	-1.46961396963402\\
10.5454545454545	-1.48118280031121\\
10.5681818181818	-1.49264289760849\\
10.5909090909091	-1.50399342024205\\
10.6136363636364	-1.51523353497193\\
10.6363636363636	-1.5263624166632\\
10.6590909090909	-1.53737924834653\\
10.6818181818182	-1.54828322127816\\
10.7045454545455	-1.55907353499928\\
10.7272727272727	-1.56974939739478\\
10.75	-1.58031002475138\\
10.7727272727273	-1.59075464181522\\
10.7954545454545	-1.60108248184872\\
10.8181818181818	-1.61129278668688\\
10.8409090909091	-1.62138480679294\\
10.8636363636364	-1.6313578013134\\
10.8863636363636	-1.64121103813242\\
10.9090909090909	-1.65094379392555\\
10.9318181818182	-1.66055535421281\\
10.9545454545455	-1.67004501341118\\
10.9772727272727	-1.67941207488639\\
11	-1.68865585100403\\
11.0227272727273	-1.69777566318005\\
11.0454545454545	-1.70677084193056\\
11.0681818181818	-1.71564072692103\\
11.0909090909091	-1.72438466701468\\
11.1136363636364	-1.73300202032034\\
11.1363636363636	-1.74149215423955\\
11.1590909090909	-1.74985444551303\\
11.1818181818182	-1.75808828026637\\
11.2045454545455	-1.76619305405515\\
11.2272727272727	-1.77416817190929\\
11.25	-1.78201304837674\\
11.2727272727273	-1.78972710756642\\
11.2954545454545	-1.79730978319057\\
11.3181818181818	-1.80476051860623\\
11.3409090909091	-1.81207876685618\\
11.3636363636364	-1.81926399070904\\
11.3863636363636	-1.82631566269874\\
11.4090909090909	-1.83323326516322\\
11.4318181818182	-1.84001629028247\\
11.4545454545455	-1.84666424011573\\
11.4772727272727	-1.85317662663814\\
11.5	-1.8595529717765\\
11.5227272727273	-1.86579280744439\\
11.5454545454545	-1.87189567557651\\
11.5681818181818	-1.87786112816236\\
11.5909090909091	-1.88368872727905\\
11.6136363636364	-1.88937804512352\\
11.6363636363636	-1.89492866404391\\
11.6590909090909	-1.9003401765702\\
11.6818181818182	-1.9056121854442\\
11.7045454545455	-1.91074430364859\\
11.7272727272727	-1.91573615443547\\
11.75	-1.92058737135389\\
11.7727272727273	-1.92529759827684\\
11.7954545454545	-1.92986648942736\\
11.8181818181818	-1.93429370940391\\
11.8409090909091	-1.93857893320503\\
11.8636363636364	-1.94272184625316\\
11.8863636363636	-1.94672214441774\\
11.9090909090909	-1.95057953403757\\
11.9318181818182	-1.95429373194232\\
11.9545454545455	-1.95786446547334\\
11.9772727272727	-1.9612914725037\\
12	-1.96457450145738\\
12.0227272727273	-1.96771331132777\\
12.0454545454545	-1.97070767169539\\
12.0681818181818	-1.97355736274472\\
12.0909090909091	-1.97626217528044\\
12.1136363636364	-1.97882191074272\\
12.1363636363636	-1.9812363812218\\
12.1590909090909	-1.98350540947181\\
12.1818181818182	-1.98562882892376\\
12.2045454545455	-1.98760648369781\\
12.2272727272727	-1.98943822861464\\
12.25	-1.99112392920616\\
12.2727272727273	-1.99266346172538\\
12.2954545454545	-1.99405671315548\\
12.3181818181818	-1.99530358121811\\
12.3409090909091	-1.99640397438088\\
12.3636363636364	-1.99735781186412\\
12.3863636363636	-1.99816502364679\\
12.4090909090909	-1.9988255504716\\
12.4318181818182	-1.99933934384939\\
12.4545454545455	-1.99970636606268\\
12.4772727272727	-1.99992659016842\\
12.5	-2\\
12.5227272727273	-1.99992659016842\\
12.5454545454545	-1.99970636606268\\
12.5681818181818	-1.99933934384939\\
12.5909090909091	-1.9988255504716\\
12.6136363636364	-1.99816502364679\\
12.6363636363636	-1.99735781186412\\
12.6590909090909	-1.99640397438088\\
12.6818181818182	-1.99530358121811\\
12.7045454545455	-1.99405671315548\\
12.7272727272727	-1.99266346172538\\
12.75	-1.99112392920616\\
12.7727272727273	-1.98943822861464\\
12.7954545454545	-1.98760648369781\\
12.8181818181818	-1.98562882892376\\
12.8409090909091	-1.98350540947181\\
12.8636363636364	-1.9812363812218\\
12.8863636363636	-1.97882191074272\\
12.9090909090909	-1.97626217528045\\
12.9318181818182	-1.97355736274472\\
12.9545454545455	-1.97070767169539\\
12.9772727272727	-1.96771331132777\\
13	-1.96457450145738\\
13.0227272727273	-1.9612914725037\\
13.0454545454545	-1.95786446547334\\
13.0681818181818	-1.95429373194232\\
13.0909090909091	-1.95057953403757\\
13.1136363636364	-1.94672214441775\\
13.1363636363636	-1.94272184625316\\
13.1590909090909	-1.93857893320503\\
13.1818181818182	-1.93429370940391\\
13.2045454545455	-1.92986648942736\\
13.2272727272727	-1.92529759827684\\
13.25	-1.92058737135389\\
13.2727272727273	-1.91573615443547\\
13.2954545454545	-1.91074430364859\\
13.3181818181818	-1.9056121854442\\
13.3409090909091	-1.90034017657021\\
13.3636363636364	-1.89492866404391\\
13.3863636363636	-1.88937804512352\\
13.4090909090909	-1.88368872727905\\
13.4318181818182	-1.87786112816236\\
13.4545454545455	-1.87189567557651\\
13.4772727272727	-1.86579280744439\\
13.5	-1.8595529717765\\
13.5227272727273	-1.85317662663814\\
13.5454545454545	-1.84666424011573\\
13.5681818181818	-1.84001629028247\\
13.5909090909091	-1.83323326516322\\
13.6136363636364	-1.82631566269874\\
13.6363636363636	-1.81926399070904\\
13.6590909090909	-1.81207876685618\\
13.6818181818182	-1.80476051860623\\
13.7045454545455	-1.79730978319057\\
13.7272727272727	-1.78972710756642\\
13.75	-1.78201304837674\\
13.7727272727273	-1.77416817190929\\
13.7954545454545	-1.76619305405515\\
13.8181818181818	-1.75808828026637\\
13.8409090909091	-1.74985444551303\\
13.8636363636364	-1.74149215423956\\
13.8863636363636	-1.73300202032034\\
13.9090909090909	-1.72438466701468\\
13.9318181818182	-1.71564072692103\\
13.9545454545455	-1.70677084193056\\
13.9772727272727	-1.69777566318005\\
14	-1.68865585100403\\
14.0227272727273	-1.67941207488639\\
14.0454545454545	-1.67004501341119\\
14.0681818181818	-1.66055535421281\\
14.0909090909091	-1.65094379392555\\
14.1136363636364	-1.64121103813243\\
14.1363636363636	-1.6313578013134\\
14.1590909090909	-1.62138480679294\\
14.1818181818182	-1.61129278668688\\
14.2045454545455	-1.60108248184872\\
14.2272727272727	-1.59075464181522\\
14.25	-1.58031002475138\\
14.2727272727273	-1.56974939739478\\
14.2954545454545	-1.55907353499928\\
14.3181818181818	-1.54828322127817\\
14.3409090909091	-1.53737924834653\\
14.3636363636364	-1.5263624166632\\
14.3863636363636	-1.51523353497193\\
14.4090909090909	-1.50399342024205\\
14.4318181818182	-1.49264289760849\\
14.4545454545455	-1.48118280031121\\
14.4772727272727	-1.46961396963402\\
14.5	-1.45793725484282\\
14.5227272727273	-1.44615351312329\\
14.5454545454545	-1.43426360951793\\
14.5681818181818	-1.42226841686254\\
14.5909090909091	-1.41016881572221\\
14.6136363636364	-1.39796569432662\\
14.6363636363636	-1.38565994850485\\
14.6590909090909	-1.37325248161963\\
14.6818181818182	-1.36074420450101\\
14.7045454545455	-1.34813603537952\\
14.7272727272727	-1.33542889981873\\
14.75	-1.3226237306473\\
14.7727272727273	-1.30972146789057\\
14.7954545454545	-1.29672305870146\\
14.8181818181818	-1.28362945729101\\
14.8409090909091	-1.27044162485828\\
14.8636363636364	-1.25716052951983\\
14.8863636363636	-1.24378714623865\\
14.9090909090909	-1.23032245675253\\
14.9318181818182	-1.21676744950207\\
14.9545454545455	-1.20312311955808\\
14.9772727272727	-1.1893904685485\\
15	-1.17557050458495\\
15.0227272727273	-1.16166424218864\\
15.0454545454545	-1.14767270221596\\
15.0681818181818	-1.13359691178349\\
15.0909090909091	-1.11943790419265\\
15.1136363636364	-1.10519671885381\\
15.1363636363636	-1.09087440120996\\
15.1590909090909	-1.07647200266005\\
15.1818181818182	-1.06199058048172\\
15.2045454545455	-1.04743119775373\\
15.2272727272727	-1.03279492327792\\
15.25	-1.01808283150074\\
15.2727272727273	-1.00329600243437\\
15.2954545454545	-0.988435521577423\\
15.3181818181818	-0.973502479835313\\
15.3409090909091	-0.958497973440114\\
15.3636363636364	-0.943423103870115\\
15.3863636363636	-0.92827897776895\\
15.4090909090909	-0.913066706864366\\
15.4318181818182	-0.89778740788661\\
15.4545454545455	-0.882442202486444\\
15.4772727272727	-0.867032217152814\\
15.5	-0.851558583130146\\
15.5227272727273	-0.83602243633531\\
15.5454545454545	-0.820424917274225\\
15.5681818181818	-0.804767170958135\\
15.5909090909091	-0.789050346819567\\
15.6136363636364	-0.773275598627928\\
15.6363636363636	-0.757444084404827\\
15.6590909090909	-0.74155696633906\\
15.6818181818182	-0.725615410701283\\
15.7045454545455	-0.709620587758415\\
15.7272727272727	-0.693573671687714\\
15.75	-0.677475840490583\\
15.7727272727273	-0.661328275906103\\
15.7954545454545	-0.64513216332427\\
15.8181818181818	-0.628888691698977\\
15.8409090909091	-0.612599053460744\\
15.8636363636364	-0.596264444429172\\
15.8863636363636	-0.579886063725153\\
15.9090909090909	-0.56346511368286\\
15.9318181818182	-0.54700279976147\\
15.9545454545455	-0.530500330456676\\
15.9772727272727	-0.51395891721197\\
16	-0.497379774329709\\
16.0227272727273	-0.480764118881986\\
16.0454545454545	-0.464113170621265\\
16.0681818181818	-0.447428151890851\\
16.0909090909091	-0.430710287535166\\
16.1136363636364	-0.413960804809809\\
16.1363636363636	-0.397180933291492\\
16.1590909090909	-0.380371904787754\\
16.1818181818182	-0.36353495324655\\
16.2045454545455	-0.346671314665648\\
16.2272727272727	-0.329782227001928\\
16.25	-0.312868930080462\\
16.2727272727273	-0.295932665503534\\
16.2954545454545	-0.278974676559473\\
16.3181818181818	-0.261996208131387\\
16.3409090909091	-0.244998506605789\\
16.3636363636364	-0.227982819781083\\
16.3863636363636	-0.210950396775972\\
16.4090909090909	-0.193902487937763\\
16.4318181818182	-0.176840344750574\\
16.4545454545455	-0.159765219743456\\
16.4772727272727	-0.142678366398466\\
16.5	-0.12558103905863\\
16.5227272727273	-0.108474492835864\\
16.5454545454545	-0.0913599835188471\\
16.5681818181818	-0.074238767480829\\
16.5909090909091	-0.0571121015873933\\
16.6136363636364	-0.0399812431041989\\
16.6363636363636	-0.02284744960468\\
16.6590909090909	-0.00571197887773709\\
16.6818181818182	0.011423911164615\\
16.7045454545455	0.0285589625795742\\
16.7272727272727	0.0456919174858968\\
16.75	0.0628215181562557\\
16.7727272727273	0.0799465071095453\\
16.7954545454545	0.0970656272032125\\
16.8181818181818	0.114177621725534\\
16.8409090909091	0.131281234487877\\
16.8636363636364	0.148375209916902\\
16.8863636363636	0.16545829314676\\
16.9090909090909	0.182529230111183\\
16.9318181818182	0.199586767635563\\
16.9545454545455	0.216629653528948\\
16.9772727272727	0.23365663667595\\
17	0.250666467128607\\
17.0227272727273	0.26765789619813\\
17.0454545454545	0.28462967654657\\
17.0681818181818	0.30158056227839\\
17.0909090909091	0.318509309031928\\
17.1136363636364	0.335414674070731\\
17.1363636363636	0.3522954163748\\
17.1590909090909	0.369150296731689\\
17.1818181818182	0.385978077827465\\
17.2045454545455	0.402777524337555\\
17.2272727272727	0.419547403017417\\
17.25	0.436286482793084\\
17.2727272727273	0.452993534851529\\
17.2954545454545	0.46966733273087\\
17.3181818181818	0.486306652410417\\
17.3409090909091	0.502910272400516\\
17.3636363636364	0.519476973832215\\
17.3863636363636	0.536005540546757\\
17.4090909090909	0.552494759184839\\
17.4318181818182	0.5689434192757\\
17.4545454545455	0.585350313325973\\
17.4772727272727	0.60171423690833\\
17.5	0.618033988749893\\
17.5227272727273	0.634308370820434\\
17.5454545454545	0.650536188420299\\
17.5681818181818	0.666716250268133\\
17.5909090909091	0.682847368588327\\
17.6136363636364	0.698928359198195\\
17.6363636363636	0.714958041594927\\
17.6590909090909	0.730935239042239\\
17.6818181818182	0.746858778656755\\
17.7045454545455	0.762727491494113\\
17.7272727272727	0.778540212634782\\
17.75	0.79429578126956\\
17.7727272727273	0.809993040784811\\
17.7954545454545	0.825630838847352\\
17.8181818181818	0.841208027489068\\
17.8409090909091	0.856723463191162\\
17.8636363636364	0.87217600696811\\
17.8863636363636	0.887564524451277\\
17.9090909090909	0.902887885972188\\
17.9318181818182	0.91814496664545\\
17.9545454545455	0.933334646451345\\
17.9772727272727	0.948455810318037\\
18	0.963507348203428\\
18.0227272727273	0.978488155176663\\
18.0454545454545	0.993397131499221\\
18.0681818181818	1.00823318270566\\
18.0909090909091	1.02299521968397\\
18.1136363636364	1.03768215875549\\
18.1363636363636	1.05229292175449\\
18.1590909090909	1.06682643610735\\
18.1818181818182	1.08128163491119\\
18.2045454545455	1.09565745701233\\
18.2272727272727	1.10995284708408\\
18.25	1.12416675570426\\
18.2727272727273	1.13829813943223\\
18.2954545454545	1.15234596088551\\
18.3181818181818	1.16630918881586\\
18.3409090909091	1.1801867981851\\
18.3636363636364	1.19397777024025\\
18.3863636363636	1.20768109258836\\
18.4090909090909	1.22129575927088\\
18.4318181818182	1.23482077083739\\
18.4545454545455	1.24825513441908\\
18.4772727272727	1.26159786380158\\
18.5	1.27484797949738\\
18.5227272727273	1.28800450881771\\
18.5454545454545	1.30106648594397\\
18.5681818181818	1.31403295199863\\
18.5909090909091	1.32690295511558\\
18.6136363636364	1.33967555051008\\
18.6363636363636	1.35234980054804\\
18.6590909090909	1.3649247748149\\
18.6818181818182	1.37739955018392\\
18.7045454545455	1.38977321088394\\
18.7272727272727	1.40204484856662\\
18.75	1.41421356237309\\
18.7727272727273	1.42627845900013\\
18.7954545454545	1.43823865276571\\
18.8181818181818	1.45009326567401\\
18.8409090909091	1.4618414274799\\
18.8636363636364	1.47348227575281\\
18.8863636363636	1.48501495594001\\
18.9090909090909	1.4964386214294\\
18.9318181818182	1.50775243361162\\
18.9545454545455	1.51895556194162\\
18.9772727272727	1.53004718399964\\
19	1.54102648555158\\
19.0227272727273	1.55189266060874\\
19.0454545454545	1.56264491148706\\
19.0681818181818	1.57328244886561\\
19.0909090909091	1.58380449184455\\
19.1136363636364	1.59421026800249\\
19.1363636363636	1.60449901345315\\
19.1590909090909	1.61466997290146\\
19.1818181818182	1.624722399699\\
19.2045454545455	1.63465555589881\\
19.2272727272727	1.64446871230957\\
19.25	1.65416114854912\\
19.2727272727273	1.66373215309735\\
19.2954545454545	1.67318102334842\\
19.3181818181818	1.68250706566236\\
19.3409090909091	1.69170959541598\\
19.3636363636364	1.70078793705311\\
19.3863636363636	1.70974142413423\\
19.4090909090909	1.71856939938536\\
19.4318181818182	1.72727121474631\\
19.4545454545455	1.7358462314183\\
19.4772727272727	1.74429381991079\\
19.5	1.75261336008773\\
19.5227272727273	1.76080424121307\\
19.5454545454545	1.76886586199563\\
19.5681818181818	1.77679763063316\\
19.5909090909091	1.78459896485588\\
19.6136363636364	1.79226929196914\\
19.6363636363636	1.79980804889554\\
19.6590909090909	1.80721468221618\\
19.6818181818182	1.81448864821138\\
19.7045454545455	1.8216294129005\\
19.7272727272727	1.82863645208121\\
19.75	1.83550925136796\\
19.7727272727273	1.8422473062297\\
19.7954545454545	1.84885012202696\\
19.8181818181818	1.85531721404814\\
19.8409090909091	1.8616481075451\\
19.8636363636364	1.86784233776803\\
19.8863636363636	1.87389944999952\\
19.9090909090909	1.87981899958799\\
19.9318181818182	1.8856005519803\\
19.9545454545455	1.89124368275365\\
19.9772727272727	1.89674797764677\\
20	1.90211303259031\\
20.0227272727273	1.90733845373647\\
20.0454545454545	1.91242385748797\\
20.0681818181818	1.91736887052619\\
20.0909090909091	1.92217312983854\\
20.1136363636364	1.92683628274516\\
20.1363636363636	1.93135798692478\\
20.1590909090909	1.93573791043985\\
20.1818181818182	1.93997573176093\\
20.2045454545455	1.94407113979028\\
20.2272727272727	1.94802383388467\\
20.25	1.95183352387749\\
20.2727272727273	1.95549993010006\\
20.2954545454545	1.9590227834021\\
20.3181818181818	1.96240182517155\\
20.3409090909091	1.96563680735351\\
20.3636363636364	1.9687274924685\\
20.3863636363636	1.97167365362984\\
20.4090909090909	1.97447507456033\\
20.4318181818182	1.97713154960813\\
20.4545454545455	1.97964288376187\\
20.4772727272727	1.98200889266491\\
20.5	1.98422940262896\\
20.5227272727273	1.98630425064673\\
20.5454545454545	1.988233284404\\
20.5681818181818	1.99001636229072\\
20.5909090909091	1.99165335341144\\
20.6136363636364	1.99314413759491\\
20.6363636363636	1.99448860540293\\
20.6590909090909	1.99568665813833\\
20.6818181818182	1.99673820785227\\
20.7045454545455	1.99764317735066\\
20.7272727272727	1.99840150019984\\
20.75	1.99901312073146\\
20.7727272727273	1.99947799404656\\
20.7954545454545	1.99979608601887\\
20.8181818181818	1.9999673732973\\
20.8409090909091	1.99999184330769\\
20.8636363636364	1.9998694942537\\
20.8863636363636	1.99960033511696\\
20.9090909090909	1.99918438565638\\
20.9318181818182	1.99862167640675\\
20.9545454545455	1.99791224867646\\
20.9772727272727	1.99705615454448\\
21	1.99605345685654\\
21.0227272727273	1.99490422922051\\
21.0454545454545	1.99360855600098\\
21.0681818181818	1.99216653231312\\
21.0909090909091	1.99057826401565\\
21.1136363636364	1.98884386770306\\
21.1363636363636	1.9869634706971\\
21.1590909090909	1.9849372110374\\
21.1818181818182	1.98276523747134\\
21.2045454545455	1.98044770944313\\
21.2272727272727	1.97798479708211\\
21.25	1.97537668119028\\
21.2727272727273	1.97262355322896\\
21.2954545454545	1.96972561530483\\
21.3181818181818	1.96668308015502\\
21.3409090909091	1.96349617113152\\
21.3636363636364	1.96016512218479\\
21.3863636363636	1.95669017784656\\
21.4090909090909	1.95307159321193\\
21.4318181818182	1.94930963392057\\
21.4545454545455	1.94540457613728\\
21.4772727272727	1.94135670653171\\
21.5	1.93716632225726\\
21.5227272727273	1.93283373092936\\
21.5454545454545	1.92835925060279\\
21.5681818181818	1.92374320974841\\
21.5909090909091	1.91898594722899\\
21.6136363636364	1.91408781227439\\
21.6363636363636	1.90904916445585\\
21.6590909090909	1.90387037365967\\
21.6818181818182	1.89855182006\\
21.7045454545455	1.89309389409098\\
21.7272727272727	1.88749699641802\\
21.75	1.88176153790845\\
21.7727272727273	1.87588793960131\\
21.7954545454545	1.86987663267645\\
21.8181818181818	1.86372805842291\\
21.8409090909091	1.85744266820648\\
21.8636363636364	1.85102092343661\\
21.8863636363636	1.8444632955325\\
21.9090909090909	1.83777026588851\\
21.9318181818182	1.83094232583881\\
21.9545454545455	1.82397997662134\\
21.9772727272727	1.81688372934098\\
22	1.80965410493204\\
22.0227272727273	1.80229163412004\\
22.0454545454545	1.79479685738272\\
22.0681818181818	1.78717032491037\\
22.0909090909091	1.77941259656547\\
22.1136363636364	1.77152424184154\\
22.1363636363636	1.76350583982137\\
22.1590909090909	1.75535797913451\\
22.1818181818182	1.74708125791404\\
22.2045454545455	1.73867628375266\\
22.2272727272727	1.73014367365812\\
22.25	1.72148405400789\\
22.2727272727273	1.71269806050318\\
22.2954545454545	1.7037863381223\\
22.3181818181818	1.69474954107329\\
22.3409090909091	1.68558833274589\\
22.3636363636364	1.67630338566287\\
22.3863636363636	1.66689538143064\\
22.4090909090909	1.65736501068918\\
22.4318181818182	1.64771297306143\\
22.4545454545455	1.63793997710183\\
22.4772727272727	1.62804674024437\\
22.5	1.6180339887499\\
22.5227272727273	1.60790245765282\\
22.5454545454545	1.59765289070713\\
22.5681818181818	1.5872860403318\\
22.5909090909091	1.57680266755559\\
22.6136363636364	1.56620354196111\\
22.6363636363636	1.5554894416284\\
22.6590909090909	1.54466115307775\\
22.6818181818182	1.533719471212\\
22.7045454545455	1.52266519925817\\
22.7272727272727	1.51149914870852\\
22.75	1.50022213926092\\
22.7727272727273	1.48883499875875\\
22.7954545454545	1.47733856313006\\
22.8181818181818	1.46573367632627\\
22.8409090909091	1.45402119026015\\
22.8636363636364	1.44220196474334\\
22.8863636363636	1.43027686742319\\
22.9090909090909	1.41824677371909\\
22.9318181818182	1.4061125667582\\
22.9545454545455	1.39387513731059\\
22.9772727272727	1.3815353837239\\
23	1.36909421185738\\
23.0227272727273	1.35655253501534\\
23.0454545454545	1.34391127388018\\
23.0681818181818	1.33117135644474\\
23.0909090909091	1.31833371794422\\
23.1136363636364	1.30539930078749\\
23.1363636363636	1.29236905448795\\
23.1590909090909	1.27924393559378\\
23.1818181818182	1.26602490761774\\
23.2045454545455	1.25271294096646\\
23.2272727272727	1.23930901286917\\
23.25	1.22581410730595\\
23.2727272727273	1.21222921493558\\
23.2954545454545	1.19855533302269\\
23.3181818181818	1.18479346536468\\
23.3409090909091	1.1709446222179\\
23.3636363636364	1.15700982022362\\
23.3863636363636	1.1429900823333\\
23.4090909090909	1.12888643773354\\
23.4318181818182	1.11469992177051\\
23.4545454545455	1.10043157587395\\
23.4772727272727	1.08608244748075\\
23.5	1.07165358995799\\
23.5227272727273	1.05714606252568\\
23.5454545454545	1.04256093017897\\
23.5681818181818	1.02789926360995\\
23.5909090909091	1.01316213912912\\
23.6136363636364	0.998350638586285\\
23.6363636363636	0.98346584929121\\
23.6590909090909	0.968508863933776\\
23.6818181818182	0.95348078050375\\
23.7045454545455	0.938382702210217\\
23.7272727272727	0.923215737400553\\
23.75	0.907980999479096\\
23.7727272727273	0.892679606825385\\
23.7954545454545	0.877312682712085\\
23.8181818181818	0.861881355222496\\
23.8409090909091	0.846386757167783\\
23.8636363636364	0.830830026003775\\
23.8863636363636	0.815212303747481\\
23.9090909090909	0.79953473689327\\
23.9318181818182	0.78379847632868\\
23.9545454545455	0.768004677249942\\
23.9772727272727	0.752154499077201\\
24	0.736249105369359\\
24.0227272727273	0.720289663738687\\
24.0454545454545	0.704277345765114\\
24.0681818181818	0.688213326910198\\
24.0909090909091	0.672098786430861\\
24.1136363636364	0.65593490729281\\
24.1363636363636	0.639722876083683\\
24.1590909090909	0.623463882925964\\
24.1818181818182	0.607159121389606\\
24.2045454545455	0.590809788404401\\
24.2272727272727	0.574417084172135\\
24.25	0.55798221207846\\
24.2727272727273	0.54150637860457\\
24.2954545454545	0.52499079323863\\
24.3181818181818	0.508436668386977\\
24.3409090909091	0.491845219285125\\
24.3636363636364	0.475217663908561\\
24.3863636363636	0.458555222883325\\
24.4090909090909	0.441859119396409\\
24.4318181818182	0.425130579105956\\
24.4545454545455	0.408370830051288\\
24.4772727272727	0.391581102562766\\
24.5	0.374762629171455\\
24.5227272727273	0.357916644518646\\
24.5454545454545	0.34104438526525\\
24.5681818181818	0.324147090000967\\
24.5909090909091	0.307225999153402\\
24.6136363636364	0.290282354896979\\
24.6363636363636	0.273317401061775\\
24.6590909090909	0.256332383042184\\
24.6818181818182	0.239328547705532\\
24.7045454545455	0.222307143300497\\
24.7272727272727	0.205269419365517\\
24.75	0.188216626637032\\
24.7727272727273	0.171150016957681\\
24.7954545454545	0.154070843184411\\
24.8181818181818	0.136980359096494\\
24.8409090909091	0.119879819303482\\
24.8636363636364	0.102770479153129\\
24.8863636363636	0.0856535946392096\\
24.9090909090909	0.06853042230933\\
24.9318181818182	0.0514022191726905\\
24.9545454545455	0.0342702426078021\\
24.9772727272727	0.0171357502701721\\
25	7.34788079488412e-16\\
25.0227272727273	-0.0171357502701671\\
25.0454545454545	-0.0342702426077971\\
25.0681818181818	-0.0514022191726854\\
25.0909090909091	-0.0685304223093285\\
25.1136363636364	-0.0856535946392046\\
25.1363636363636	-0.102770479153124\\
25.1590909090909	-0.119879819303481\\
25.1818181818182	-0.136980359096489\\
25.2045454545455	-0.154070843184409\\
25.2272727272727	-0.17115001695768\\
25.25	-0.188216626637027\\
25.2727272727273	-0.205269419365512\\
25.2954545454545	-0.222307143300496\\
25.3181818181818	-0.239328547705527\\
25.3409090909091	-0.256332383042183\\
25.3636363636364	-0.27331740106177\\
25.3863636363636	-0.290282354896974\\
25.4090909090909	-0.307225999153397\\
25.4318181818182	-0.324147090000962\\
25.4545454545455	-0.341044385265245\\
25.4772727272727	-0.357916644518645\\
25.5	-0.37476262917145\\
25.5227272727273	-0.391581102562761\\
25.5454545454545	-0.408370830051286\\
25.5681818181818	-0.425130579105952\\
25.5909090909091	-0.441859119396404\\
25.6136363636364	-0.458555222883323\\
25.6363636363636	-0.475217663908559\\
25.6590909090909	-0.491845219285123\\
25.6818181818182	-0.508436668386973\\
25.7045454545455	-0.524990793238626\\
25.7272727272727	-0.541506378604569\\
25.75	-0.557982212078455\\
25.7727272727273	-0.57441708417213\\
25.7954545454545	-0.590809788404399\\
25.8181818181818	-0.607159121389602\\
25.8409090909091	-0.623463882925963\\
25.8636363636364	-0.639722876083679\\
25.8863636363636	-0.655934907292806\\
25.9090909090909	-0.672098786430857\\
25.9318181818182	-0.688213326910196\\
25.9545454545455	-0.704277345765109\\
25.9772727272727	-0.720289663738686\\
26	-0.736249105369355\\
26.0227272727273	-0.752154499077196\\
26.0454545454545	-0.768004677249941\\
26.0681818181818	-0.783798476328675\\
26.0909090909091	-0.799534736893265\\
26.1136363636364	-0.81521230374748\\
26.1363636363636	-0.83083002600377\\
26.1590909090909	-0.846386757167779\\
26.1818181818182	-0.861881355222494\\
26.2045454545455	-0.87731268271208\\
26.2272727272727	-0.892679606825381\\
26.25	-0.907980999479091\\
26.2727272727273	-0.923215737400551\\
26.2954545454545	-0.938382702210212\\
26.3181818181818	-0.953480780503748\\
26.3409090909091	-0.968508863933772\\
26.3636363636364	-0.983465849291209\\
26.3863636363636	-0.99835063858628\\
26.4090909090909	-1.01316213912911\\
26.4318181818182	-1.02789926360995\\
26.4545454545455	-1.04256093017897\\
26.4772727272727	-1.05714606252568\\
26.5	-1.07165358995799\\
26.5227272727273	-1.08608244748075\\
26.5454545454545	-1.10043157587395\\
26.5681818181818	-1.1146999217705\\
26.5909090909091	-1.12888643773353\\
26.6136363636364	-1.1429900823333\\
26.6363636363636	-1.15700982022362\\
26.6590909090909	-1.1709446222179\\
26.6818181818182	-1.18479346536467\\
26.7045454545455	-1.19855533302269\\
26.7272727272727	-1.21222921493557\\
26.75	-1.22581410730595\\
26.7727272727273	-1.23930901286916\\
26.7954545454545	-1.25271294096646\\
26.8181818181818	-1.26602490761774\\
26.8409090909091	-1.27924393559377\\
26.8636363636364	-1.29236905448795\\
26.8863636363636	-1.30539930078749\\
26.9090909090909	-1.31833371794421\\
26.9318181818182	-1.33117135644474\\
26.9545454545455	-1.34391127388017\\
26.9772727272727	-1.35655253501534\\
27	-1.36909421185738\\
27.0227272727273	-1.3815353837239\\
27.0454545454545	-1.39387513731058\\
27.0681818181818	-1.40611256675819\\
27.0909090909091	-1.41824677371909\\
27.1136363636364	-1.43027686742319\\
27.1363636363636	-1.44220196474334\\
27.1590909090909	-1.45402119026015\\
27.1818181818182	-1.46573367632626\\
27.2045454545455	-1.47733856313006\\
27.2272727272727	-1.48883499875874\\
27.25	-1.50022213926092\\
27.2727272727273	-1.51149914870852\\
27.2954545454545	-1.52266519925817\\
27.3181818181818	-1.53371947121199\\
27.3409090909091	-1.54466115307774\\
27.3636363636364	-1.55548944162839\\
27.3863636363636	-1.56620354196111\\
27.4090909090909	-1.57680266755559\\
27.4318181818182	-1.5872860403318\\
27.4545454545455	-1.59765289070713\\
27.4772727272727	-1.60790245765282\\
27.5	-1.61803398874989\\
27.5227272727273	-1.62804674024436\\
27.5454545454545	-1.63793997710182\\
27.5681818181818	-1.64771297306143\\
27.5909090909091	-1.65736501068918\\
27.6136363636364	-1.66689538143064\\
27.6363636363636	-1.67630338566287\\
27.6590909090909	-1.68558833274589\\
27.6818181818182	-1.69474954107328\\
27.7045454545455	-1.7037863381223\\
27.7272727272727	-1.71269806050318\\
27.75	-1.72148405400789\\
27.7727272727273	-1.73014367365812\\
27.7954545454545	-1.73867628375266\\
27.8181818181818	-1.74708125791404\\
27.8409090909091	-1.75535797913451\\
27.8636363636364	-1.76350583982137\\
27.8863636363636	-1.77152424184154\\
27.9090909090909	-1.77941259656547\\
27.9318181818182	-1.78717032491037\\
27.9545454545455	-1.79479685738272\\
27.9772727272727	-1.80229163412004\\
28	-1.80965410493204\\
28.0227272727273	-1.81688372934098\\
28.0454545454545	-1.82397997662134\\
28.0681818181818	-1.83094232583881\\
28.0909090909091	-1.83777026588851\\
28.1136363636364	-1.8444632955325\\
28.1363636363636	-1.85102092343661\\
28.1590909090909	-1.85744266820648\\
28.1818181818182	-1.8637280584229\\
28.2045454545455	-1.86987663267645\\
28.2272727272727	-1.8758879396013\\
28.25	-1.88176153790845\\
28.2727272727273	-1.88749699641802\\
28.2954545454545	-1.89309389409098\\
28.3181818181818	-1.89855182006\\
28.3409090909091	-1.90387037365967\\
28.3636363636364	-1.90904916445585\\
28.3863636363636	-1.91408781227439\\
28.4090909090909	-1.91898594722899\\
28.4318181818182	-1.92374320974841\\
28.4545454545454	-1.92835925060279\\
28.4772727272727	-1.93283373092936\\
28.5	-1.93716632225726\\
28.5227272727273	-1.94135670653171\\
28.5454545454545	-1.94540457613728\\
28.5681818181818	-1.94930963392057\\
28.5909090909091	-1.95307159321193\\
28.6136363636364	-1.95669017784656\\
28.6363636363636	-1.96016512218479\\
28.6590909090909	-1.96349617113152\\
28.6818181818182	-1.96668308015502\\
28.7045454545454	-1.96972561530483\\
28.7272727272727	-1.97262355322896\\
28.75	-1.97537668119028\\
28.7727272727273	-1.97798479708211\\
28.7954545454545	-1.98044770944313\\
28.8181818181818	-1.98276523747134\\
28.8409090909091	-1.9849372110374\\
28.8636363636364	-1.9869634706971\\
28.8863636363636	-1.98884386770306\\
28.9090909090909	-1.99057826401565\\
28.9318181818182	-1.99216653231312\\
28.9545454545455	-1.99360855600098\\
28.9772727272727	-1.99490422922051\\
29	-1.99605345685654\\
29.0227272727273	-1.99705615454448\\
29.0454545454545	-1.99791224867646\\
29.0681818181818	-1.99862167640675\\
29.0909090909091	-1.99918438565638\\
29.1136363636364	-1.99960033511696\\
29.1363636363636	-1.9998694942537\\
29.1590909090909	-1.99999184330769\\
29.1818181818182	-1.9999673732973\\
29.2045454545455	-1.99979608601887\\
29.2272727272727	-1.99947799404656\\
29.25	-1.99901312073146\\
29.2727272727273	-1.99840150019984\\
29.2954545454545	-1.99764317735066\\
29.3181818181818	-1.99673820785227\\
29.3409090909091	-1.99568665813833\\
29.3636363636364	-1.99448860540293\\
29.3863636363636	-1.99314413759491\\
29.4090909090909	-1.99165335341144\\
29.4318181818182	-1.99001636229072\\
29.4545454545455	-1.988233284404\\
29.4772727272727	-1.98630425064673\\
29.5	-1.98422940262896\\
29.5227272727273	-1.98200889266491\\
29.5454545454545	-1.97964288376187\\
29.5681818181818	-1.97713154960813\\
29.5909090909091	-1.97447507456033\\
29.6136363636364	-1.97167365362984\\
29.6363636363636	-1.9687274924685\\
29.6590909090909	-1.96563680735351\\
29.6818181818182	-1.96240182517155\\
29.7045454545455	-1.9590227834021\\
29.7272727272727	-1.95549993010006\\
29.75	-1.95183352387749\\
29.7727272727273	-1.94802383388467\\
29.7954545454545	-1.94407113979028\\
29.8181818181818	-1.93997573176093\\
29.8409090909091	-1.93573791043985\\
29.8636363636364	-1.93135798692478\\
29.8863636363636	-1.92683628274516\\
29.9090909090909	-1.92217312983854\\
29.9318181818182	-1.91736887052619\\
29.9545454545455	-1.91242385748798\\
29.9772727272727	-1.90733845373647\\
30	-1.90211303259031\\
30.0227272727273	-1.89674797764678\\
30.0454545454545	-1.89124368275365\\
30.0681818181818	-1.8856005519803\\
30.0909090909091	-1.87981899958799\\
30.1136363636364	-1.87389944999952\\
30.1363636363636	-1.86784233776803\\
30.1590909090909	-1.8616481075451\\
30.1818181818182	-1.85531721404814\\
30.2045454545455	-1.84885012202696\\
30.2272727272727	-1.8422473062297\\
30.25	-1.83550925136796\\
30.2727272727273	-1.82863645208122\\
30.2954545454545	-1.8216294129005\\
30.3181818181818	-1.81448864821138\\
30.3409090909091	-1.80721468221618\\
30.3636363636364	-1.79980804889554\\
30.3863636363636	-1.79226929196915\\
30.4090909090909	-1.78459896485588\\
30.4318181818182	-1.77679763063316\\
30.4545454545455	-1.76886586199563\\
30.4772727272727	-1.76080424121308\\
30.5	-1.75261336008773\\
30.5227272727273	-1.74429381991079\\
30.5454545454545	-1.7358462314183\\
30.5681818181818	-1.72727121474632\\
30.5909090909091	-1.71856939938536\\
30.6136363636364	-1.70974142413423\\
30.6363636363636	-1.70078793705311\\
30.6590909090909	-1.69170959541598\\
30.6818181818182	-1.68250706566236\\
30.7045454545455	-1.67318102334842\\
30.7272727272727	-1.66373215309735\\
30.75	-1.65416114854912\\
30.7727272727273	-1.64446871230958\\
30.7954545454545	-1.63465555589882\\
30.8181818181818	-1.624722399699\\
30.8409090909091	-1.61466997290146\\
30.8636363636364	-1.60449901345315\\
30.8863636363636	-1.59421026800249\\
30.9090909090909	-1.58380449184455\\
30.9318181818182	-1.57328244886561\\
30.9545454545455	-1.56264491148707\\
30.9772727272727	-1.55189266060874\\
31	-1.54102648555158\\
31.0227272727273	-1.53004718399965\\
31.0454545454545	-1.51895556194163\\
31.0681818181818	-1.50775243361162\\
31.0909090909091	-1.4964386214294\\
31.1136363636364	-1.48501495594001\\
31.1363636363636	-1.47348227575281\\
31.1590909090909	-1.46184142747991\\
31.1818181818182	-1.45009326567402\\
31.2045454545455	-1.43823865276571\\
31.2272727272727	-1.42627845900014\\
31.25	-1.4142135623731\\
31.2727272727273	-1.40204484856662\\
31.2954545454545	-1.38977321088395\\
31.3181818181818	-1.37739955018393\\
31.3409090909091	-1.3649247748149\\
31.3636363636364	-1.35234980054804\\
31.3863636363636	-1.33967555051008\\
31.4090909090909	-1.32690295511559\\
31.4318181818182	-1.31403295199863\\
31.4545454545454	-1.30106648594398\\
31.4772727272727	-1.28800450881771\\
31.5	-1.27484797949738\\
31.5227272727273	-1.26159786380158\\
31.5454545454545	-1.24825513441908\\
31.5681818181818	-1.23482077083739\\
31.5909090909091	-1.22129575927088\\
31.6136363636364	-1.20768109258837\\
31.6363636363636	-1.19397777024025\\
31.6590909090909	-1.1801867981851\\
31.6818181818182	-1.16630918881587\\
31.7045454545454	-1.15234596088551\\
31.7272727272727	-1.13829813943224\\
31.75	-1.12416675570426\\
31.7727272727273	-1.10995284708408\\
31.7954545454545	-1.09565745701234\\
31.8181818181818	-1.0812816349112\\
31.8409090909091	-1.06682643610735\\
31.8636363636364	-1.0522929217545\\
31.8863636363636	-1.03768215875549\\
31.9090909090909	-1.02299521968397\\
31.9318181818182	-1.00823318270567\\
31.9545454545454	-0.993397131499224\\
31.9772727272727	-0.978488155176668\\
32	-0.96350734820343\\
32.0227272727273	-0.948455810318039\\
32.0454545454545	-0.93333464645135\\
32.0681818181818	-0.918144966645456\\
32.0909090909091	-0.902887885972191\\
32.1136363636364	-0.887564524451277\\
32.1363636363636	-0.872176006968111\\
32.1590909090909	-0.856723463191165\\
32.1818181818182	-0.841208027489074\\
32.2045454545455	-0.82563083884736\\
32.2272727272727	-0.809993040784812\\
32.25	-0.794295781269561\\
32.2727272727273	-0.778540212634785\\
32.2954545454545	-0.762727491494116\\
32.3181818181818	-0.746858778656758\\
32.3409090909091	-0.730935239042244\\
32.3636363636364	-0.714958041594935\\
32.3863636363636	-0.698928359198195\\
32.4090909090909	-0.682847368588328\\
32.4318181818182	-0.666716250268138\\
32.4545454545455	-0.650536188420303\\
32.4772727272727	-0.634308370820439\\
32.5	-0.618033988749896\\
32.5227272727273	-0.60171423690833\\
32.5454545454545	-0.585350313325976\\
32.5681818181818	-0.568943419275705\\
32.5909090909091	-0.552494759184846\\
32.6136363636364	-0.536005540546762\\
32.6363636363636	-0.519476973832219\\
32.6590909090909	-0.502910272400518\\
32.6818181818182	-0.486306652410422\\
32.7045454545455	-0.469667332730873\\
32.7272727272727	-0.452993534851532\\
32.75	-0.436286482793087\\
32.7727272727273	-0.419547403017419\\
32.7954545454545	-0.402777524337555\\
32.8181818181818	-0.385978077827468\\
32.8409090909091	-0.369150296731694\\
32.8636363636364	-0.352295416374808\\
32.8863636363636	-0.335414674070733\\
32.9090909090909	-0.318509309031931\\
32.9318181818182	-0.301580562278396\\
32.9545454545455	-0.284629676546573\\
32.9772727272727	-0.267657896198135\\
33	-0.250666467128615\\
33.0227272727273	-0.233656636675954\\
33.0454545454545	-0.216629653528953\\
33.0681818181818	-0.199586767635568\\
33.0909090909091	-0.182529230111184\\
33.1136363636364	-0.165458293146763\\
33.1363636363636	-0.148375209916905\\
33.1590909090909	-0.13128123448788\\
33.1818181818182	-0.114177621725538\\
33.2045454545455	-0.0970656272032175\\
33.2272727272727	-0.0799465071095486\\
33.25	-0.0628215181562608\\
33.2727272727273	-0.0456919174858983\\
33.2954545454545	-0.0285589625795757\\
33.3181818181818	-0.01142391116462\\
33.3409090909091	0.00571197887773029\\
33.3636363636364	0.0228474496046767\\
33.3863636363636	0.0399812431041992\\
33.4090909090909	0.0571121015873918\\
33.4318181818182	0.0742387674808257\\
33.4545454545455	0.091359983518842\\
33.4772727272727	0.108474492835855\\
33.5	0.125581039058625\\
33.5227272727273	0.142678366398465\\
33.5454545454545	0.159765219743453\\
33.5681818181818	0.176840344750571\\
33.5909090909091	0.19390248793776\\
33.6136363636364	0.210950396775967\\
33.6363636363636	0.227982819781078\\
33.6590909090909	0.244998506605789\\
33.6818181818182	0.261996208131385\\
33.7045454545455	0.278974676559468\\
33.7272727272727	0.295932665503529\\
33.75	0.312868930080456\\
33.7727272727273	0.329782227001925\\
33.7954545454545	0.346671314665649\\
33.8181818181818	0.363534953246545\\
33.8409090909091	0.380371904787749\\
33.8636363636364	0.397180933291485\\
33.8863636363636	0.413960804809801\\
33.9090909090909	0.430710287535161\\
33.9318181818182	0.44742815189085\\
33.9545454545455	0.46411317062126\\
33.9772727272727	0.480764118881983\\
34	0.497379774329706\\
34.0227272727273	0.513958917211967\\
34.0454545454545	0.530500330456675\\
34.0681818181818	0.547002799761471\\
34.0909090909091	0.563465113682858\\
34.1136363636364	0.579886063725148\\
34.1363636363636	0.596264444429164\\
34.1590909090909	0.612599053460743\\
34.1818181818182	0.628888691698974\\
34.2045454545455	0.645132163324267\\
34.2272727272727	0.6613282759061\\
34.25	0.677475840490578\\
34.2727272727273	0.693573671687706\\
34.2954545454545	0.709620587758412\\
34.3181818181818	0.725615410701278\\
34.3409090909091	0.741556966339055\\
34.3636363636364	0.757444084404822\\
34.3863636363636	0.773275598627924\\
34.4090909090909	0.789050346819564\\
34.4318181818182	0.804767170958133\\
34.4545454545455	0.820424917274222\\
34.4772727272727	0.836022436335309\\
34.5	0.851558583130143\\
34.5227272727273	0.867032217152808\\
34.5454545454545	0.882442202486443\\
34.5681818181818	0.897787407886607\\
34.5909090909091	0.913066706864362\\
34.6136363636364	0.928278977768944\\
34.6363636363636	0.94342310387011\\
34.6590909090909	0.958497973440115\\
34.6818181818182	0.973502479835312\\
34.7045454545455	0.98843552157742\\
34.7272727272727	1.00329600243436\\
34.75	1.01808283150074\\
34.7727272727273	1.03279492327792\\
34.7954545454545	1.04743119775373\\
34.8181818181818	1.06199058048171\\
34.8409090909091	1.07647200266005\\
34.8636363636364	1.09087440120996\\
34.8863636363636	1.1051967188538\\
34.9090909090909	1.11943790419265\\
34.9318181818182	1.13359691178349\\
34.9545454545455	1.14767270221595\\
34.9772727272727	1.16166424218863\\
35	1.17557050458494\\
35.0227272727273	1.1893904685485\\
35.0454545454545	1.20312311955808\\
35.0681818181818	1.21676744950207\\
35.0909090909091	1.23032245675253\\
35.1136363636364	1.24378714623864\\
35.1363636363636	1.25716052951983\\
35.1590909090909	1.27044162485827\\
35.1818181818182	1.283629457291\\
35.2045454545455	1.29672305870146\\
35.2272727272727	1.30972146789057\\
35.25	1.3226237306473\\
35.2727272727273	1.33542889981872\\
35.2954545454545	1.34813603537952\\
35.3181818181818	1.36074420450101\\
35.3409090909091	1.37325248161963\\
35.3636363636364	1.38565994850485\\
35.3863636363636	1.39796569432661\\
35.4090909090909	1.41016881572221\\
35.4318181818182	1.42226841686254\\
35.4545454545455	1.43426360951793\\
35.4772727272727	1.44615351312329\\
35.5	1.45793725484282\\
35.5227272727273	1.46961396963401\\
35.5454545454545	1.48118280031121\\
35.5681818181818	1.49264289760849\\
35.5909090909091	1.50399342024205\\
35.6136363636364	1.51523353497193\\
35.6363636363636	1.5263624166632\\
35.6590909090909	1.53737924834653\\
35.6818181818182	1.54828322127816\\
35.7045454545455	1.55907353499928\\
35.7272727272727	1.56974939739477\\
35.75	1.58031002475138\\
35.7727272727273	1.59075464181522\\
35.7954545454545	1.60108248184872\\
35.8181818181818	1.61129278668688\\
35.8409090909091	1.62138480679294\\
35.8636363636364	1.6313578013134\\
35.8863636363636	1.64121103813242\\
35.9090909090909	1.65094379392555\\
35.9318181818182	1.66055535421281\\
35.9545454545455	1.67004501341118\\
35.9772727272727	1.67941207488639\\
36	1.68865585100403\\
36.0227272727273	1.69777566318004\\
36.0454545454545	1.70677084193056\\
36.0681818181818	1.71564072692103\\
36.0909090909091	1.72438466701467\\
36.1136363636364	1.73300202032033\\
36.1363636363636	1.74149215423955\\
36.1590909090909	1.74985444551303\\
36.1818181818182	1.75808828026637\\
36.2045454545455	1.76619305405515\\
36.2272727272727	1.77416817190929\\
36.25	1.78201304837673\\
36.2727272727273	1.78972710756642\\
36.2954545454545	1.79730978319057\\
36.3181818181818	1.80476051860623\\
36.3409090909091	1.81207876685618\\
36.3636363636364	1.81926399070904\\
36.3863636363636	1.82631566269873\\
36.4090909090909	1.83323326516322\\
36.4318181818182	1.84001629028246\\
36.4545454545455	1.84666424011573\\
36.4772727272727	1.85317662663814\\
36.5	1.8595529717765\\
36.5227272727273	1.86579280744439\\
36.5454545454545	1.87189567557651\\
36.5681818181818	1.87786112816236\\
36.5909090909091	1.88368872727905\\
36.6136363636364	1.88937804512352\\
36.6363636363636	1.89492866404391\\
36.6590909090909	1.9003401765702\\
36.6818181818182	1.90561218544419\\
36.7045454545455	1.91074430364859\\
36.7272727272727	1.91573615443547\\
36.75	1.92058737135389\\
36.7727272727273	1.92529759827684\\
36.7954545454545	1.92986648942736\\
36.8181818181818	1.93429370940391\\
36.8409090909091	1.93857893320503\\
36.8636363636364	1.94272184625316\\
36.8863636363636	1.94672214441774\\
36.9090909090909	1.95057953403757\\
36.9318181818182	1.95429373194232\\
36.9545454545455	1.95786446547334\\
36.9772727272727	1.9612914725037\\
37	1.96457450145738\\
37.0227272727273	1.96771331132777\\
37.0454545454545	1.97070767169539\\
37.0681818181818	1.97355736274472\\
37.0909090909091	1.97626217528044\\
37.1136363636364	1.97882191074272\\
37.1363636363636	1.9812363812218\\
37.1590909090909	1.9835054094718\\
37.1818181818182	1.98562882892376\\
37.2045454545455	1.98760648369781\\
37.2272727272727	1.98943822861464\\
37.25	1.99112392920616\\
37.2727272727273	1.99266346172538\\
37.2954545454545	1.99405671315548\\
37.3181818181818	1.99530358121811\\
37.3409090909091	1.99640397438088\\
37.3636363636364	1.99735781186412\\
37.3863636363636	1.99816502364679\\
37.4090909090909	1.9988255504716\\
37.4318181818182	1.99933934384939\\
37.4545454545455	1.99970636606268\\
37.4772727272727	1.99992659016842\\
37.5	2\\
37.5227272727273	1.99992659016842\\
37.5454545454545	1.99970636606268\\
37.5681818181818	1.99933934384939\\
37.5909090909091	1.9988255504716\\
37.6136363636364	1.99816502364679\\
37.6363636363636	1.99735781186412\\
37.6590909090909	1.99640397438088\\
37.6818181818182	1.99530358121811\\
37.7045454545454	1.99405671315548\\
37.7272727272727	1.99266346172538\\
37.75	1.99112392920616\\
37.7727272727273	1.98943822861464\\
37.7954545454545	1.98760648369781\\
37.8181818181818	1.98562882892377\\
37.8409090909091	1.98350540947181\\
37.8636363636364	1.9812363812218\\
37.8863636363636	1.97882191074272\\
37.9090909090909	1.97626217528045\\
37.9318181818182	1.97355736274472\\
37.9545454545454	1.97070767169539\\
37.9772727272727	1.96771331132777\\
38	1.96457450145738\\
38.0227272727273	1.9612914725037\\
38.0454545454545	1.95786446547334\\
38.0681818181818	1.95429373194232\\
38.0909090909091	1.95057953403757\\
38.1136363636364	1.94672214441775\\
38.1363636363636	1.94272184625316\\
38.1590909090909	1.93857893320504\\
38.1818181818182	1.93429370940392\\
38.2045454545454	1.92986648942736\\
38.2272727272727	1.92529759827684\\
38.25	1.92058737135389\\
38.2727272727273	1.91573615443547\\
38.2954545454545	1.91074430364859\\
38.3181818181818	1.9056121854442\\
38.3409090909091	1.90034017657021\\
38.3636363636364	1.89492866404391\\
38.3863636363636	1.88937804512352\\
38.4090909090909	1.88368872727905\\
38.4318181818182	1.87786112816236\\
38.4545454545454	1.87189567557652\\
38.4772727272727	1.86579280744439\\
38.5	1.8595529717765\\
38.5227272727273	1.85317662663814\\
38.5454545454545	1.84666424011573\\
38.5681818181818	1.84001629028247\\
38.5909090909091	1.83323326516323\\
38.6136363636364	1.82631566269874\\
38.6363636363636	1.81926399070904\\
38.6590909090909	1.81207876685618\\
38.6818181818182	1.80476051860623\\
38.7045454545455	1.79730978319057\\
38.7272727272727	1.78972710756642\\
38.75	1.78201304837674\\
38.7727272727273	1.77416817190929\\
38.7954545454545	1.76619305405515\\
38.8181818181818	1.75808828026637\\
38.8409090909091	1.74985444551303\\
38.8636363636364	1.74149215423955\\
38.8863636363636	1.73300202032034\\
38.9090909090909	1.72438466701468\\
38.9318181818182	1.71564072692103\\
38.9545454545455	1.70677084193057\\
38.9772727272727	1.69777566318005\\
39	1.68865585100403\\
39.0227272727273	1.6794120748864\\
39.0454545454545	1.67004501341119\\
39.0681818181818	1.66055535421281\\
39.0909090909091	1.65094379392555\\
39.1136363636364	1.64121103813243\\
39.1363636363636	1.6313578013134\\
39.1590909090909	1.62138480679294\\
39.1818181818182	1.61129278668688\\
39.2045454545455	1.60108248184872\\
39.2272727272727	1.59075464181523\\
39.25	1.58031002475138\\
39.2727272727273	1.56974939739478\\
39.2954545454545	1.55907353499929\\
39.3181818181818	1.54828322127817\\
39.3409090909091	1.53737924834654\\
39.3636363636364	1.5263624166632\\
39.3863636363636	1.51523353497193\\
39.4090909090909	1.50399342024205\\
39.4318181818182	1.4926428976085\\
39.4545454545455	1.48118280031122\\
39.4772727272727	1.46961396963402\\
39.5	1.45793725484282\\
39.5227272727273	1.4461535131233\\
39.5454545454545	1.43426360951793\\
39.5681818181818	1.42226841686254\\
39.5909090909091	1.41016881572221\\
39.6136363636364	1.39796569432662\\
39.6363636363636	1.38565994850485\\
39.6590909090909	1.37325248161963\\
39.6818181818182	1.36074420450101\\
39.7045454545455	1.34813603537952\\
39.7272727272727	1.33542889981873\\
39.75	1.3226237306473\\
39.7727272727273	1.30972146789057\\
39.7954545454545	1.29672305870146\\
39.8181818181818	1.283629457291\\
39.8409090909091	1.27044162485828\\
39.8636363636364	1.25716052951983\\
39.8863636363636	1.24378714623865\\
39.9090909090909	1.23032245675253\\
39.9318181818182	1.21676744950208\\
39.9545454545455	1.20312311955808\\
39.9772727272727	1.18939046854851\\
40	1.17557050458494\\
40.0227272727273	1.16166424218864\\
40.0454545454545	1.14767270221596\\
40.0681818181818	1.1335969117835\\
40.0909090909091	1.11943790419266\\
40.1136363636364	1.10519671885381\\
40.1363636363636	1.09087440120996\\
40.1590909090909	1.07647200266005\\
40.1818181818182	1.06199058048172\\
40.2045454545455	1.04743119775373\\
40.2272727272727	1.03279492327793\\
40.25	1.01808283150074\\
40.2727272727273	1.00329600243437\\
40.2954545454545	0.988435521577422\\
40.3181818181818	0.973502479835317\\
40.3409090909091	0.958497973440117\\
40.3636363636364	0.943423103870118\\
40.3863636363636	0.928278977768946\\
40.4090909090909	0.913066706864367\\
40.4318181818182	0.897787407886609\\
40.4545454545455	0.882442202486448\\
40.4772727272727	0.867032217152817\\
40.5	0.851558583130155\\
40.5227272727273	0.836022436335311\\
40.5454545454545	0.820424917274227\\
40.5681818181818	0.804767170958138\\
40.5909090909091	0.789050346819566\\
40.6136363636364	0.773275598627932\\
40.6363636363636	0.757444084404831\\
40.6590909090909	0.741556966339057\\
40.6818181818182	0.72561541070128\\
40.7045454545455	0.709620587758418\\
40.7272727272727	0.693573671687715\\
40.75	0.67747584049059\\
40.7727272727273	0.661328275906102\\
40.7954545454545	0.645132163324272\\
40.8181818181818	0.628888691698976\\
40.8409090909091	0.612599053460748\\
40.8636363636364	0.596264444429173\\
40.8863636363636	0.579886063725154\\
40.9090909090909	0.56346511368286\\
40.9318181818182	0.547002799761473\\
40.9545454545455	0.530500330456677\\
40.9772727272727	0.513958917211976\\
41	0.497379774329715\\
41.0227272727273	0.480764118881989\\
41.0454545454545	0.464113170621265\\
41.0681818181818	0.447428151890852\\
41.0909090909091	0.430710287535167\\
41.1136363636364	0.41396080480981\\
41.1363636363636	0.397180933291498\\
41.1590909090909	0.380371904787752\\
41.1818181818182	0.36353495324655\\
41.2045454545455	0.346671314665651\\
41.2272727272727	0.329782227001931\\
41.25	0.312868930080465\\
41.2727272727273	0.295932665503534\\
41.2954545454545	0.27897467655947\\
41.3181818181818	0.261996208131391\\
41.3409090909091	0.244998506605791\\
41.3636363636364	0.227982819781087\\
41.3863636363636	0.210950396775976\\
41.4090909090909	0.193902487937766\\
41.4318181818182	0.176840344750573\\
41.4545454545455	0.159765219743462\\
41.4772727272727	0.14267836639847\\
41.5	0.125581039058627\\
41.5227272727273	0.108474492835861\\
41.5454545454545	0.0913599835188443\\
41.5681818181818	0.0742387674808315\\
41.5909090909091	0.0571121015873941\\
41.6136363636364	0.0399812431042049\\
41.6363636363636	0.022847449604686\\
41.6590909090909	0.00571197887773605\\
41.6818181818182	-0.0114239111646178\\
41.7045454545455	-0.02855896257957\\
41.7272727272727	-0.0456919174858961\\
41.75	-0.0628215181562479\\
41.7727272727273	-0.0799465071095428\\
41.7954545454545	-0.0970656272032118\\
41.8181818181818	-0.114177621725535\\
41.8409090909091	-0.131281234487871\\
41.8636363636364	-0.1483752099169\\
41.8863636363636	-0.165458293146758\\
41.9090909090909	-0.182529230111182\\
41.9318181818182	-0.199586767635562\\
41.9545454545455	-0.216629653528947\\
41.9772727272727	-0.233656636675951\\
42	-0.250666467128602\\
42.0227272727273	-0.267657896198125\\
42.0454545454545	-0.284629676546567\\
42.0681818181818	-0.301580562278394\\
42.0909090909091	-0.318509309031925\\
42.1136363636364	-0.335414674070731\\
42.1363636363636	-0.352295416374796\\
42.1590909090909	-0.369150296731692\\
42.1818181818182	-0.385978077827462\\
42.2045454545455	-0.402777524337553\\
42.2272727272727	-0.419547403017413\\
42.25	-0.436286482793081\\
42.2727272727273	-0.452993534851527\\
42.2954545454545	-0.469667332730871\\
42.3181818181818	-0.486306652410413\\
42.3409090909091	-0.502910272400516\\
42.3636363636364	-0.519476973832217\\
42.3863636363636	-0.536005540546753\\
42.4090909090909	-0.552494759184837\\
42.4318181818182	-0.5689434192757\\
42.4545454545455	-0.585350313325974\\
42.4772727272727	-0.601714236908325\\
42.5	-0.61803398874989\\
42.5227272727273	-0.634308370820424\\
42.5454545454545	-0.650536188420298\\
42.5681818181818	-0.666716250268136\\
42.5909090909091	-0.68284736858833\\
42.6136363636364	-0.698928359198193\\
42.6363636363636	-0.714958041594926\\
42.6590909090909	-0.730935239042239\\
42.6818181818182	-0.746858778656756\\
42.7045454545455	-0.762727491494111\\
42.7272727272727	-0.778540212634779\\
42.75	-0.794295781269559\\
42.7727272727273	-0.809993040784803\\
42.7954545454545	-0.825630838847349\\
42.8181818181818	-0.841208027489065\\
42.8409090909091	-0.85672346319116\\
42.8636363636364	-0.872176006968109\\
42.8863636363636	-0.887564524451278\\
42.9090909090909	-0.902887885972183\\
42.9318181818182	-0.918144966645454\\
42.9545454545455	-0.933334646451345\\
42.9772727272727	-0.948455810318037\\
43	-0.963507348203424\\
43.0227272727273	-0.97848815517666\\
43.0454545454545	-0.993397131499213\\
43.0681818181818	-1.00823318270566\\
43.0909090909091	-1.02299521968396\\
43.1136363636364	-1.03768215875549\\
43.1363636363636	-1.0522929217545\\
43.1590909090909	-1.06682643610734\\
43.1818181818182	-1.08128163491119\\
43.2045454545455	-1.09565745701233\\
43.2272727272727	-1.10995284708408\\
43.25	-1.12416675570426\\
43.2727272727273	-1.13829813943223\\
43.2954545454545	-1.1523459608855\\
43.3181818181818	-1.16630918881586\\
43.3409090909091	-1.18018679818509\\
43.3636363636364	-1.19397777024024\\
43.3863636363636	-1.20768109258836\\
43.4090909090909	-1.22129575927088\\
43.4318181818182	-1.23482077083739\\
43.4545454545455	-1.24825513441908\\
43.4772727272727	-1.26159786380158\\
43.5	-1.27484797949738\\
43.5227272727273	-1.28800450881771\\
43.5454545454545	-1.30106648594397\\
43.5681818181818	-1.31403295199862\\
43.5909090909091	-1.32690295511558\\
43.6136363636364	-1.33967555051008\\
43.6363636363636	-1.35234980054803\\
43.6590909090909	-1.3649247748149\\
43.6818181818182	-1.37739955018392\\
43.7045454545455	-1.38977321088395\\
43.7272727272727	-1.40204484856662\\
43.75	-1.4142135623731\\
43.7727272727273	-1.42627845900013\\
43.7954545454545	-1.43823865276571\\
43.8181818181818	-1.45009326567401\\
43.8409090909091	-1.46184142747991\\
43.8636363636364	-1.47348227575281\\
43.8863636363636	-1.48501495594001\\
43.9090909090909	-1.4964386214294\\
43.9318181818182	-1.50775243361162\\
43.9545454545455	-1.51895556194162\\
43.9772727272727	-1.53004718399965\\
44	-1.54102648555158\\
44.0227272727273	-1.55189266060874\\
44.0454545454545	-1.56264491148706\\
44.0681818181818	-1.57328244886561\\
44.0909090909091	-1.58380449184455\\
44.1136363636364	-1.59421026800249\\
44.1363636363636	-1.60449901345315\\
44.1590909090909	-1.61466997290146\\
44.1818181818182	-1.624722399699\\
44.2045454545455	-1.63465555589881\\
44.2272727272727	-1.64446871230957\\
44.25	-1.65416114854912\\
44.2727272727273	-1.66373215309735\\
44.2954545454545	-1.67318102334842\\
44.3181818181818	-1.68250706566236\\
44.3409090909091	-1.69170959541598\\
44.3636363636364	-1.70078793705311\\
44.3863636363636	-1.70974142413423\\
44.4090909090909	-1.71856939938536\\
44.4318181818182	-1.72727121474631\\
44.4545454545455	-1.7358462314183\\
44.4772727272727	-1.74429381991078\\
44.5	-1.75261336008772\\
44.5227272727273	-1.76080424121307\\
44.5454545454545	-1.76886586199563\\
44.5681818181818	-1.77679763063316\\
44.5909090909091	-1.78459896485588\\
44.6136363636364	-1.79226929196914\\
44.6363636363636	-1.79980804889554\\
44.6590909090909	-1.80721468221618\\
44.6818181818182	-1.81448864821137\\
44.7045454545455	-1.8216294129005\\
44.7272727272727	-1.82863645208121\\
44.75	-1.83550925136796\\
44.7727272727273	-1.8422473062297\\
44.7954545454545	-1.84885012202696\\
44.8181818181818	-1.85531721404814\\
44.8409090909091	-1.8616481075451\\
44.8636363636364	-1.86784233776803\\
44.8863636363636	-1.87389944999952\\
44.9090909090909	-1.87981899958799\\
44.9318181818182	-1.88560055198029\\
44.9545454545454	-1.89124368275365\\
44.9772727272727	-1.89674797764677\\
45	-1.90211303259031\\
45.0227272727273	-1.90733845373647\\
45.0454545454545	-1.91242385748797\\
45.0681818181818	-1.91736887052619\\
45.0909090909091	-1.92217312983854\\
45.1136363636364	-1.92683628274516\\
45.1363636363636	-1.93135798692478\\
45.1590909090909	-1.93573791043985\\
45.1818181818182	-1.93997573176093\\
45.2045454545454	-1.94407113979028\\
45.2272727272727	-1.94802383388467\\
45.25	-1.95183352387749\\
45.2727272727273	-1.95549993010006\\
45.2954545454545	-1.9590227834021\\
45.3181818181818	-1.96240182517154\\
45.3409090909091	-1.96563680735351\\
45.3636363636364	-1.9687274924685\\
45.3863636363636	-1.97167365362984\\
45.4090909090909	-1.97447507456033\\
45.4318181818182	-1.97713154960813\\
45.4545454545454	-1.97964288376186\\
45.4772727272727	-1.98200889266491\\
45.5	-1.98422940262896\\
45.5227272727273	-1.98630425064673\\
45.5454545454545	-1.988233284404\\
45.5681818181818	-1.99001636229072\\
45.5909090909091	-1.99165335341143\\
45.6136363636364	-1.99314413759491\\
45.6363636363636	-1.99448860540293\\
45.6590909090909	-1.99568665813833\\
45.6818181818182	-1.99673820785227\\
45.7045454545454	-1.99764317735066\\
45.7272727272727	-1.99840150019984\\
45.75	-1.99901312073146\\
45.7727272727273	-1.99947799404656\\
45.7954545454545	-1.99979608601887\\
45.8181818181818	-1.9999673732973\\
45.8409090909091	-1.99999184330769\\
45.8636363636364	-1.9998694942537\\
45.8863636363636	-1.99960033511696\\
45.9090909090909	-1.99918438565638\\
45.9318181818182	-1.99862167640675\\
45.9545454545454	-1.99791224867646\\
45.9772727272727	-1.99705615454448\\
46	-1.99605345685654\\
46.0227272727273	-1.99490422922051\\
46.0454545454545	-1.99360855600098\\
46.0681818181818	-1.99216653231313\\
46.0909090909091	-1.99057826401565\\
46.1136363636364	-1.98884386770306\\
46.1363636363636	-1.9869634706971\\
46.1590909090909	-1.9849372110374\\
46.1818181818182	-1.98276523747134\\
46.2045454545454	-1.98044770944313\\
46.2272727272727	-1.97798479708212\\
46.25	-1.97537668119027\\
46.2727272727273	-1.97262355322896\\
46.2954545454545	-1.96972561530483\\
46.3181818181818	-1.96668308015502\\
46.3409090909091	-1.96349617113152\\
46.3636363636364	-1.96016512218479\\
46.3863636363636	-1.95669017784656\\
46.4090909090909	-1.95307159321193\\
46.4318181818182	-1.94930963392057\\
46.4545454545455	-1.94540457613728\\
46.4772727272727	-1.94135670653171\\
46.5	-1.93716632225726\\
46.5227272727273	-1.93283373092936\\
46.5454545454545	-1.92835925060279\\
46.5681818181818	-1.92374320974841\\
46.5909090909091	-1.918985947229\\
46.6136363636364	-1.91408781227439\\
46.6363636363636	-1.90904916445585\\
46.6590909090909	-1.90387037365967\\
46.6818181818182	-1.89855182006\\
46.7045454545455	-1.89309389409098\\
46.7272727272727	-1.88749699641803\\
46.75	-1.88176153790845\\
46.7727272727273	-1.87588793960131\\
46.7954545454545	-1.86987663267645\\
46.8181818181818	-1.8637280584229\\
46.8409090909091	-1.85744266820648\\
46.8636363636364	-1.85102092343661\\
46.8863636363636	-1.8444632955325\\
46.9090909090909	-1.83777026588851\\
46.9318181818182	-1.83094232583882\\
46.9545454545455	-1.82397997662134\\
46.9772727272727	-1.81688372934098\\
47	-1.80965410493204\\
47.0227272727273	-1.80229163412004\\
47.0454545454545	-1.79479685738272\\
47.0681818181818	-1.78717032491037\\
47.0909090909091	-1.77941259656547\\
47.1136363636364	-1.77152424184154\\
47.1363636363636	-1.76350583982137\\
47.1590909090909	-1.75535797913451\\
47.1818181818182	-1.74708125791404\\
47.2045454545455	-1.73867628375266\\
47.2272727272727	-1.73014367365813\\
47.25	-1.72148405400789\\
47.2727272727273	-1.71269806050318\\
47.2954545454545	-1.7037863381223\\
47.3181818181818	-1.69474954107329\\
47.3409090909091	-1.68558833274589\\
47.3636363636364	-1.67630338566287\\
47.3863636363636	-1.66689538143063\\
47.4090909090909	-1.65736501068919\\
47.4318181818182	-1.64771297306143\\
47.4545454545455	-1.63793997710183\\
47.4772727272727	-1.62804674024437\\
47.5	-1.6180339887499\\
47.5227272727273	-1.60790245765282\\
47.5454545454545	-1.59765289070713\\
47.5681818181818	-1.58728604033181\\
47.5909090909091	-1.57680266755559\\
47.6136363636364	-1.56620354196111\\
47.6363636363636	-1.55548944162839\\
47.6590909090909	-1.54466115307775\\
47.6818181818182	-1.533719471212\\
47.7045454545455	-1.52266519925818\\
47.7272727272727	-1.51149914870852\\
47.75	-1.50022213926093\\
47.7727272727273	-1.48883499875874\\
47.7954545454545	-1.47733856313006\\
47.8181818181818	-1.46573367632627\\
47.8409090909091	-1.45402119026015\\
47.8636363636364	-1.44220196474335\\
47.8863636363636	-1.4302768674232\\
47.9090909090909	-1.41824677371909\\
47.9318181818182	-1.40611256675819\\
47.9545454545455	-1.39387513731059\\
47.9772727272727	-1.3815353837239\\
48	-1.36909421185738\\
48.0227272727273	-1.35655253501534\\
48.0454545454545	-1.34391127388018\\
48.0681818181818	-1.33117135644474\\
48.0909090909091	-1.31833371794422\\
48.1136363636364	-1.30539930078749\\
48.1363636363636	-1.29236905448795\\
48.1590909090909	-1.27924393559378\\
48.1818181818182	-1.26602490761774\\
48.2045454545455	-1.25271294096646\\
48.2272727272727	-1.23930901286917\\
48.25	-1.22581410730596\\
48.2727272727273	-1.21222921493558\\
48.2954545454545	-1.1985553330227\\
48.3181818181818	-1.18479346536467\\
48.3409090909091	-1.17094462221791\\
48.3636363636364	-1.15700982022363\\
48.3863636363636	-1.14299008233331\\
48.4090909090909	-1.12888643773354\\
48.4318181818182	-1.11469992177051\\
48.4545454545455	-1.10043157587396\\
48.4772727272727	-1.08608244748075\\
48.5	-1.071653589958\\
48.5227272727273	-1.05714606252568\\
48.5454545454545	-1.04256093017897\\
48.5681818181818	-1.02789926360996\\
48.5909090909091	-1.01316213912912\\
48.6136363636364	-0.998350638586288\\
48.6363636363636	-0.983465849291214\\
48.6590909090909	-0.968508863933777\\
48.6818181818182	-0.953480780503753\\
48.7045454545455	-0.938382702210224\\
48.7272727272727	-0.923215737400553\\
48.75	-0.907980999479093\\
48.7727272727273	-0.892679606825383\\
48.7954545454545	-0.877312682712079\\
48.8181818181818	-0.8618813552225\\
48.8409090909091	-0.846386757167784\\
48.8636363636364	-0.830830026003779\\
48.8863636363636	-0.815212303747485\\
48.9090909090909	-0.799534736893271\\
48.9318181818182	-0.783798476328677\\
48.9545454545455	-0.76800467724995\\
48.9772727272727	-0.752154499077205\\
49	-0.736249105369366\\
49.0227272727273	-0.720289663738688\\
49.0454545454545	-0.704277345765111\\
49.0681818181818	-0.688213326910195\\
49.0909090909091	-0.672098786430865\\
49.1136363636364	-0.655934907292811\\
49.1363636363636	-0.639722876083684\\
49.1590909090909	-0.623463882925965\\
49.1818181818182	-0.60715912138961\\
49.2045454545455	-0.590809788404405\\
49.2272727272727	-0.574417084172136\\
49.25	-0.557982212078467\\
49.2727272727273	-0.541506378604578\\
49.2954545454545	-0.524990793238628\\
49.3181818181818	-0.508436668386971\\
49.3409090909091	-0.491845219285126\\
49.3636363636364	-0.475217663908561\\
49.3863636363636	-0.458555222883332\\
49.4090909090909	-0.441859119396406\\
49.4318181818182	-0.425130579105957\\
49.4545454545455	-0.408370830051289\\
49.4772727272727	-0.39158110256277\\
49.5	-0.374762629171456\\
49.5227272727273	-0.35791664451865\\
49.5454545454545	-0.341044385265251\\
49.5681818181818	-0.324147090000974\\
49.5909090909091	-0.307225999153399\\
49.6136363636364	-0.290282354896976\\
49.6363636363636	-0.273317401061776\\
49.6590909090909	-0.256332383042188\\
49.6818181818182	-0.239328547705533\\
49.7045454545455	-0.222307143300498\\
49.7272727272727	-0.205269419365521\\
49.75	-0.188216626637033\\
49.7727272727273	-0.171150016957689\\
49.7954545454545	-0.154070843184411\\
49.8181818181818	-0.136980359096498\\
49.8409090909091	-0.119879819303486\\
49.8636363636364	-0.10277047915313\\
49.8863636363636	-0.0856535946392068\\
49.9090909090909	-0.0685304223093343\\
49.9318181818182	-0.0514022191726876\\
49.9545454545455	-0.0342702426078028\\
49.9772727272727	-0.0171357502701728\\
50	-1.46957615897682e-15\\
50.0227272727273	0.017135750270177\\
50.0454545454546	0.0342702426077999\\
50.0681818181818	0.0514022191726918\\
50.0909090909091	0.0685304223093314\\
50.1136363636364	0.085653594639211\\
50.1363636363636	0.102770479153127\\
50.1590909090909	0.119879819303483\\
50.1818181818182	0.136980359096495\\
50.2045454545455	0.154070843184415\\
50.2272727272727	0.171150016957686\\
50.25	0.188216626637037\\
50.2727272727273	0.205269419365518\\
50.2954545454546	0.222307143300502\\
50.3181818181818	0.239328547705537\\
50.3409090909091	0.256332383042193\\
50.3636363636364	0.273317401061773\\
50.3863636363636	0.29028235489698\\
50.4090909090909	0.307225999153403\\
50.4318181818182	0.324147090000972\\
50.4545454545455	0.341044385265255\\
50.4772727272727	0.357916644518654\\
50.5	0.37476262917146\\
50.5227272727273	0.391581102562767\\
50.5454545454546	0.408370830051293\\
50.5681818181818	0.425130579105954\\
50.5909090909091	0.441859119396411\\
50.6136363636364	0.458555222883329\\
50.6363636363636	0.475217663908565\\
50.6590909090909	0.491845219285123\\
50.6818181818182	0.508436668386975\\
50.7045454545455	0.524990793238632\\
50.7272727272727	0.541506378604575\\
50.75	0.557982212078464\\
50.7727272727273	0.57441708417214\\
50.7954545454546	0.590809788404409\\
50.8181818181818	0.607159121389608\\
50.8409090909091	0.623463882925969\\
50.8636363636364	0.639722876083688\\
50.8863636363636	0.655934907292815\\
50.9090909090909	0.672098786430863\\
50.9318181818182	0.688213326910199\\
50.9545454545455	0.704277345765108\\
50.9772727272727	0.720289663738692\\
51	0.736249105369364\\
51.0227272727273	0.752154499077209\\
51.0454545454546	0.768004677249947\\
51.0681818181818	0.783798476328681\\
51.0909090909091	0.799534736893268\\
51.1136363636364	0.815212303747489\\
51.1363636363636	0.830830026003776\\
51.1590909090909	0.846386757167788\\
51.1818181818182	0.861881355222503\\
51.2045454545455	0.877312682712083\\
51.2272727272727	0.892679606825386\\
51.25	0.907980999479097\\
51.2727272727273	0.923215737400557\\
51.2954545454546	0.938382702210221\\
51.3181818181818	0.953480780503757\\
51.3409090909091	0.968508863933774\\
51.3636363636364	0.983465849291217\\
51.3863636363636	0.998350638586286\\
51.4090909090909	1.01316213912912\\
51.4318181818182	1.02789926360996\\
51.4545454545455	1.04256093017897\\
51.4772727272727	1.05714606252568\\
51.5	1.071653589958\\
51.5227272727273	1.08608244748075\\
51.5454545454546	1.10043157587396\\
51.5681818181818	1.11469992177052\\
51.5909090909091	1.12888643773354\\
51.6136363636364	1.14299008233331\\
51.6363636363636	1.15700982022363\\
51.6590909090909	1.17094462221791\\
51.6818181818182	1.18479346536468\\
51.7045454545455	1.1985553330227\\
51.7272727272727	1.21222921493558\\
51.75	1.22581410730596\\
51.7727272727273	1.23930901286916\\
51.7954545454546	1.25271294096646\\
51.8181818181818	1.26602490761774\\
51.8409090909091	1.27924393559378\\
51.8636363636364	1.29236905448795\\
51.8863636363636	1.3053993007875\\
51.9090909090909	1.31833371794422\\
51.9318181818182	1.33117135644474\\
51.9545454545455	1.34391127388018\\
51.9772727272727	1.35655253501534\\
52	1.36909421185738\\
52.0227272727273	1.38153538372391\\
52.0454545454546	1.39387513731059\\
52.0681818181818	1.4061125667582\\
52.0909090909091	1.4182467737191\\
52.1136363636364	1.4302768674232\\
52.1363636363636	1.44220196474335\\
52.1590909090909	1.45402119026015\\
52.1818181818182	1.46573367632627\\
52.2045454545455	1.47733856313006\\
52.2272727272727	1.48883499875875\\
52.25	1.50022213926092\\
52.2727272727273	1.51149914870852\\
52.2954545454546	1.52266519925817\\
52.3181818181818	1.533719471212\\
52.3409090909091	1.54466115307774\\
52.3636363636364	1.5554894416284\\
52.3863636363636	1.56620354196111\\
52.4090909090909	1.57680266755559\\
52.4318181818182	1.58728604033181\\
52.4545454545455	1.59765289070713\\
52.4772727272727	1.60790245765282\\
52.5	1.6180339887499\\
52.5227272727273	1.62804674024437\\
52.5454545454546	1.63793997710183\\
52.5681818181818	1.64771297306143\\
52.5909090909091	1.65736501068918\\
52.6136363636364	1.66689538143064\\
52.6363636363636	1.67630338566288\\
52.6590909090909	1.68558833274589\\
52.6818181818182	1.69474954107329\\
52.7045454545455	1.7037863381223\\
52.7272727272727	1.71269806050318\\
52.75	1.72148405400789\\
52.7727272727273	1.73014367365813\\
52.7954545454546	1.73867628375267\\
52.8181818181818	1.74708125791404\\
52.8409090909091	1.75535797913451\\
52.8636363636364	1.76350583982137\\
52.8863636363636	1.77152424184154\\
52.9090909090909	1.77941259656547\\
52.9318181818182	1.78717032491037\\
52.9545454545455	1.79479685738272\\
52.9772727272727	1.80229163412004\\
53	1.80965410493204\\
53.0227272727273	1.81688372934098\\
53.0454545454546	1.82397997662134\\
53.0681818181818	1.83094232583881\\
53.0909090909091	1.83777026588851\\
53.1136363636364	1.8444632955325\\
53.1363636363636	1.85102092343661\\
53.1590909090909	1.85744266820648\\
53.1818181818182	1.86372805842291\\
53.2045454545455	1.86987663267645\\
53.2272727272727	1.8758879396013\\
53.25	1.88176153790845\\
53.2727272727273	1.88749699641803\\
53.2954545454546	1.89309389409098\\
53.3181818181818	1.89855182006001\\
53.3409090909091	1.90387037365967\\
53.3636363636364	1.90904916445585\\
53.3863636363636	1.91408781227439\\
53.4090909090909	1.918985947229\\
53.4318181818182	1.92374320974841\\
53.4545454545455	1.92835925060279\\
53.4772727272727	1.93283373092936\\
53.5	1.93716632225726\\
53.5227272727273	1.94135670653171\\
53.5454545454546	1.94540457613728\\
53.5681818181818	1.94930963392057\\
53.5909090909091	1.95307159321193\\
53.6136363636364	1.95669017784656\\
53.6363636363636	1.96016512218479\\
53.6590909090909	1.96349617113152\\
53.6818181818182	1.96668308015502\\
53.7045454545455	1.96972561530483\\
53.7272727272727	1.97262355322896\\
53.75	1.97537668119028\\
53.7727272727273	1.97798479708211\\
53.7954545454546	1.98044770944313\\
53.8181818181818	1.98276523747134\\
53.8409090909091	1.9849372110374\\
53.8636363636364	1.9869634706971\\
53.8863636363636	1.98884386770306\\
53.9090909090909	1.99057826401565\\
53.9318181818182	1.99216653231312\\
53.9545454545455	1.99360855600098\\
53.9772727272727	1.99490422922051\\
54	1.99605345685654\\
54.0227272727273	1.99705615454448\\
54.0454545454546	1.99791224867646\\
54.0681818181818	1.99862167640675\\
54.0909090909091	1.99918438565638\\
54.1136363636364	1.99960033511696\\
54.1363636363636	1.9998694942537\\
54.1590909090909	1.99999184330769\\
54.1818181818182	1.9999673732973\\
54.2045454545455	1.99979608601887\\
54.2272727272727	1.99947799404656\\
54.25	1.99901312073146\\
54.2727272727273	1.99840150019984\\
54.2954545454546	1.99764317735066\\
54.3181818181818	1.99673820785227\\
54.3409090909091	1.99568665813833\\
54.3636363636364	1.99448860540293\\
54.3863636363636	1.99314413759491\\
54.4090909090909	1.99165335341144\\
54.4318181818182	1.99001636229072\\
54.4545454545455	1.988233284404\\
54.4772727272727	1.98630425064673\\
54.5	1.98422940262896\\
54.5227272727273	1.98200889266491\\
54.5454545454546	1.97964288376186\\
54.5681818181818	1.97713154960813\\
54.5909090909091	1.97447507456033\\
54.6136363636364	1.97167365362984\\
54.6363636363636	1.9687274924685\\
54.6590909090909	1.96563680735351\\
54.6818181818182	1.96240182517154\\
54.7045454545455	1.9590227834021\\
54.7272727272727	1.95549993010006\\
54.75	1.95183352387749\\
54.7727272727273	1.94802383388467\\
54.7954545454546	1.94407113979028\\
54.8181818181818	1.93997573176093\\
54.8409090909091	1.93573791043985\\
54.8636363636364	1.93135798692477\\
54.8863636363636	1.92683628274516\\
54.9090909090909	1.92217312983854\\
54.9318181818182	1.91736887052619\\
54.9545454545455	1.91242385748797\\
54.9772727272727	1.90733845373647\\
55	1.90211303259031\\
55.0227272727273	1.89674797764677\\
55.0454545454546	1.89124368275365\\
55.0681818181818	1.88560055198029\\
55.0909090909091	1.87981899958799\\
55.1136363636364	1.87389944999952\\
55.1363636363636	1.86784233776803\\
55.1590909090909	1.8616481075451\\
55.1818181818182	1.85531721404814\\
55.2045454545455	1.84885012202695\\
55.2272727272727	1.8422473062297\\
55.25	1.83550925136796\\
55.2727272727273	1.82863645208121\\
55.2954545454546	1.8216294129005\\
55.3181818181818	1.81448864821137\\
55.3409090909091	1.80721468221618\\
55.3636363636364	1.79980804889554\\
55.3863636363636	1.79226929196914\\
55.4090909090909	1.78459896485588\\
55.4318181818182	1.77679763063316\\
55.4545454545455	1.76886586199563\\
55.4772727272727	1.76080424121307\\
55.5	1.75261336008772\\
55.5227272727273	1.74429381991079\\
55.5454545454546	1.7358462314183\\
55.5681818181818	1.72727121474631\\
55.5909090909091	1.71856939938535\\
55.6136363636364	1.70974142413423\\
55.6363636363636	1.70078793705311\\
55.6590909090909	1.69170959541598\\
55.6818181818182	1.68250706566236\\
55.7045454545455	1.67318102334842\\
55.7272727272727	1.66373215309735\\
55.75	1.65416114854912\\
55.7727272727273	1.64446871230957\\
55.7954545454546	1.63465555589881\\
55.8181818181818	1.624722399699\\
55.8409090909091	1.61466997290146\\
55.8636363636364	1.60449901345315\\
55.8863636363636	1.59421026800248\\
55.9090909090909	1.58380449184455\\
55.9318181818182	1.5732824488656\\
55.9545454545455	1.56264491148706\\
55.9772727272727	1.55189266060874\\
56	1.54102648555158\\
56.0227272727273	1.53004718399964\\
56.0454545454546	1.51895556194162\\
56.0681818181818	1.50775243361161\\
56.0909090909091	1.4964386214294\\
56.1136363636364	1.48501495594001\\
56.1363636363636	1.47348227575281\\
56.1590909090909	1.4618414274799\\
56.1818181818182	1.45009326567401\\
56.2045454545455	1.43823865276571\\
56.2272727272727	1.42627845900013\\
56.25	1.41421356237309\\
56.2727272727273	1.40204484856662\\
56.2954545454546	1.38977321088395\\
56.3181818181818	1.37739955018392\\
56.3409090909091	1.3649247748149\\
56.3636363636364	1.35234980054803\\
56.3863636363636	1.33967555051008\\
56.4090909090909	1.32690295511558\\
56.4318181818182	1.31403295199863\\
56.4545454545455	1.30106648594397\\
56.4772727272727	1.28800450881771\\
56.5	1.27484797949737\\
56.5227272727273	1.26159786380158\\
56.5454545454546	1.24825513441908\\
56.5681818181818	1.23482077083739\\
56.5909090909091	1.22129575927087\\
56.6136363636364	1.20768109258836\\
56.6363636363636	1.19397777024024\\
56.6590909090909	1.1801867981851\\
56.6818181818182	1.16630918881586\\
56.7045454545455	1.1523459608855\\
56.7272727272727	1.13829813943223\\
56.75	1.12416675570425\\
56.7727272727273	1.10995284708408\\
56.7954545454546	1.09565745701233\\
56.8181818181818	1.0812816349112\\
56.8409090909091	1.06682643610734\\
56.8636363636364	1.0522929217545\\
56.8863636363636	1.03768215875548\\
56.9090909090909	1.02299521968396\\
56.9318181818182	1.00823318270566\\
56.9545454545455	0.993397131499215\\
56.9772727272727	0.978488155176656\\
57	0.963507348203427\\
57.0227272727273	0.948455810318033\\
57.0454545454546	0.933334646451347\\
57.0681818181818	0.918144966645451\\
57.0909090909091	0.902887885972185\\
57.1136363636364	0.887564524451274\\
57.1363636363636	0.872176006968112\\
57.1590909090909	0.856723463191162\\
57.1818181818182	0.841208027489062\\
57.2045454545455	0.825630838847351\\
57.2272727272727	0.8099930407848\\
57.25	0.794295781269555\\
57.2727272727273	0.778540212634776\\
57.2954545454546	0.762727491494114\\
57.3181818181818	0.746858778656752\\
57.3409090909091	0.730935239042235\\
57.3636363636364	0.714958041594922\\
57.3863636363636	0.698928359198195\\
57.4090909090909	0.682847368588326\\
57.4318181818182	0.666716250268139\\
57.4545454545455	0.650536188420301\\
57.4772727272727	0.634308370820426\\
57.5	0.618033988749886\\
57.5227272727273	0.601714236908327\\
57.5454545454546	0.58535031332597\\
57.5681818181818	0.568943419275696\\
57.5909090909091	0.552494759184833\\
57.6136363636364	0.536005540546756\\
57.6363636363636	0.519476973832213\\
57.6590909090909	0.502910272400512\\
57.6818181818182	0.48630665241042\\
57.7045454545455	0.46966733273087\\
57.7272727272727	0.452993534851526\\
57.75	0.436286482793074\\
57.7727272727273	0.419547403017413\\
57.7954545454546	0.402777524337549\\
57.8181818181818	0.385978077827465\\
57.8409090909091	0.369150296731688\\
57.8636363636364	0.352295416374799\\
57.8863636363636	0.335414674070727\\
57.9090909090909	0.318509309031928\\
57.9318181818182	0.30158056227839\\
57.9545454545455	0.284629676546567\\
57.9772727272727	0.267657896198125\\
58	0.250666467128609\\
58.0227272727273	0.233656636675951\\
58.0454545454546	0.216629653528939\\
58.0681818181818	0.199586767635561\\
58.0909090909091	0.182529230111178\\
58.1136363636364	0.165458293146753\\
58.1363636363636	0.148375209916895\\
58.1590909090909	0.131281234487874\\
58.1818181818182	0.114177621725531\\
58.2045454545455	0.0970656272032147\\
58.2272727272727	0.0799465071095387\\
58.25	0.0628215181562544\\
58.2727272727273	0.0456919174858955\\
58.2954545454546	0.0285589625795693\\
58.3181818181818	0.0114239111646101\\
58.3409090909091	-0.00571197887773666\\
58.3636363636364	-0.0228474496046902\\
58.3863636363636	-0.039981243104202\\
58.4090909090909	-0.0571121015873982\\
58.4318181818182	-0.0742387674808356\\
58.4545454545455	-0.0913599835188484\\
58.4772727272727	-0.108474492835865\\
58.5	-0.125581039058631\\
58.5227272727273	-0.142678366398471\\
58.5454545454546	-0.159765219743456\\
58.5681818181818	-0.176840344750574\\
58.5909090909091	-0.193902487937767\\
58.6136363636364	-0.210950396775984\\
58.6363636363636	-0.227982819781088\\
58.6590909090909	-0.244998506605795\\
58.6818181818182	-0.261996208131388\\
58.7045454545455	-0.278974676559474\\
58.7272727272727	-0.295932665503531\\
58.75	-0.312868930080469\\
58.7727272727273	-0.329782227001928\\
58.7954545454546	-0.346671314665651\\
58.8181818181818	-0.363534953246544\\
58.8409090909091	-0.380371904787752\\
58.8636363636364	-0.397180933291498\\
58.8863636363636	-0.413960804809818\\
58.9090909090909	-0.430710287535174\\
58.9318181818182	-0.447428151890856\\
58.9545454545455	-0.464113170621269\\
58.9772727272727	-0.480764118881986\\
59	-0.497379774329719\\
59.0227272727273	-0.513958917211973\\
59.0454545454546	-0.530500330456681\\
59.0681818181818	-0.54700279976147\\
59.0909090909091	-0.563465113682861\\
59.1136363636364	-0.579886063725148\\
59.1363636363636	-0.596264444429173\\
59.1590909090909	-0.612599053460749\\
59.1818181818182	-0.628888691698983\\
59.2045454545455	-0.64513216332427\\
59.2272727272727	-0.661328275906106\\
59.25	-0.677475840490588\\
59.2727272727273	-0.693573671687718\\
59.2954545454546	-0.709620587758415\\
59.3181818181818	-0.725615410701284\\
59.3409090909091	-0.741556966339061\\
59.3636363636364	-0.757444084404825\\
59.3863636363636	-0.773275598627933\\
59.4090909090909	-0.789050346819567\\
59.4318181818182	-0.804767170958145\\
59.4545454545455	-0.820424917274228\\
59.4772727272727	-0.836022436335315\\
59.5	-0.851558583130152\\
59.5227272727273	-0.86703221715282\\
59.5454545454546	-0.882442202486446\\
59.5681818181818	-0.897787407886613\\
59.5909090909091	-0.913066706864364\\
59.6136363636364	-0.928278977768949\\
59.6363636363636	-0.943423103870116\\
59.6590909090909	-0.958497973440117\\
59.6818181818182	-0.973502479835311\\
59.7045454545455	-0.988435521577423\\
59.7272727272727	-1.00329600243437\\
59.75	-1.01808283150075\\
59.7727272727273	-1.03279492327793\\
59.7954545454546	-1.04743119775373\\
59.8181818181818	-1.06199058048172\\
59.8409090909091	-1.07647200266005\\
59.8636363636364	-1.09087440120996\\
59.8863636363636	-1.10519671885381\\
59.9090909090909	-1.11943790419266\\
59.9318181818182	-1.13359691178349\\
59.9545454545455	-1.14767270221596\\
59.9772727272727	-1.16166424218864\\
60	-1.17557050458495\\
60.0227272727273	-1.18939046854851\\
60.0454545454546	-1.20312311955808\\
60.0681818181818	-1.21676744950208\\
60.0909090909091	-1.23032245675253\\
60.1136363636364	-1.24378714623865\\
60.1363636363636	-1.25716052951984\\
60.1590909090909	-1.27044162485828\\
60.1818181818182	-1.28362945729101\\
60.2045454545455	-1.29672305870146\\
60.2272727272727	-1.30972146789057\\
60.25	-1.3226237306473\\
60.2727272727273	-1.33542889981873\\
60.2954545454546	-1.34813603537953\\
60.3181818181818	-1.36074420450102\\
60.3409090909091	-1.37325248161963\\
60.3636363636364	-1.38565994850485\\
60.3863636363636	-1.39796569432662\\
60.4090909090909	-1.41016881572221\\
60.4318181818182	-1.42226841686255\\
60.4545454545455	-1.43426360951793\\
60.4772727272727	-1.4461535131233\\
60.5	-1.45793725484283\\
60.5227272727273	-1.46961396963402\\
60.5454545454546	-1.48118280031122\\
60.5681818181818	-1.4926428976085\\
60.5909090909091	-1.50399342024206\\
60.6136363636364	-1.51523353497193\\
60.6363636363636	-1.5263624166632\\
60.6590909090909	-1.53737924834654\\
60.6818181818182	-1.54828322127817\\
60.7045454545455	-1.55907353499929\\
60.7272727272727	-1.56974939739478\\
60.75	-1.58031002475138\\
60.7727272727273	-1.59075464181523\\
60.7954545454546	-1.60108248184872\\
60.8181818181818	-1.61129278668688\\
60.8409090909091	-1.62138480679294\\
60.8636363636364	-1.63135780131341\\
60.8863636363636	-1.64121103813243\\
60.9090909090909	-1.65094379392555\\
60.9318181818182	-1.66055535421281\\
60.9545454545455	-1.67004501341119\\
60.9772727272727	-1.6794120748864\\
61	-1.68865585100403\\
61.0227272727273	-1.69777566318005\\
61.0454545454546	-1.70677084193057\\
61.0681818181818	-1.71564072692103\\
61.0909090909091	-1.72438466701468\\
61.1136363636364	-1.73300202032034\\
61.1363636363636	-1.74149215423956\\
61.1590909090909	-1.74985444551304\\
61.1818181818182	-1.75808828026637\\
61.2045454545455	-1.76619305405515\\
61.2272727272727	-1.77416817190929\\
61.25	-1.78201304837674\\
61.2727272727273	-1.78972710756642\\
61.2954545454546	-1.79730978319057\\
61.3181818181818	-1.80476051860623\\
61.3409090909091	-1.81207876685618\\
61.3636363636364	-1.81926399070904\\
61.3863636363636	-1.82631566269874\\
61.4090909090909	-1.83323326516323\\
61.4318181818182	-1.84001629028247\\
61.4545454545455	-1.84666424011574\\
61.4772727272727	-1.85317662663814\\
61.5	-1.8595529717765\\
61.5227272727273	-1.86579280744439\\
61.5454545454546	-1.87189567557652\\
61.5681818181818	-1.87786112816236\\
61.5909090909091	-1.88368872727905\\
61.6136363636364	-1.88937804512352\\
61.6363636363636	-1.89492866404391\\
61.6590909090909	-1.90034017657021\\
61.6818181818182	-1.9056121854442\\
61.7045454545455	-1.91074430364859\\
61.7272727272727	-1.91573615443547\\
61.75	-1.92058737135389\\
61.7727272727273	-1.92529759827684\\
61.7954545454546	-1.92986648942736\\
61.8181818181818	-1.93429370940392\\
61.8409090909091	-1.93857893320503\\
61.8636363636364	-1.94272184625316\\
61.8863636363636	-1.94672214441775\\
61.9090909090909	-1.95057953403757\\
61.9318181818182	-1.95429373194232\\
61.9545454545455	-1.95786446547334\\
61.9772727272727	-1.9612914725037\\
62	-1.96457450145738\\
62.0227272727273	-1.96771331132777\\
62.0454545454546	-1.97070767169539\\
62.0681818181818	-1.97355736274472\\
62.0909090909091	-1.97626217528045\\
62.1136363636364	-1.97882191074272\\
62.1363636363636	-1.9812363812218\\
62.1590909090909	-1.98350540947181\\
62.1818181818182	-1.98562882892376\\
62.2045454545455	-1.98760648369781\\
62.2272727272727	-1.98943822861464\\
62.25	-1.99112392920616\\
62.2727272727273	-1.99266346172538\\
62.2954545454546	-1.99405671315548\\
62.3181818181818	-1.99530358121811\\
62.3409090909091	-1.99640397438088\\
62.3636363636364	-1.99735781186412\\
62.3863636363636	-1.99816502364679\\
62.4090909090909	-1.9988255504716\\
62.4318181818182	-1.99933934384939\\
62.4545454545455	-1.99970636606268\\
62.4772727272727	-1.99992659016842\\
62.5	-2\\
62.5227272727273	-1.99992659016842\\
62.5454545454545	-1.99970636606268\\
62.5681818181818	-1.99933934384939\\
62.5909090909091	-1.9988255504716\\
62.6136363636364	-1.99816502364679\\
62.6363636363636	-1.99735781186412\\
62.6590909090909	-1.99640397438088\\
62.6818181818182	-1.9953035812181\\
62.7045454545455	-1.99405671315548\\
62.7272727272727	-1.99266346172538\\
62.75	-1.99112392920616\\
62.7727272727273	-1.98943822861464\\
62.7954545454545	-1.98760648369781\\
62.8181818181818	-1.98562882892376\\
62.8409090909091	-1.9835054094718\\
62.8636363636364	-1.9812363812218\\
62.8863636363636	-1.97882191074272\\
62.9090909090909	-1.97626217528044\\
62.9318181818182	-1.97355736274472\\
62.9545454545455	-1.97070767169539\\
62.9772727272727	-1.96771331132777\\
63	-1.96457450145738\\
63.0227272727273	-1.9612914725037\\
63.0454545454545	-1.95786446547335\\
63.0681818181818	-1.95429373194232\\
63.0909090909091	-1.95057953403757\\
63.1136363636364	-1.94672214441774\\
63.1363636363636	-1.94272184625316\\
63.1590909090909	-1.93857893320503\\
63.1818181818182	-1.93429370940391\\
63.2045454545455	-1.92986648942736\\
63.2272727272727	-1.92529759827684\\
63.25	-1.92058737135389\\
63.2727272727273	-1.91573615443546\\
63.2954545454545	-1.9107443036486\\
63.3181818181818	-1.90561218544419\\
63.3409090909091	-1.90034017657021\\
63.3636363636364	-1.8949286640439\\
63.3863636363636	-1.88937804512352\\
63.4090909090909	-1.88368872727905\\
63.4318181818182	-1.87786112816236\\
63.4545454545455	-1.87189567557651\\
63.4772727272727	-1.86579280744438\\
63.5	-1.8595529717765\\
63.5227272727273	-1.85317662663814\\
63.5454545454545	-1.84666424011573\\
63.5681818181818	-1.84001629028246\\
63.5909090909091	-1.83323326516322\\
63.6136363636364	-1.82631566269873\\
63.6363636363636	-1.81926399070904\\
63.6590909090909	-1.81207876685617\\
63.6818181818182	-1.80476051860623\\
63.7045454545455	-1.79730978319057\\
63.7272727272727	-1.78972710756642\\
63.75	-1.78201304837674\\
63.7727272727273	-1.77416817190929\\
63.7954545454545	-1.76619305405515\\
63.8181818181818	-1.75808828026637\\
63.8409090909091	-1.74985444551303\\
63.8636363636364	-1.74149215423955\\
63.8863636363636	-1.73300202032033\\
63.9090909090909	-1.72438466701467\\
63.9318181818182	-1.71564072692103\\
63.9545454545455	-1.70677084193056\\
63.9772727272727	-1.69777566318004\\
64	-1.68865585100403\\
64.0227272727273	-1.67941207488639\\
64.0454545454545	-1.67004501341119\\
64.0681818181818	-1.6605553542128\\
64.0909090909091	-1.65094379392554\\
64.1136363636364	-1.64121103813242\\
64.1363636363636	-1.6313578013134\\
64.1590909090909	-1.62138480679294\\
64.1818181818182	-1.61129278668688\\
64.2045454545455	-1.60108248184872\\
64.2272727272727	-1.59075464181522\\
64.25	-1.58031002475138\\
64.2727272727273	-1.56974939739477\\
64.2954545454545	-1.55907353499928\\
64.3181818181818	-1.54828322127816\\
64.3409090909091	-1.53737924834653\\
64.3636363636364	-1.5263624166632\\
64.3863636363636	-1.51523353497193\\
64.4090909090909	-1.50399342024205\\
64.4318181818182	-1.49264289760849\\
64.4545454545455	-1.48118280031121\\
64.4772727272727	-1.46961396963401\\
64.5	-1.45793725484282\\
64.5227272727273	-1.44615351312329\\
64.5454545454545	-1.43426360951793\\
64.5681818181818	-1.42226841686253\\
64.5909090909091	-1.41016881572221\\
64.6136363636364	-1.39796569432661\\
64.6363636363636	-1.38565994850484\\
64.6590909090909	-1.37325248161962\\
64.6818181818182	-1.36074420450101\\
64.7045454545455	-1.34813603537952\\
64.7272727272727	-1.33542889981872\\
64.75	-1.32262373064731\\
64.7727272727273	-1.30972146789057\\
64.7954545454545	-1.29672305870146\\
64.8181818181818	-1.283629457291\\
64.8409090909091	-1.27044162485827\\
64.8636363636364	-1.25716052951983\\
64.8863636363636	-1.24378714623864\\
64.9090909090909	-1.23032245675252\\
64.9318181818182	-1.21676744950207\\
64.9545454545455	-1.20312311955808\\
64.9772727272727	-1.18939046854849\\
65	-1.17557050458495\\
65.0227272727273	-1.16166424218863\\
65.0454545454545	-1.14767270221595\\
65.0681818181818	-1.13359691178348\\
65.0909090909091	-1.11943790419265\\
65.1136363636364	-1.1051967188538\\
65.1363636363636	-1.09087440120995\\
65.1590909090909	-1.07647200266004\\
65.1818181818182	-1.06199058048171\\
65.2045454545455	-1.04743119775373\\
65.2272727272727	-1.03279492327791\\
65.25	-1.01808283150074\\
65.2727272727273	-1.00329600243436\\
65.2954545454545	-0.988435521577426\\
65.3181818181818	-0.973502479835308\\
65.3409090909091	-0.958497973440114\\
65.3636363636364	-0.943423103870106\\
65.3863636363636	-0.92827897776894\\
65.4090909090909	-0.913066706864361\\
65.4318181818182	-0.897787407886603\\
65.4545454545455	-0.882442202486443\\
65.4772727272727	-0.867032217152805\\
65.5	-0.851558583130149\\
65.5227272727273	-0.836022436335299\\
65.5454545454545	-0.820424917274224\\
65.5681818181818	-0.804767170958129\\
65.5909090909091	-0.789050346819564\\
65.6136363636364	-0.773275598627917\\
65.6363636363636	-0.757444084404822\\
65.6590909090909	-0.741556966339051\\
65.6818181818182	-0.725615410701274\\
65.7045454545455	-0.709620587758412\\
65.7272727272727	-0.693573671687702\\
65.75	-0.677475840490584\\
65.7727272727273	-0.661328275906096\\
65.7954545454545	-0.645132163324273\\
65.8181818181818	-0.628888691698973\\
65.8409090909091	-0.612599053460752\\
65.8636363636364	-0.596264444429163\\
65.8863636363636	-0.579886063725151\\
65.9090909090909	-0.563465113682851\\
65.9318181818182	-0.547002799761467\\
65.9545454545455	-0.530500330456671\\
65.9772727272727	-0.513958917211969\\
66	-0.497379774329709\\
66.0227272727273	-0.480764118881976\\
66.0454545454545	-0.464113170621266\\
66.0681818181818	-0.447428151890846\\
66.0909090909091	-0.430710287535171\\
66.1136363636364	-0.413960804809807\\
66.1363636363636	-0.397180933291488\\
66.1590909090909	-0.380371904787749\\
66.1818181818182	-0.363534953246541\\
66.2045454545455	-0.346671314665641\\
66.2272727272727	-0.329782227001924\\
66.25	-0.312868930080458\\
66.2727272727273	-0.295932665503521\\
66.2954545454545	-0.278974676559471\\
66.3181818181818	-0.261996208131378\\
66.3409090909091	-0.244998506605792\\
66.3636363636364	-0.227982819781077\\
66.3863636363636	-0.210950396775966\\
66.4090909090909	-0.193902487937763\\
66.4318181818182	-0.17684034475057\\
66.4545454545455	-0.159765219743452\\
66.4772727272727	-0.14267836639846\\
66.5	-0.125581039058621\\
66.5227272727273	-0.108474492835847\\
66.5454545454545	-0.091359983518845\\
66.5681818181818	-0.074238767480818\\
66.5909090909091	-0.0571121015873948\\
66.6136363636364	-0.0399812431041915\\
66.6363636363636	-0.0228474496046726\\
66.6590909090909	-0.00571197887773323\\
66.6818181818182	0.0114239111646206\\
66.7045454545455	0.0285589625795728\\
66.7272727272727	0.045691917485906\\
66.75	0.0628215181562578\\
66.7727272727273	0.0799465071095563\\
66.7954545454545	0.097065627203211\\
66.8181818181818	0.114177621725549\\
66.8409090909091	0.131281234487877\\
66.8636363636364	0.148375209916913\\
66.8863636363636	0.165458293146764\\
66.9090909090909	0.182529230111188\\
66.9318181818182	0.199586767635565\\
66.9545454545455	0.21662965352895\\
66.9772727272727	0.233656636675954\\
67	0.250666467128605\\
67.0227272727273	0.267657896198135\\
67.0454545454545	0.28462967654657\\
67.0681818181818	0.3015805622784\\
67.0909090909091	0.318509309031925\\
67.1136363636364	0.335414674070744\\
67.1363636363636	0.352295416374809\\
67.1590909090909	0.369150296731698\\
67.1818181818182	0.385978077827469\\
67.2045454545455	0.402777524337559\\
67.2272727272727	0.419547403017416\\
67.25	0.436286482793084\\
67.2727272727273	0.452993534851536\\
67.2954545454545	0.469667332730874\\
67.3181818181818	0.48630665241043\\
67.3409090909091	0.502910272400515\\
67.3636363636364	0.519476973832223\\
67.3863636363636	0.536005540546766\\
67.4090909090909	0.552494759184843\\
67.4318181818182	0.568943419275706\\
67.4545454545455	0.58535031332598\\
67.4772727272727	0.601714236908331\\
67.5	0.618033988749896\\
67.5227272727273	0.634308370820443\\
67.5454545454545	0.650536188420297\\
67.5681818181818	0.666716250268142\\
67.5909090909091	0.682847368588329\\
67.6136363636364	0.698928359198205\\
67.6363636363636	0.714958041594939\\
67.6590909090909	0.730935239042245\\
67.6818181818182	0.746858778656762\\
67.7045454545455	0.762727491494117\\
67.7272727272727	0.778540212634785\\
67.75	0.794295781269558\\
67.7727272727273	0.809993040784822\\
67.7954545454545	0.825630838847354\\
67.8181818181818	0.841208027489078\\
67.8409090909091	0.856723463191165\\
67.8636363636364	0.872176006968121\\
67.8863636363636	0.887564524451277\\
67.9090909090909	0.902887885972195\\
67.9318181818182	0.91814496664546\\
67.9545454545455	0.93333464645135\\
67.9772727272727	0.948455810318042\\
68	0.96350734820343\\
68.0227272727273	0.978488155176671\\
68.0454545454545	0.993397131499218\\
68.0681818181818	1.00823318270567\\
68.0909090909091	1.02299521968397\\
68.1136363636364	1.03768215875549\\
68.1363636363636	1.0522929217545\\
68.1590909090909	1.06682643610735\\
68.1818181818182	1.0812816349112\\
68.2045454545455	1.09565745701234\\
68.2272727272727	1.10995284708408\\
68.25	1.12416675570426\\
68.2727272727273	1.13829813943224\\
68.2954545454545	1.1523459608855\\
68.3181818181818	1.16630918881587\\
68.3409090909091	1.18018679818509\\
68.3636363636364	1.19397777024025\\
68.3863636363636	1.20768109258836\\
68.4090909090909	1.22129575927088\\
68.4318181818182	1.23482077083739\\
68.4545454545455	1.24825513441909\\
68.4772727272727	1.26159786380158\\
68.5	1.27484797949738\\
68.5227272727273	1.28800450881771\\
68.5454545454545	1.30106648594397\\
68.5681818181818	1.31403295199863\\
68.5909090909091	1.32690295511558\\
68.6136363636364	1.33967555051008\\
68.6363636363636	1.35234980054803\\
68.6590909090909	1.3649247748149\\
68.6818181818182	1.37739955018392\\
68.7045454545455	1.38977321088395\\
68.7272727272727	1.40204484856663\\
68.75	1.4142135623731\\
68.7727272727273	1.42627845900014\\
68.7954545454545	1.43823865276571\\
68.8181818181818	1.45009326567402\\
68.8409090909091	1.46184142747991\\
68.8636363636364	1.47348227575281\\
68.8863636363637	1.48501495594002\\
68.9090909090909	1.4964386214294\\
68.9318181818182	1.50775243361163\\
68.9545454545455	1.51895556194162\\
68.9772727272727	1.53004718399965\\
69	1.54102648555158\\
69.0227272727273	1.55189266060875\\
69.0454545454545	1.56264491148707\\
69.0681818181818	1.57328244886562\\
69.0909090909091	1.58380449184455\\
69.1136363636364	1.5942102680025\\
69.1363636363637	1.60449901345315\\
69.1590909090909	1.61466997290146\\
69.1818181818182	1.62472239969901\\
69.2045454545455	1.63465555589881\\
69.2272727272727	1.64446871230958\\
69.25	1.65416114854912\\
69.2727272727273	1.66373215309735\\
69.2954545454545	1.67318102334842\\
69.3181818181818	1.68250706566236\\
69.3409090909091	1.69170959541598\\
69.3636363636364	1.70078793705312\\
69.3863636363637	1.70974142413424\\
69.4090909090909	1.71856939938536\\
69.4318181818182	1.72727121474632\\
69.4545454545455	1.7358462314183\\
69.4772727272727	1.74429381991079\\
69.5	1.75261336008773\\
69.5227272727273	1.76080424121308\\
69.5454545454545	1.76886586199563\\
69.5681818181818	1.77679763063316\\
69.5909090909091	1.78459896485588\\
69.6136363636364	1.79226929196915\\
69.6363636363636	1.79980804889554\\
69.6590909090909	1.80721468221619\\
69.6818181818182	1.81448864821138\\
69.7045454545455	1.8216294129005\\
69.7272727272727	1.82863645208122\\
69.75	1.83550925136796\\
69.7727272727273	1.8422473062297\\
69.7954545454545	1.84885012202696\\
69.8181818181818	1.85531721404814\\
69.8409090909091	1.8616481075451\\
69.8636363636364	1.86784233776803\\
69.8863636363636	1.87389944999952\\
69.9090909090909	1.87981899958799\\
69.9318181818182	1.8856005519803\\
69.9545454545455	1.89124368275365\\
69.9772727272727	1.89674797764678\\
70	1.90211303259031\\
70.0227272727273	1.90733845373647\\
70.0454545454545	1.91242385748797\\
70.0681818181818	1.91736887052619\\
70.0909090909091	1.92217312983854\\
70.1136363636364	1.92683628274516\\
70.1363636363636	1.93135798692478\\
70.1590909090909	1.93573791043985\\
70.1818181818182	1.93997573176094\\
70.2045454545455	1.94407113979028\\
70.2272727272727	1.94802383388467\\
70.25	1.95183352387749\\
70.2727272727273	1.95549993010006\\
70.2954545454545	1.9590227834021\\
70.3181818181818	1.96240182517155\\
70.3409090909091	1.96563680735351\\
70.3636363636364	1.9687274924685\\
70.3863636363636	1.97167365362984\\
70.4090909090909	1.97447507456033\\
70.4318181818182	1.97713154960813\\
70.4545454545455	1.97964288376187\\
70.4772727272727	1.98200889266491\\
70.5	1.98422940262896\\
70.5227272727273	1.98630425064673\\
70.5454545454545	1.988233284404\\
70.5681818181818	1.99001636229072\\
70.5909090909091	1.99165335341144\\
70.6136363636364	1.99314413759491\\
70.6363636363636	1.99448860540293\\
70.6590909090909	1.99568665813833\\
70.6818181818182	1.99673820785227\\
70.7045454545455	1.99764317735066\\
70.7272727272727	1.99840150019984\\
70.75	1.99901312073146\\
70.7727272727273	1.99947799404656\\
70.7954545454545	1.99979608601887\\
70.8181818181818	1.9999673732973\\
70.8409090909091	1.99999184330769\\
70.8636363636364	1.9998694942537\\
70.8863636363636	1.99960033511696\\
70.9090909090909	1.99918438565638\\
70.9318181818182	1.99862167640675\\
70.9545454545455	1.99791224867646\\
70.9772727272727	1.99705615454448\\
71	1.99605345685654\\
71.0227272727273	1.99490422922051\\
71.0454545454545	1.99360855600098\\
71.0681818181818	1.99216653231312\\
71.0909090909091	1.99057826401565\\
71.1136363636364	1.98884386770306\\
71.1363636363636	1.9869634706971\\
71.1590909090909	1.9849372110374\\
71.1818181818182	1.98276523747134\\
71.2045454545455	1.98044770944313\\
71.2272727272727	1.97798479708211\\
71.25	1.97537668119027\\
71.2727272727273	1.97262355322896\\
71.2954545454545	1.96972561530483\\
71.3181818181818	1.96668308015502\\
71.3409090909091	1.96349617113152\\
71.3636363636364	1.96016512218478\\
71.3863636363636	1.95669017784656\\
71.4090909090909	1.95307159321193\\
71.4318181818182	1.94930963392057\\
71.4545454545455	1.94540457613728\\
71.4772727272727	1.9413567065317\\
71.5	1.93716632225726\\
71.5227272727273	1.93283373092936\\
71.5454545454545	1.92835925060279\\
71.5681818181818	1.92374320974841\\
71.5909090909091	1.91898594722899\\
71.6136363636364	1.91408781227438\\
71.6363636363636	1.90904916445585\\
71.6590909090909	1.90387037365966\\
71.6818181818182	1.89855182006\\
71.7045454545455	1.89309389409098\\
71.7272727272727	1.88749699641802\\
71.75	1.88176153790845\\
71.7727272727273	1.8758879396013\\
71.7954545454545	1.86987663267644\\
71.8181818181818	1.8637280584229\\
71.8409090909091	1.85744266820648\\
71.8636363636364	1.85102092343661\\
71.8863636363636	1.8444632955325\\
71.9090909090909	1.8377702658885\\
71.9318181818182	1.83094232583881\\
71.9545454545455	1.82397997662134\\
71.9772727272727	1.81688372934098\\
72	1.80965410493204\\
72.0227272727273	1.80229163412004\\
72.0454545454545	1.79479685738272\\
72.0681818181818	1.78717032491037\\
72.0909090909091	1.77941259656547\\
72.1136363636364	1.77152424184154\\
72.1363636363636	1.76350583982137\\
72.1590909090909	1.75535797913451\\
72.1818181818182	1.74708125791403\\
72.2045454545455	1.73867628375266\\
72.2272727272727	1.73014367365812\\
72.25	1.72148405400788\\
72.2727272727273	1.71269806050318\\
72.2954545454545	1.7037863381223\\
72.3181818181818	1.69474954107329\\
72.3409090909091	1.68558833274589\\
72.3636363636364	1.67630338566287\\
72.3863636363636	1.66689538143063\\
72.4090909090909	1.65736501068918\\
72.4318181818182	1.64771297306143\\
72.4545454545455	1.63793997710182\\
72.4772727272727	1.62804674024436\\
72.5	1.61803398874989\\
72.5227272727273	1.60790245765282\\
72.5454545454545	1.59765289070712\\
72.5681818181818	1.5872860403318\\
72.5909090909091	1.57680266755559\\
72.6136363636364	1.56620354196111\\
72.6363636363636	1.55548944162839\\
72.6590909090909	1.54466115307774\\
72.6818181818182	1.533719471212\\
72.7045454545455	1.52266519925817\\
72.7272727272727	1.51149914870851\\
72.75	1.50022213926092\\
72.7727272727273	1.48883499875874\\
72.7954545454545	1.47733856313006\\
72.8181818181818	1.46573367632626\\
72.8409090909091	1.45402119026015\\
72.8636363636364	1.44220196474334\\
72.8863636363636	1.43027686742319\\
72.9090909090909	1.41824677371909\\
72.9318181818182	1.40611256675819\\
72.9545454545455	1.39387513731058\\
72.9772727272727	1.3815353837239\\
73	1.36909421185737\\
73.0227272727273	1.35655253501533\\
73.0454545454545	1.34391127388017\\
73.0681818181818	1.33117135644473\\
73.0909090909091	1.31833371794422\\
73.1136363636364	1.30539930078749\\
73.1363636363636	1.29236905448795\\
73.1590909090909	1.27924393559377\\
73.1818181818182	1.26602490761774\\
73.2045454545455	1.25271294096646\\
73.2272727272727	1.23930901286916\\
73.25	1.22581410730595\\
73.2727272727273	1.21222921493557\\
73.2954545454545	1.19855533302269\\
73.3181818181818	1.18479346536467\\
73.3409090909091	1.1709446222179\\
73.3636363636364	1.15700982022362\\
73.3863636363636	1.1429900823333\\
73.4090909090909	1.12888643773353\\
73.4318181818182	1.11469992177051\\
73.4545454545455	1.10043157587395\\
73.4772727272727	1.08608244748074\\
73.5	1.07165358995798\\
73.5227272727273	1.05714606252568\\
73.5454545454545	1.04256093017896\\
73.5681818181818	1.02789926360995\\
73.5909090909091	1.01316213912911\\
73.6136363636364	0.998350638586283\\
73.6363636363636	0.983465849291208\\
73.6590909090909	0.968508863933765\\
73.6818181818182	0.953480780503754\\
73.7045454545455	0.938382702210212\\
73.7272727272727	0.923215737400548\\
73.75	0.907980999479094\\
73.7727272727273	0.892679606825377\\
73.7954545454545	0.87731268271208\\
73.8181818181818	0.861881355222494\\
73.8409090909091	0.846386757167778\\
73.8636363636364	0.830830026003773\\
73.8863636363636	0.815212303747479\\
73.9090909090909	0.799534736893258\\
73.9318181818182	0.783798476328678\\
73.9545454545455	0.768004677249937\\
73.9772727272727	0.752154499077192\\
74	0.736249105369354\\
74.0227272727273	0.720289663738682\\
74.0454545454545	0.704277345765105\\
74.0681818181818	0.688213326910189\\
74.0909090909091	0.672098786430859\\
74.1136363636364	0.655934907292805\\
74.1363636363636	0.639722876083678\\
74.1590909090909	0.623463882925952\\
74.1818181818182	0.607159121389604\\
74.2045454545455	0.590809788404392\\
74.2272727272727	0.57441708417213\\
74.25	0.557982212078454\\
74.2727272727273	0.541506378604571\\
74.2954545454545	0.524990793238628\\
74.3181818181818	0.508436668386972\\
74.3409090909091	0.491845219285119\\
74.3636363636364	0.475217663908555\\
74.3863636363636	0.458555222883326\\
74.4090909090909	0.4418591193964\\
74.4318181818182	0.425130579105958\\
74.4545454545455	0.408370830051275\\
74.4772727272727	0.391581102562757\\
74.5	0.374762629171442\\
74.5227272727273	0.357916644518644\\
74.5454545454545	0.341044385265245\\
74.5681818181818	0.324147090000968\\
74.5909090909091	0.307225999153393\\
74.6136363636364	0.29028235489697\\
74.6363636363636	0.27331740106177\\
74.6590909090909	0.256332383042182\\
74.6818181818182	0.239328547705534\\
74.7045454545455	0.222307143300492\\
74.7272727272727	0.205269419365508\\
74.75	0.188216626637026\\
74.7727272727273	0.171150016957676\\
74.7954545454546	0.154070843184405\\
74.8181818181818	0.136980359096492\\
74.8409090909091	0.11987981930348\\
74.8636363636364	0.10277047915313\\
74.8863636363636	0.0856535946392005\\
74.9090909090909	0.0685304223093279\\
74.9318181818182	0.0514022191726884\\
74.9545454545455	0.0342702426077894\\
74.9772727272727	0.0171357502701736\\
75	-4.90106311913577e-15\\
75.0227272727273	-0.0171357502701763\\
75.0454545454546	-0.0342702426077992\\
75.0681818181818	-0.0514022191726911\\
75.0909090909091	-0.0685304223093306\\
75.1136363636364	-0.0856535946392031\\
75.1363636363636	-0.102770479153126\\
75.1590909090909	-0.119879819303483\\
75.1818181818182	-0.136980359096494\\
75.2045454545455	-0.154070843184422\\
75.2272727272727	-0.171150016957678\\
75.25	-0.188216626637036\\
75.2727272727273	-0.205269419365524\\
75.2954545454546	-0.222307143300501\\
75.3181818181818	-0.239328547705536\\
75.3409090909091	-0.256332383042185\\
75.3636363636364	-0.273317401061772\\
75.3863636363636	-0.290282354896972\\
75.4090909090909	-0.307225999153396\\
75.4318181818182	-0.324147090000964\\
75.4545454545455	-0.341044385265254\\
75.4772727272727	-0.357916644518647\\
75.5	-0.374762629171459\\
75.5227272727273	-0.391581102562767\\
75.5454545454546	-0.408370830051292\\
75.5681818181818	-0.42513057910596\\
75.5909090909091	-0.44185911939641\\
75.6136363636364	-0.458555222883329\\
75.6363636363636	-0.475217663908558\\
75.6590909090909	-0.491845219285122\\
75.6818181818182	-0.508436668386968\\
75.7045454545455	-0.524990793238631\\
75.7272727272727	-0.541506378604567\\
75.75	-0.55798221207847\\
75.7727272727273	-0.574417084172139\\
75.7954545454546	-0.590809788404408\\
75.8181818181818	-0.607159121389607\\
75.8409090909091	-0.623463882925968\\
75.8636363636364	-0.639722876083681\\
75.8863636363636	-0.655934907292808\\
75.9090909090909	-0.672098786430862\\
75.9318181818182	-0.688213326910192\\
75.9545454545455	-0.704277345765114\\
75.9772727272727	-0.720289663738678\\
76	-0.736249105369357\\
76.0227272727273	-0.752154499077208\\
76.0454545454546	-0.768004677249953\\
76.0681818181818	-0.78379847632868\\
76.0909090909091	-0.799534736893274\\
76.1136363636364	-0.815212303747482\\
76.1363636363636	-0.830830026003775\\
76.1590909090909	-0.846386757167781\\
76.1818181818182	-0.861881355222496\\
76.2045454545455	-0.877312682712088\\
76.2272727272727	-0.892679606825379\\
76.25	-0.907980999479096\\
76.2727272727273	-0.923215737400556\\
76.2954545454546	-0.93838270221022\\
76.3181818181818	-0.953480780503757\\
76.3409090909091	-0.968508863933774\\
76.3636363636364	-0.983465849291211\\
76.3863636363636	-0.998350638586285\\
76.4090909090909	-1.01316213912912\\
76.4318181818182	-1.02789926360996\\
76.4545454545455	-1.04256093017898\\
76.4772727272727	-1.05714606252568\\
76.5	-1.071653589958\\
76.5227272727273	-1.08608244748076\\
76.5454545454546	-1.10043157587396\\
76.5681818181818	-1.11469992177051\\
76.5909090909091	-1.12888643773354\\
76.6136363636364	-1.14299008233331\\
76.6363636363636	-1.15700982022362\\
76.6590909090909	-1.1709446222179\\
76.6818181818182	-1.18479346536467\\
76.7045454545455	-1.1985553330227\\
76.7272727272727	-1.21222921493558\\
76.75	-1.22581410730596\\
76.7727272727273	-1.23930901286916\\
76.7954545454546	-1.25271294096646\\
76.8181818181818	-1.26602490761774\\
76.8409090909091	-1.27924393559378\\
76.8636363636364	-1.29236905448795\\
76.8863636363636	-1.30539930078749\\
76.9090909090909	-1.31833371794422\\
76.9318181818182	-1.33117135644474\\
76.9545454545455	-1.34391127388018\\
76.9772727272727	-1.35655253501534\\
77	-1.36909421185739\\
77.0227272727273	-1.38153538372391\\
77.0454545454546	-1.39387513731059\\
77.0681818181818	-1.4061125667582\\
77.0909090909091	-1.41824677371909\\
77.1136363636364	-1.43027686742319\\
77.1363636363636	-1.44220196474334\\
77.1590909090909	-1.45402119026015\\
77.1818181818182	-1.46573367632626\\
77.2045454545455	-1.47733856313006\\
77.2272727272727	-1.48883499875874\\
77.25	-1.50022213926092\\
77.2727272727273	-1.51149914870852\\
77.2954545454546	-1.52266519925817\\
77.3181818181818	-1.533719471212\\
77.3409090909091	-1.54466115307775\\
77.3636363636364	-1.5554894416284\\
77.3863636363636	-1.56620354196111\\
77.4090909090909	-1.57680266755559\\
77.4318181818182	-1.58728604033181\\
77.4545454545455	-1.59765289070713\\
77.4772727272727	-1.60790245765282\\
77.5	-1.6180339887499\\
77.5227272727273	-1.62804674024437\\
77.5454545454546	-1.63793997710183\\
77.5681818181818	-1.64771297306143\\
77.5909090909091	-1.65736501068918\\
77.6136363636364	-1.66689538143064\\
77.6363636363636	-1.67630338566287\\
77.6590909090909	-1.68558833274589\\
77.6818181818182	-1.69474954107328\\
77.7045454545455	-1.7037863381223\\
77.7272727272727	-1.71269806050318\\
77.75	-1.72148405400789\\
77.7727272727273	-1.73014367365812\\
77.7954545454546	-1.73867628375267\\
77.8181818181818	-1.74708125791404\\
77.8409090909091	-1.75535797913451\\
77.8636363636364	-1.76350583982137\\
77.8863636363636	-1.77152424184154\\
77.9090909090909	-1.77941259656547\\
77.9318181818182	-1.78717032491037\\
77.9545454545455	-1.79479685738272\\
77.9772727272727	-1.80229163412004\\
78	-1.80965410493204\\
78.0227272727273	-1.81688372934098\\
78.0454545454546	-1.82397997662134\\
78.0681818181818	-1.83094232583881\\
78.0909090909091	-1.83777026588851\\
78.1136363636364	-1.84446329553251\\
78.1363636363636	-1.85102092343661\\
78.1590909090909	-1.85744266820648\\
78.1818181818182	-1.8637280584229\\
78.2045454545455	-1.86987663267645\\
78.2272727272727	-1.8758879396013\\
78.25	-1.88176153790845\\
78.2727272727273	-1.88749699641802\\
78.2954545454546	-1.89309389409098\\
78.3181818181818	-1.89855182006001\\
78.3409090909091	-1.90387037365967\\
78.3636363636364	-1.90904916445585\\
78.3863636363636	-1.91408781227439\\
78.4090909090909	-1.91898594722899\\
78.4318181818182	-1.92374320974841\\
78.4545454545455	-1.92835925060279\\
78.4772727272727	-1.93283373092936\\
78.5	-1.93716632225726\\
78.5227272727273	-1.9413567065317\\
78.5454545454546	-1.94540457613728\\
78.5681818181818	-1.94930963392057\\
78.5909090909091	-1.95307159321193\\
78.6136363636364	-1.95669017784656\\
78.6363636363636	-1.96016512218479\\
78.6590909090909	-1.96349617113152\\
78.6818181818182	-1.96668308015502\\
78.7045454545455	-1.96972561530483\\
78.7272727272727	-1.97262355322896\\
78.75	-1.97537668119028\\
78.7727272727273	-1.97798479708211\\
78.7954545454546	-1.98044770944313\\
78.8181818181818	-1.98276523747134\\
78.8409090909091	-1.9849372110374\\
78.8636363636364	-1.9869634706971\\
78.8863636363636	-1.98884386770306\\
78.9090909090909	-1.99057826401565\\
78.9318181818182	-1.99216653231312\\
78.9545454545455	-1.99360855600098\\
78.9772727272727	-1.99490422922051\\
79	-1.99605345685654\\
79.0227272727273	-1.99705615454448\\
79.0454545454546	-1.99791224867646\\
79.0681818181818	-1.99862167640675\\
79.0909090909091	-1.99918438565638\\
79.1136363636364	-1.99960033511696\\
79.1363636363636	-1.9998694942537\\
79.1590909090909	-1.99999184330769\\
79.1818181818182	-1.9999673732973\\
79.2045454545455	-1.99979608601887\\
79.2272727272727	-1.99947799404656\\
79.25	-1.99901312073146\\
79.2727272727273	-1.99840150019984\\
79.2954545454546	-1.99764317735066\\
79.3181818181818	-1.99673820785227\\
79.3409090909091	-1.99568665813833\\
79.3636363636364	-1.99448860540293\\
79.3863636363636	-1.99314413759491\\
79.4090909090909	-1.99165335341144\\
79.4318181818182	-1.99001636229072\\
79.4545454545455	-1.988233284404\\
79.4772727272727	-1.98630425064673\\
79.5	-1.98422940262895\\
79.5227272727273	-1.98200889266491\\
79.5454545454546	-1.97964288376186\\
79.5681818181818	-1.97713154960813\\
79.5909090909091	-1.97447507456032\\
79.6136363636364	-1.97167365362984\\
79.6363636363636	-1.9687274924685\\
79.6590909090909	-1.96563680735351\\
79.6818181818182	-1.96240182517154\\
79.7045454545455	-1.9590227834021\\
79.7272727272727	-1.95549993010006\\
79.75	-1.95183352387749\\
79.7727272727273	-1.94802383388467\\
79.7954545454546	-1.94407113979028\\
79.8181818181818	-1.93997573176093\\
79.8409090909091	-1.93573791043985\\
79.8636363636364	-1.93135798692478\\
79.8863636363636	-1.92683628274516\\
79.9090909090909	-1.92217312983854\\
79.9318181818182	-1.91736887052619\\
79.9545454545455	-1.91242385748797\\
79.9772727272727	-1.90733845373647\\
80	-1.90211303259031\\
80.0227272727273	-1.89674797764677\\
80.0454545454546	-1.89124368275365\\
80.0681818181818	-1.88560055198029\\
80.0909090909091	-1.87981899958799\\
80.1136363636364	-1.87389944999952\\
80.1363636363636	-1.86784233776803\\
80.1590909090909	-1.86164810754511\\
80.1818181818182	-1.85531721404814\\
80.2045454545455	-1.84885012202696\\
80.2272727272727	-1.8422473062297\\
80.25	-1.83550925136796\\
80.2727272727273	-1.82863645208121\\
80.2954545454546	-1.82162941290049\\
80.3181818181818	-1.81448864821138\\
80.3409090909091	-1.80721468221618\\
80.3636363636364	-1.79980804889553\\
80.3863636363636	-1.79226929196914\\
80.4090909090909	-1.78459896485588\\
80.4318181818182	-1.77679763063316\\
80.4545454545455	-1.76886586199563\\
80.4772727272727	-1.76080424121308\\
80.5	-1.75261336008773\\
80.5227272727273	-1.74429381991079\\
80.5454545454546	-1.73584623141829\\
80.5681818181818	-1.72727121474632\\
80.5909090909091	-1.71856939938536\\
80.6136363636364	-1.70974142413423\\
80.6363636363636	-1.70078793705311\\
80.6590909090909	-1.69170959541598\\
80.6818181818182	-1.68250706566236\\
80.7045454545455	-1.67318102334842\\
80.7272727272727	-1.66373215309735\\
80.75	-1.65416114854912\\
80.7727272727273	-1.64446871230957\\
80.7954545454546	-1.63465555589881\\
80.8181818181818	-1.624722399699\\
80.8409090909091	-1.61466997290146\\
80.8636363636364	-1.60449901345314\\
80.8863636363636	-1.59421026800249\\
80.9090909090909	-1.58380449184455\\
80.9318181818182	-1.57328244886561\\
80.9545454545455	-1.56264491148706\\
80.9772727272727	-1.55189266060874\\
81	-1.54102648555158\\
81.0227272727273	-1.53004718399965\\
81.0454545454546	-1.51895556194162\\
81.0681818181818	-1.50775243361162\\
81.0909090909091	-1.4964386214294\\
81.1136363636364	-1.48501495594001\\
81.1363636363636	-1.4734822757528\\
81.1590909090909	-1.4618414274799\\
81.1818181818182	-1.45009326567402\\
81.2045454545455	-1.43823865276571\\
81.2272727272727	-1.42627845900013\\
81.25	-1.4142135623731\\
81.2727272727273	-1.40204484856663\\
81.2954545454546	-1.38977321088394\\
81.3181818181818	-1.37739955018392\\
81.3409090909091	-1.36492477481489\\
81.3636363636364	-1.35234980054804\\
81.3863636363636	-1.33967555051007\\
81.4090909090909	-1.32690295511558\\
81.4318181818182	-1.31403295199863\\
81.4545454545455	-1.30106648594398\\
81.4772727272727	-1.28800450881771\\
81.5	-1.27484797949738\\
81.5227272727273	-1.26159786380159\\
81.5454545454546	-1.24825513441908\\
81.5681818181818	-1.23482077083739\\
81.5909090909091	-1.22129575927087\\
81.6136363636364	-1.20768109258837\\
81.6363636363636	-1.19397777024024\\
81.6590909090909	-1.18018679818509\\
81.6818181818182	-1.16630918881586\\
81.7045454545455	-1.15234596088551\\
81.7272727272727	-1.13829813943223\\
81.75	-1.12416675570426\\
81.7727272727273	-1.10995284708408\\
81.7954545454546	-1.09565745701233\\
81.8181818181818	-1.0812816349112\\
81.8409090909091	-1.06682643610734\\
81.8636363636364	-1.0522929217545\\
81.8863636363636	-1.03768215875548\\
81.9090909090909	-1.02299521968396\\
81.9318181818182	-1.00823318270566\\
81.9545454545455	-0.993397131499222\\
81.9772727272727	-0.978488155176657\\
82	-0.963507348203428\\
82.0227272727273	-0.94845581031804\\
82.0454545454546	-0.933334646451342\\
82.0681818181818	-0.918144966645451\\
82.0909090909091	-0.90288788597218\\
82.1136363636364	-0.887564524451275\\
82.1363636363636	-0.872176006968112\\
82.1590909090909	-0.856723463191156\\
82.1818181818182	-0.841208027489069\\
82.2045454545455	-0.825630838847352\\
82.2272727272727	-0.809993040784813\\
82.25	-0.794295781269556\\
82.2727272727273	-0.778540212634783\\
82.2954545454546	-0.762727491494108\\
82.3181818181818	-0.746858778656753\\
82.3409090909091	-0.730935239042229\\
82.3636363636364	-0.714958041594923\\
82.3863636363636	-0.698928359198196\\
82.4090909090909	-0.682847368588333\\
82.4318181818182	-0.666716250268133\\
82.4545454545455	-0.650536188420295\\
82.4772727272727	-0.634308370820434\\
82.5	-0.618033988749887\\
82.5227272727273	-0.601714236908328\\
82.5454545454546	-0.585350313325964\\
82.5681818181818	-0.56894341927571\\
82.5909090909091	-0.552494759184827\\
82.6136363636364	-0.536005540546749\\
82.6363636363636	-0.519476973832214\\
82.6590909090909	-0.502910272400519\\
82.6818181818182	-0.486306652410413\\
82.7045454545455	-0.469667332730871\\
82.7272727272727	-0.452993534851534\\
82.75	-0.436286482793082\\
82.7727272727273	-0.419547403017413\\
82.7954545454545	-0.402777524337556\\
82.8181818181818	-0.385978077827473\\
82.8409090909091	-0.369150296731688\\
82.8636363636364	-0.352295416374792\\
82.8863636363636	-0.335414674070727\\
82.9090909090909	-0.318509309031929\\
82.9318181818182	-0.301580562278384\\
82.9545454545455	-0.284629676546567\\
82.9772727272727	-0.267657896198133\\
83	-0.25066646712861\\
83.0227272727273	-0.233656636675944\\
83.0454545454545	-0.216629653528947\\
83.0681818181818	-0.199586767635569\\
83.0909090909091	-0.182529230111179\\
83.1136363636364	-0.165458293146761\\
83.1363636363636	-0.148375209916896\\
83.1590909090909	-0.131281234487874\\
83.1818181818182	-0.114177621725539\\
83.2045454545455	-0.0970656272032083\\
83.2272727272727	-0.0799465071095465\\
83.25	-0.0628215181562622\\
83.2727272727273	-0.0456919174858962\\
83.2954545454545	-0.0285589625795772\\
83.3181818181818	-0.0114239111646179\\
83.3409090909091	0.00571197887774303\\
83.3636363636364	0.0228474496046824\\
83.3863636363637	0.0399812431042084\\
83.4090909090909	0.0571121015873975\\
83.4318181818182	0.0742387674808278\\
83.4545454545455	0.0913599835188548\\
83.4772727272727	0.108474492835864\\
83.5	0.125581039058624\\
83.5227272727273	0.142678366398456\\
83.5454545454545	0.159765219743455\\
83.5681818181818	0.176840344750573\\
83.5909090909091	0.193902487937773\\
83.6136363636364	0.210950396775976\\
83.6363636363637	0.227982819781094\\
83.6590909090909	0.244998506605795\\
83.6818181818182	0.261996208131387\\
83.7045454545455	0.27897467655948\\
83.7272727272727	0.295932665503538\\
83.75	0.312868930080461\\
83.7727272727273	0.32978222700192\\
83.7954545454545	0.346671314665651\\
83.8181818181818	0.363534953246543\\
83.8409090909091	0.380371904787758\\
83.8636363636364	0.397180933291484\\
83.8863636363637	0.413960804809817\\
83.9090909090909	0.430710287535174\\
83.9318181818182	0.447428151890855\\
83.9545454545455	0.464113170621262\\
83.9772727272727	0.480764118881992\\
84	0.497379774329711\\
84.0227272727273	0.513958917211965\\
84.0454545454545	0.53050033045668\\
84.0681818181818	0.547002799761469\\
84.0909090909091	0.563465113682867\\
84.1136363636364	0.579886063725147\\
84.1363636363637	0.596264444429173\\
84.1590909090909	0.612599053460755\\
84.1818181818182	0.628888691698983\\
84.2045454545455	0.645132163324269\\
84.2272727272727	0.661328275906112\\
84.25	0.677475840490587\\
84.2727272727273	0.693573671687711\\
84.2954545454545	0.709620587758408\\
84.3181818181818	0.725615410701284\\
84.3409090909091	0.741556966339067\\
84.3636363636364	0.757444084404817\\
84.3863636363637	0.773275598627926\\
84.4090909090909	0.789050346819573\\
84.4318181818182	0.804767170958138\\
84.4545454545455	0.820424917274227\\
84.4772727272727	0.836022436335307\\
84.5	0.851558583130151\\
84.5227272727273	0.867032217152813\\
84.5454545454545	0.882442202486439\\
84.5681818181818	0.897787407886612\\
84.5909090909091	0.913066706864364\\
84.6136363636364	0.928278977768942\\
84.6363636363637	0.943423103870115\\
84.6590909090909	0.958497973440123\\
84.6818181818182	0.973502479835317\\
84.7045454545455	0.988435521577428\\
84.7272727272727	1.00329600243437\\
84.75	1.01808283150075\\
84.7727272727273	1.03279492327793\\
84.7954545454545	1.04743119775373\\
84.8181818181818	1.06199058048172\\
84.8409090909091	1.07647200266005\\
84.8636363636364	1.09087440120996\\
84.8863636363637	1.10519671885381\\
84.9090909090909	1.11943790419265\\
84.9318181818182	1.1335969117835\\
84.9545454545455	1.14767270221596\\
84.9772727272727	1.16166424218863\\
85	1.17557050458495\\
85.0227272727273	1.1893904685485\\
85.0454545454545	1.20312311955807\\
85.0681818181818	1.21676744950206\\
85.0909090909091	1.23032245675253\\
85.1136363636364	1.24378714623865\\
85.1363636363637	1.25716052951984\\
85.1590909090909	1.27044162485828\\
85.1818181818182	1.28362945729102\\
85.2045454545455	1.29672305870147\\
85.2272727272727	1.30972146789057\\
85.25	1.3226237306473\\
85.2727272727273	1.33542889981873\\
85.2954545454545	1.34813603537952\\
85.3181818181818	1.36074420450101\\
85.3409090909091	1.37325248161963\\
85.3636363636364	1.38565994850485\\
85.3863636363637	1.39796569432662\\
85.4090909090909	1.41016881572221\\
85.4318181818182	1.42226841686254\\
85.4545454545455	1.43426360951793\\
85.4772727272727	1.4461535131233\\
85.5	1.45793725484282\\
85.5227272727273	1.46961396963402\\
85.5454545454545	1.48118280031121\\
85.5681818181818	1.49264289760849\\
85.5909090909091	1.50399342024204\\
85.6136363636364	1.51523353497193\\
85.6363636363637	1.5263624166632\\
85.6590909090909	1.53737924834653\\
85.6818181818182	1.54828322127816\\
85.7045454545455	1.55907353499929\\
85.7272727272727	1.56974939739478\\
85.75	1.58031002475138\\
85.7727272727273	1.59075464181523\\
85.7954545454545	1.60108248184873\\
85.8181818181818	1.61129278668688\\
85.8409090909091	1.62138480679294\\
85.8636363636364	1.63135780131341\\
85.8863636363637	1.64121103813243\\
85.9090909090909	1.65094379392555\\
85.9318181818182	1.66055535421281\\
85.9545454545455	1.67004501341119\\
85.9772727272727	1.6794120748864\\
86	1.68865585100403\\
86.0227272727273	1.69777566318004\\
86.0454545454545	1.70677084193057\\
86.0681818181818	1.71564072692103\\
86.0909090909091	1.72438466701467\\
86.1136363636364	1.73300202032034\\
86.1363636363637	1.74149215423956\\
86.1590909090909	1.74985444551303\\
86.1818181818182	1.75808828026637\\
86.2045454545455	1.76619305405515\\
86.2272727272727	1.77416817190929\\
86.25	1.78201304837674\\
86.2727272727273	1.78972710756643\\
86.2954545454545	1.79730978319057\\
86.3181818181818	1.80476051860623\\
86.3409090909091	1.81207876685618\\
86.3636363636364	1.81926399070904\\
86.3863636363637	1.82631566269874\\
86.4090909090909	1.83323326516323\\
86.4318181818182	1.84001629028247\\
86.4545454545455	1.84666424011573\\
86.4772727272727	1.85317662663815\\
86.5	1.8595529717765\\
86.5227272727273	1.86579280744439\\
86.5454545454545	1.87189567557651\\
86.5681818181818	1.87786112816236\\
86.5909090909091	1.88368872727905\\
86.6136363636364	1.88937804512352\\
86.6363636363637	1.89492866404391\\
86.6590909090909	1.9003401765702\\
86.6818181818182	1.9056121854442\\
86.7045454545455	1.91074430364859\\
86.7272727272727	1.91573615443547\\
86.75	1.92058737135389\\
86.7727272727273	1.92529759827684\\
86.7954545454545	1.92986648942736\\
86.8181818181818	1.93429370940391\\
86.8409090909091	1.93857893320504\\
86.8636363636364	1.94272184625316\\
86.8863636363637	1.94672214441775\\
86.9090909090909	1.95057953403757\\
86.9318181818182	1.95429373194232\\
86.9545454545455	1.95786446547335\\
86.9772727272727	1.9612914725037\\
87	1.96457450145738\\
87.0227272727273	1.96771331132777\\
87.0454545454545	1.97070767169539\\
87.0681818181818	1.97355736274472\\
87.0909090909091	1.97626217528045\\
87.1136363636364	1.97882191074272\\
87.1363636363636	1.9812363812218\\
87.1590909090909	1.98350540947181\\
87.1818181818182	1.98562882892377\\
87.2045454545455	1.98760648369781\\
87.2272727272727	1.98943822861464\\
87.25	1.99112392920616\\
87.2727272727273	1.99266346172538\\
87.2954545454545	1.99405671315548\\
87.3181818181818	1.9953035812181\\
87.3409090909091	1.99640397438088\\
87.3636363636364	1.99735781186412\\
87.3863636363636	1.99816502364679\\
87.4090909090909	1.9988255504716\\
87.4318181818182	1.99933934384939\\
87.4545454545455	1.99970636606268\\
87.4772727272727	1.99992659016842\\
87.5	2\\
87.5227272727273	1.99992659016842\\
87.5454545454545	1.99970636606268\\
87.5681818181818	1.99933934384939\\
87.5909090909091	1.9988255504716\\
87.6136363636364	1.99816502364679\\
87.6363636363636	1.99735781186412\\
87.6590909090909	1.99640397438088\\
87.6818181818182	1.9953035812181\\
87.7045454545455	1.99405671315548\\
87.7272727272727	1.99266346172538\\
87.75	1.99112392920616\\
87.7727272727273	1.98943822861464\\
87.7954545454545	1.98760648369781\\
87.8181818181818	1.98562882892377\\
87.8409090909091	1.98350540947181\\
87.8636363636364	1.9812363812218\\
87.8863636363636	1.97882191074272\\
87.9090909090909	1.97626217528045\\
87.9318181818182	1.97355736274472\\
87.9545454545455	1.97070767169539\\
87.9772727272727	1.96771331132777\\
88	1.96457450145738\\
88.0227272727273	1.9612914725037\\
88.0454545454545	1.95786446547334\\
88.0681818181818	1.95429373194232\\
88.0909090909091	1.95057953403757\\
88.1136363636364	1.94672214441774\\
88.1363636363636	1.94272184625316\\
88.1590909090909	1.93857893320504\\
88.1818181818182	1.93429370940391\\
88.2045454545455	1.92986648942735\\
88.2272727272727	1.92529759827684\\
88.25	1.92058737135389\\
88.2727272727273	1.91573615443547\\
88.2954545454545	1.91074430364859\\
88.3181818181818	1.9056121854442\\
88.3409090909091	1.90034017657021\\
88.3636363636364	1.89492866404391\\
88.3863636363636	1.88937804512352\\
88.4090909090909	1.88368872727905\\
88.4318181818182	1.87786112816236\\
88.4545454545455	1.87189567557652\\
88.4772727272727	1.86579280744439\\
88.5	1.8595529717765\\
88.5227272727273	1.85317662663814\\
88.5454545454545	1.84666424011573\\
88.5681818181818	1.84001629028246\\
88.5909090909091	1.83323326516322\\
88.6136363636364	1.82631566269874\\
88.6363636363636	1.81926399070903\\
88.6590909090909	1.81207876685618\\
88.6818181818182	1.80476051860623\\
88.7045454545455	1.79730978319057\\
88.7272727272727	1.78972710756642\\
88.75	1.78201304837673\\
88.7727272727273	1.77416817190929\\
88.7954545454545	1.76619305405515\\
88.8181818181818	1.75808828026637\\
88.8409090909091	1.74985444551303\\
88.8636363636364	1.74149215423956\\
88.8863636363636	1.73300202032034\\
88.9090909090909	1.72438466701468\\
88.9318181818182	1.71564072692103\\
88.9545454545455	1.70677084193056\\
88.9772727272727	1.69777566318004\\
89	1.68865585100403\\
89.0227272727273	1.6794120748864\\
89.0454545454545	1.67004501341119\\
89.0681818181818	1.66055535421281\\
89.0909090909091	1.65094379392554\\
89.1136363636364	1.64121103813242\\
89.1363636363636	1.6313578013134\\
89.1590909090909	1.62138480679294\\
89.1818181818182	1.61129278668687\\
89.2045454545455	1.60108248184872\\
89.2272727272727	1.59075464181522\\
89.25	1.58031002475137\\
89.2727272727273	1.56974939739477\\
89.2954545454545	1.55907353499928\\
89.3181818181818	1.54828322127816\\
89.3409090909091	1.53737924834653\\
89.3636363636364	1.5263624166632\\
89.3863636363636	1.51523353497194\\
89.4090909090909	1.50399342024205\\
89.4318181818182	1.49264289760849\\
89.4545454545455	1.48118280031122\\
89.4772727272727	1.46961396963402\\
89.5	1.45793725484282\\
89.5227272727273	1.44615351312329\\
89.5454545454545	1.43426360951793\\
89.5681818181818	1.42226841686255\\
89.5909090909091	1.41016881572221\\
89.6136363636364	1.39796569432661\\
89.6363636363636	1.38565994850485\\
89.6590909090909	1.37325248161962\\
89.6818181818182	1.36074420450101\\
89.7045454545455	1.34813603537952\\
89.7272727272727	1.33542889981872\\
89.75	1.32262373064729\\
89.7727272727273	1.30972146789056\\
89.7954545454545	1.29672305870146\\
89.8181818181818	1.283629457291\\
89.8409090909091	1.27044162485827\\
89.8636363636364	1.25716052951983\\
89.8863636363636	1.24378714623865\\
89.9090909090909	1.23032245675254\\
89.9318181818182	1.21676744950207\\
89.9545454545455	1.20312311955808\\
89.9772727272727	1.1893904685485\\
90	1.17557050458495\\
90.0227272727273	1.16166424218863\\
90.0454545454545	1.14767270221595\\
90.0681818181818	1.1335969117835\\
90.0909090909091	1.11943790419265\\
90.1136363636364	1.10519671885381\\
90.1363636363636	1.09087440120997\\
90.1590909090909	1.07647200266006\\
90.1818181818182	1.06199058048172\\
90.2045454545455	1.04743119775373\\
90.2272727272727	1.03279492327791\\
90.25	1.01808283150074\\
90.2727272727273	1.00329600243435\\
90.2954545454545	0.988435521577414\\
90.3181818181818	0.973502479835309\\
90.3409090909091	0.958497973440115\\
90.3636363636364	0.943423103870107\\
90.3863636363636	0.928278977768947\\
90.4090909090909	0.913066706864368\\
90.4318181818182	0.897787407886604\\
90.4545454545455	0.882442202486443\\
90.4772727272727	0.867032217152805\\
90.5	0.851558583130143\\
90.5227272727273	0.836022436335312\\
90.5454545454546	0.820424917274219\\
90.5681818181818	0.804767170958136\\
90.5909090909091	0.789050346819571\\
90.6136363636364	0.773275598627924\\
90.6363636363636	0.757444084404829\\
90.6590909090909	0.741556966339065\\
90.6818181818182	0.725615410701295\\
90.7045454545455	0.709620587758419\\
90.7272727272727	0.693573671687709\\
90.75	0.677475840490572\\
90.7727272727273	0.661328275906096\\
90.7954545454546	0.645132163324254\\
90.8181818181818	0.628888691698967\\
90.8409090909091	0.61259905346074\\
90.8636363636364	0.596264444429171\\
90.8863636363636	0.579886063725145\\
90.9090909090909	0.563465113682858\\
};
\addplot [color=mycolor1, dashed, line width=2.0pt]
  table[row sep=crcr]{%
90.9090909090909	0.563465113682858\\
90.9318181818182	0.547002799761474\\
90.9545454545455	0.530500330456672\\
90.9772727272727	0.513958917211956\\
91	0.497379774329703\\
91.0227272727273	0.480764118881983\\
91.0454545454546	0.464113170621253\\
91.0681818181818	0.447428151890846\\
91.0909090909091	0.430710287535165\\
91.1136363636364	0.413960804809815\\
91.1363636363636	0.397180933291489\\
91.1590909090909	0.380371904787756\\
91.1818181818182	0.363534953246555\\
91.2045454545455	0.346671314665649\\
91.2272727272727	0.329782227001918\\
91.25	0.312868930080459\\
91.2727272727273	0.295932665503536\\
91.2954545454546	0.278974676559478\\
91.3181818181818	0.261996208131385\\
91.3409090909091	0.244998506605779\\
91.3636363636364	0.227982819781078\\
91.3863636363636	0.21095039677596\\
91.4090909090909	0.193902487937757\\
91.4318181818182	0.176840344750571\\
91.4545454545455	0.15976521974346\\
91.4772727272727	0.142678366398461\\
91.5	0.125581039058615\\
91.5227272727273	0.108474492835855\\
91.5454545454546	0.0913599835188457\\
91.5681818181818	0.0742387674808187\\
91.5909090909091	0.0571121015873884\\
91.6136363636364	0.0399812431041993\\
91.6363636363636	0.0228474496046875\\
91.6590909090909	0.00571197887773397\\
91.6818181818182	-0.0114239111646128\\
91.7045454545455	-0.0285589625795649\\
91.7272727272727	-0.0456919174858982\\
91.75	-0.0628215181562642\\
91.7727272727273	-0.0799465071095485\\
91.7954545454546	-0.0970656272032103\\
91.8181818181818	-0.114177621725527\\
91.8409090909091	-0.131281234487876\\
91.8636363636364	-0.148375209916912\\
91.8863636363636	-0.165458293146763\\
91.9090909090909	-0.182529230111195\\
91.9318181818182	-0.199586767635571\\
91.9545454545455	-0.216629653528949\\
91.9772727272727	-0.23365663667596\\
92	-0.250666467128611\\
92.0227272727273	-0.267657896198142\\
92.0454545454546	-0.284629676546576\\
92.0681818181818	-0.301580562278393\\
92.0909090909091	-0.318509309031938\\
92.1136363636364	-0.335414674070736\\
92.1363636363636	-0.352295416374801\\
92.1590909090909	-0.369150296731697\\
92.1818181818182	-0.385978077827468\\
92.2045454545455	-0.402777524337551\\
92.2272727272727	-0.419547403017408\\
92.25	-0.436286482793084\\
92.2727272727273	-0.452993534851536\\
92.2954545454546	-0.469667332730873\\
92.3181818181818	-0.486306652410415\\
92.3409090909091	-0.502910272400521\\
92.3636363636364	-0.519476973832216\\
92.3863636363636	-0.536005540546751\\
92.4090909090909	-0.552494759184829\\
92.4318181818182	-0.568943419275698\\
92.4545454545455	-0.58535031332598\\
92.4772727272727	-0.60171423690833\\
92.5	-0.618033988749902\\
92.5227272727273	-0.634308370820449\\
92.5454545454546	-0.65053618842031\\
92.5681818181818	-0.666716250268141\\
92.5909090909091	-0.682847368588328\\
92.6136363636364	-0.698928359198205\\
92.6363636363636	-0.714958041594931\\
92.6590909090909	-0.730935239042237\\
92.6818181818182	-0.746858778656761\\
92.7045454545455	-0.762727491494116\\
92.7272727272727	-0.778540212634778\\
92.75	-0.794295781269551\\
92.7727272727273	-0.809993040784809\\
92.7954545454546	-0.82563083884736\\
92.8181818181818	-0.841208027489071\\
92.8409090909091	-0.856723463191158\\
92.8636363636364	-0.872176006968114\\
92.8863636363636	-0.887564524451277\\
92.9090909090909	-0.902887885972182\\
92.9318181818182	-0.918144966645441\\
92.9545454545455	-0.933334646451344\\
92.9772727272727	-0.948455810318042\\
93	-0.963507348203429\\
93.0227272727273	-0.978488155176671\\
93.0454545454546	-0.993397131499236\\
93.0681818181818	-1.00823318270567\\
93.0909090909091	-1.02299521968397\\
93.1136363636364	-1.0376821587555\\
93.1363636363636	-1.0522929217545\\
93.1590909090909	-1.06682643610735\\
93.1818181818182	-1.08128163491119\\
93.2045454545455	-1.09565745701234\\
93.2272727272727	-1.10995284708408\\
93.25	-1.12416675570426\\
93.2727272727273	-1.13829813943224\\
93.2954545454546	-1.15234596088552\\
93.3181818181818	-1.16630918881587\\
93.3409090909091	-1.1801867981851\\
93.3636363636364	-1.19397777024024\\
93.3863636363636	-1.20768109258837\\
93.4090909090909	-1.22129575927087\\
93.4318181818182	-1.23482077083738\\
93.4545454545455	-1.24825513441908\\
93.4772727272727	-1.26159786380158\\
93.5	-1.27484797949737\\
93.5227272727273	-1.28800450881771\\
93.5454545454546	-1.30106648594397\\
93.5681818181818	-1.31403295199863\\
93.5909090909091	-1.32690295511559\\
93.6136363636364	-1.33967555051008\\
93.6363636363636	-1.35234980054805\\
93.6590909090909	-1.36492477481491\\
93.6818181818182	-1.37739955018392\\
93.7045454545455	-1.38977321088394\\
93.7272727272727	-1.40204484856663\\
93.75	-1.41421356237309\\
93.7727272727273	-1.42627845900014\\
93.7954545454546	-1.43823865276571\\
93.8181818181818	-1.45009326567402\\
93.8409090909091	-1.46184142747991\\
93.8636363636364	-1.47348227575281\\
93.8863636363636	-1.48501495594001\\
93.9090909090909	-1.4964386214294\\
93.9318181818182	-1.50775243361162\\
93.9545454545455	-1.51895556194162\\
93.9772727272727	-1.53004718399965\\
94	-1.54102648555157\\
94.0227272727273	-1.55189266060875\\
94.0454545454546	-1.56264491148706\\
94.0681818181818	-1.57328244886561\\
94.0909090909091	-1.58380449184455\\
94.1136363636364	-1.5942102680025\\
94.1363636363636	-1.60449901345315\\
94.1590909090909	-1.61466997290147\\
94.1818181818182	-1.62472239969901\\
94.2045454545455	-1.63465555589881\\
94.2272727272727	-1.64446871230958\\
94.25	-1.65416114854913\\
94.2727272727273	-1.66373215309736\\
94.2954545454546	-1.67318102334842\\
94.3181818181818	-1.68250706566236\\
94.3409090909091	-1.69170959541599\\
94.3636363636364	-1.70078793705312\\
94.3863636363636	-1.70974142413423\\
94.4090909090909	-1.71856939938536\\
94.4318181818182	-1.72727121474632\\
94.4545454545455	-1.7358462314183\\
94.4772727272727	-1.74429381991078\\
94.5	-1.75261336008773\\
94.5227272727273	-1.76080424121307\\
94.5454545454546	-1.76886586199563\\
94.5681818181818	-1.77679763063316\\
94.5909090909091	-1.78459896485588\\
94.6136363636364	-1.79226929196914\\
94.6363636363636	-1.79980804889554\\
94.6590909090909	-1.80721468221618\\
94.6818181818182	-1.81448864821138\\
94.7045454545455	-1.8216294129005\\
94.7272727272727	-1.82863645208122\\
94.75	-1.83550925136797\\
94.7727272727273	-1.8422473062297\\
94.7954545454546	-1.84885012202696\\
94.8181818181818	-1.85531721404814\\
94.8409090909091	-1.8616481075451\\
94.8636363636364	-1.86784233776803\\
94.8863636363636	-1.87389944999952\\
94.9090909090909	-1.87981899958799\\
94.9318181818182	-1.8856005519803\\
94.9545454545455	-1.89124368275365\\
94.9772727272727	-1.89674797764677\\
95	-1.9021130325903\\
95.0227272727273	-1.90733845373647\\
95.0454545454545	-1.91242385748797\\
95.0681818181818	-1.91736887052619\\
95.0909090909091	-1.92217312983854\\
95.1136363636364	-1.92683628274516\\
95.1363636363636	-1.93135798692478\\
95.1590909090909	-1.93573791043985\\
95.1818181818182	-1.93997573176093\\
95.2045454545455	-1.94407113979028\\
95.2272727272727	-1.94802383388467\\
95.25	-1.95183352387749\\
95.2727272727273	-1.95549993010006\\
95.2954545454545	-1.9590227834021\\
95.3181818181818	-1.96240182517155\\
95.3409090909091	-1.96563680735351\\
95.3636363636364	-1.9687274924685\\
95.3863636363636	-1.97167365362984\\
95.4090909090909	-1.97447507456033\\
95.4318181818182	-1.97713154960813\\
95.4545454545455	-1.97964288376187\\
95.4772727272727	-1.98200889266491\\
95.5	-1.98422940262896\\
95.5227272727273	-1.98630425064673\\
95.5454545454545	-1.988233284404\\
95.5681818181818	-1.99001636229072\\
95.5909090909091	-1.99165335341144\\
95.6136363636364	-1.99314413759491\\
95.6363636363636	-1.99448860540293\\
95.6590909090909	-1.99568665813833\\
95.6818181818182	-1.99673820785227\\
95.7045454545455	-1.99764317735066\\
95.7272727272727	-1.99840150019984\\
95.75	-1.99901312073146\\
95.7727272727273	-1.99947799404656\\
95.7954545454545	-1.99979608601887\\
95.8181818181818	-1.9999673732973\\
95.8409090909091	-1.99999184330769\\
95.8636363636364	-1.9998694942537\\
95.8863636363636	-1.99960033511696\\
95.9090909090909	-1.99918438565638\\
95.9318181818182	-1.99862167640675\\
95.9545454545455	-1.99791224867646\\
95.9772727272727	-1.99705615454448\\
96	-1.99605345685654\\
96.0227272727273	-1.99490422922051\\
96.0454545454545	-1.99360855600098\\
96.0681818181818	-1.99216653231312\\
96.0909090909091	-1.99057826401565\\
96.1136363636364	-1.98884386770306\\
96.1363636363636	-1.9869634706971\\
96.1590909090909	-1.9849372110374\\
96.1818181818182	-1.98276523747134\\
96.2045454545455	-1.98044770944313\\
96.2272727272727	-1.97798479708211\\
96.25	-1.97537668119028\\
96.2727272727273	-1.97262355322896\\
96.2954545454545	-1.96972561530483\\
96.3181818181818	-1.96668308015502\\
96.3409090909091	-1.96349617113152\\
96.3636363636364	-1.96016512218478\\
96.3863636363636	-1.95669017784656\\
96.4090909090909	-1.95307159321192\\
96.4318181818182	-1.94930963392057\\
96.4545454545455	-1.94540457613728\\
96.4772727272727	-1.9413567065317\\
96.5	-1.93716632225726\\
96.5227272727273	-1.93283373092936\\
96.5454545454545	-1.92835925060279\\
96.5681818181818	-1.92374320974841\\
96.5909090909091	-1.91898594722899\\
96.6136363636364	-1.91408781227439\\
96.6363636363636	-1.90904916445585\\
96.6590909090909	-1.90387037365966\\
96.6818181818182	-1.89855182006\\
96.7045454545455	-1.89309389409098\\
96.7272727272727	-1.88749699641803\\
96.75	-1.88176153790845\\
96.7727272727273	-1.87588793960131\\
96.7954545454545	-1.86987663267645\\
96.8181818181818	-1.86372805842291\\
96.8409090909091	-1.85744266820648\\
96.8636363636364	-1.85102092343661\\
96.8863636363636	-1.8444632955325\\
96.9090909090909	-1.83777026588851\\
96.9318181818182	-1.83094232583881\\
96.9545454545455	-1.82397997662134\\
96.9772727272727	-1.81688372934098\\
97	-1.80965410493203\\
97.0227272727273	-1.80229163412004\\
97.0454545454545	-1.79479685738272\\
97.0681818181818	-1.78717032491038\\
97.0909090909091	-1.77941259656546\\
97.1136363636364	-1.77152424184153\\
97.1363636363636	-1.76350583982137\\
97.1590909090909	-1.75535797913451\\
97.1818181818182	-1.74708125791403\\
97.2045454545455	-1.73867628375266\\
97.2272727272727	-1.73014367365812\\
97.25	-1.72148405400789\\
97.2727272727273	-1.71269806050318\\
97.2954545454545	-1.7037863381223\\
97.3181818181818	-1.69474954107329\\
97.3409090909091	-1.68558833274589\\
97.3636363636364	-1.67630338566287\\
97.3863636363636	-1.66689538143064\\
97.4090909090909	-1.65736501068919\\
97.4318181818182	-1.64771297306143\\
97.4545454545455	-1.63793997710183\\
97.4772727272727	-1.62804674024436\\
97.5	-1.61803398874989\\
97.5227272727273	-1.60790245765281\\
97.5454545454545	-1.59765289070713\\
97.5681818181818	-1.5872860403318\\
97.5909090909091	-1.57680266755558\\
97.6136363636364	-1.5662035419611\\
97.6363636363636	-1.55548944162839\\
97.6590909090909	-1.54466115307774\\
97.6818181818182	-1.533719471212\\
97.7045454545455	-1.52266519925817\\
97.7272727272727	-1.51149914870852\\
97.75	-1.50022213926092\\
97.7727272727273	-1.48883499875874\\
97.7954545454545	-1.47733856313006\\
97.8181818181818	-1.46573367632627\\
97.8409090909091	-1.45402119026016\\
97.8636363636364	-1.44220196474334\\
97.8863636363636	-1.43027686742319\\
97.9090909090909	-1.41824677371909\\
97.9318181818182	-1.4061125667582\\
97.9545454545455	-1.39387513731059\\
97.9772727272727	-1.38153538372391\\
98	-1.36909421185738\\
98.0227272727273	-1.35655253501535\\
98.0454545454545	-1.34391127388018\\
98.0681818181818	-1.33117135644473\\
98.0909090909091	-1.31833371794422\\
98.1136363636364	-1.30539930078749\\
98.1363636363636	-1.29236905448794\\
98.1590909090909	-1.27924393559377\\
98.1818181818182	-1.26602490761774\\
98.2045454545455	-1.25271294096645\\
98.2272727272727	-1.23930901286916\\
98.25	-1.22581410730595\\
98.2727272727273	-1.21222921493558\\
98.2954545454545	-1.19855533302269\\
98.3181818181818	-1.18479346536467\\
98.3409090909091	-1.17094462221791\\
98.3636363636364	-1.15700982022362\\
98.3863636363636	-1.14299008233329\\
98.4090909090909	-1.12888643773353\\
98.4318181818182	-1.11469992177051\\
98.4545454545455	-1.10043157587396\\
98.4772727272727	-1.08608244748075\\
98.5	-1.071653589958\\
98.5227272727273	-1.05714606252569\\
98.5454545454545	-1.04256093017897\\
98.5681818181818	-1.02789926360996\\
98.5909090909091	-1.01316213912911\\
98.6136363636364	-0.998350638586271\\
98.6363636363636	-0.983465849291203\\
98.6590909090909	-0.968508863933759\\
98.6818181818182	-0.953480780503742\\
98.7045454545455	-0.938382702210212\\
98.7272727272727	-0.923215737400542\\
98.75	-0.907980999479088\\
98.7727272727273	-0.892679606825384\\
98.7954545454545	-0.877312682712087\\
98.8181818181818	-0.861881355222495\\
98.8409090909091	-0.846386757167785\\
98.8636363636364	-0.830830026003767\\
98.8863636363636	-0.81521230374748\\
98.9090909090909	-0.799534736893259\\
98.9318181818182	-0.783798476328672\\
98.9545454545455	-0.768004677249944\\
98.9772727272727	-0.752154499077206\\
99	-0.736249105369355\\
99.0227272727273	-0.720289663738689\\
99.0454545454545	-0.704277345765119\\
99.0681818181818	-0.688213326910196\\
99.0909090909091	-0.672098786430867\\
99.1136363636364	-0.655934907292806\\
99.1363636363637	-0.639722876083685\\
99.1590909090909	-0.62346388292596\\
99.1818181818182	-0.607159121389605\\
99.2045454545455	-0.590809788404393\\
99.2272727272727	-0.57441708417213\\
99.25	-0.557982212078448\\
99.2727272727273	-0.541506378604565\\
99.2954545454545	-0.524990793238629\\
99.3181818181818	-0.50843666838698\\
99.3409090909091	-0.49184521928512\\
99.3636363636364	-0.475217663908549\\
99.3863636363637	-0.45855522288332\\
99.4090909090909	-0.441859119396408\\
99.4318181818182	-0.425130579105945\\
99.4545454545455	-0.408370830051283\\
99.4772727272727	-0.391581102562765\\
99.5	-0.374762629171443\\
99.5227272727273	-0.357916644518645\\
99.5454545454545	-0.341044385265252\\
99.5681818181818	-0.324147090000976\\
99.5909090909091	-0.307225999153401\\
99.6136363636364	-0.290282354896985\\
99.6363636363637	-0.27331740106177\\
99.6590909090909	-0.25633238304219\\
99.6818181818182	-0.239328547705527\\
99.7045454545455	-0.222307143300499\\
99.7272727272727	-0.205269419365508\\
99.75	-0.188216626637027\\
99.7727272727273	-0.171150016957669\\
99.7954545454545	-0.154070843184406\\
99.8181818181818	-0.136980359096492\\
99.8409090909091	-0.119879819303474\\
99.8636363636364	-0.102770479153124\\
99.8863636363637	-0.0856535946391941\\
99.9090909090909	-0.0685304223093216\\
99.9318181818182	-0.0514022191726891\\
99.9545454545455	-0.0342702426077901\\
99.9772727272727	-0.0171357502701672\\
100	-2.93915231795365e-15\\
};
\addlegendentry{$f$ = 60 Hz}

\addplot [color=mycolor2, line width=2.0pt, forget plot]
  table[row sep=crcr]{%
0	1.5\\
0.0227272727272727	1.49990212069028\\
0.0454545454545454	1.49960849553492\\
0.0681818181818182	1.4991191628537\\
0.0909090909090909	1.49843418650735\\
0.113636363636364	1.4975536558892\\
0.136363636363636	1.49647768591358\\
0.159090909090909	1.49520641700074\\
0.181818181818182	1.49374001505858\\
0.204545454545455	1.49207867146098\\
0.227272727272727	1.49022260302283\\
0.25	1.48817205197172\\
0.272727272727273	1.48592728591635\\
0.295454545454545	1.48348859781159\\
0.318181818181818	1.48085630592024\\
0.340909090909091	1.47803075377154\\
0.363636363636364	1.47501231011626\\
0.386363636363636	1.47180136887866\\
0.409090909090909	1.46839834910501\\
0.431818181818182	1.46480369490895\\
0.454545454545455	1.4610178754135\\
0.477272727272727	1.45704138468987\\
0.5	1.45287474169295\\
0.522727272727273	1.44851849019358\\
0.545454545454545	1.44397319870763\\
0.568181818181818	1.43923946042175\\
0.590909090909091	1.43431789311598\\
0.613636363636364	1.42920913908315\\
0.636363636363636	1.423913865045\\
0.659090909090909	1.41843276206524\\
0.681818181818182	1.41276654545929\\
0.704545454545455	1.40691595470098\\
0.727272727272727	1.40088175332602\\
0.75	1.39466472883238\\
0.772727272727273	1.38826569257746\\
0.795454545454545	1.38168547967228\\
0.818181818181818	1.37492494887242\\
0.840909090909091	1.36798498246601\\
0.863636363636364	1.36086648615853\\
0.886363636363636	1.35357038895467\\
0.909090909090909	1.34609764303704\\
0.931818181818182	1.33844922364191\\
0.954545454545454	1.33062612893197\\
0.977272727272727	1.32262937986603\\
1	1.3144600200658\\
1.02272727272727	1.30611911567967\\
1.04545454545455	1.29760775524359\\
1.06818181818182	1.28892704953902\\
1.09090909090909	1.28007813144792\\
1.11363636363636	1.27106215580496\\
1.13636363636364	1.26188029924677\\
1.15909090909091	1.25253376005839\\
1.18181818181818	1.24302375801689\\
1.20454545454545	1.23335153423218\\
1.22727272727273	1.22351835098505\\
1.25	1.21352549156242\\
1.27272727272727	1.20337426008986\\
1.29545454545455	1.19306598136142\\
1.31818181818182	1.18260200066669\\
1.34090909090909	1.1719836836153\\
1.36363636363636	1.16121241595862\\
1.38636363636364	1.150289603409\\
1.40909090909091	1.13921667145622\\
1.43181818181818	1.12799506518154\\
1.45454545454545	1.11662624906906\\
1.47727272727273	1.10511170681461\\
1.5	1.09345294113212\\
1.52272727272727	1.0816514735575\\
1.54545454545455	1.0697088442501\\
1.56818181818182	1.05762661179166\\
1.59090909090909	1.04540635298294\\
1.61363636363636	1.03304966263794\\
1.63636363636364	1.02055815337576\\
1.65909090909091	1.00793345541013\\
1.68181818181818	0.995177216336688\\
1.70454545454545	0.982291100917928\\
1.72727272727273	0.969276790865962\\
1.75	0.956135984623035\\
1.77272727272727	0.942870397139873\\
1.79545454545455	0.929481759651872\\
1.81818181818182	0.915971819453157\\
1.84090909090909	0.902342339668558\\
1.86363636363636	0.888595099023505\\
1.88636363636364	0.874731891611898\\
1.90909090909091	0.860754526661967\\
1.93181818181818	0.846664828300154\\
1.95454545454545	0.832464635313061\\
1.97727272727273	0.81815580090747\\
2	0.803740192468495\\
2.02272727272727	0.789219691315873\\
2.04545454545454	0.774596192458444\\
2.06818181818182	0.759871604346838\\
2.09090909090909	0.745047848624417\\
2.11363636363636	0.730126859876485\\
2.13636363636364	0.715110585377813\\
2.15909090909091	0.700000984838511\\
2.18181818181818	0.684800030148274\\
2.20454545454545	0.669509705119039\\
2.22727272727273	0.654132005226083\\
2.25	0.638668937347609\\
2.27272727272727	0.623122519502829\\
2.29545454545455	0.607494780588609\\
2.31818181818182	0.591787760114674\\
2.34090909090909	0.576003507937458\\
2.36363636363636	0.560144083992567\\
2.38636363636364	0.544211558025962\\
2.40909090909091	0.528208009323832\\
2.43181818181818	0.512135526441247\\
2.45454545454545	0.495996206929577\\
2.47727272727273	0.479792157062762\\
2.5	0.463525491562421\\
2.52272727272727	0.447198333321878\\
2.54545454545454	0.4308128131291\\
2.56818181818182	0.414371069388631\\
2.59090909090909	0.397875247842507\\
2.61363636363636	0.381327501290231\\
2.63636363636364	0.364729989307815\\
2.65909090909091	0.348084877965947\\
2.68181818181818	0.331394339547307\\
2.70454545454545	0.314660552263064\\
2.72727272727273	0.297885699968619\\
2.75	0.281071971878588\\
2.77272727272727	0.264221562281103\\
2.79545454545455	0.247336670251445\\
2.81818181818182	0.230419499365049\\
2.84090909090909	0.213472257409928\\
2.86363636363636	0.196497156098539\\
2.88636363636364	0.179496410779149\\
2.90909090909091	0.162472240146713\\
2.93181818181818	0.145426865953322\\
2.95454545454545	0.12836251271826\\
2.97727272727273	0.111281407437678\\
3	0.094185779293971\\
3.02272727272727	0.077077859364845\\
3.04545454545455	0.0599598803321602\\
3.06818181818182	0.0428340761905448\\
3.09090909090909	0.0257026819558485\\
3.11363636363636	0.00856793337346312\\
3.13636363636364	-0.00856793337346275\\
3.15909090909091	-0.0257026819558482\\
3.18181818181818	-0.0428340761905444\\
3.20454545454545	-0.0599598803321592\\
3.22727272727273	-0.077077859364844\\
3.25	-0.09418577929397\\
3.27272727272727	-0.111281407437678\\
3.29545454545455	-0.128362512718259\\
3.31818181818182	-0.145426865953322\\
3.34090909090909	-0.162472240146712\\
3.36363636363636	-0.179496410779148\\
3.38636363636364	-0.196497156098538\\
3.40909090909091	-0.213472257409927\\
3.43181818181818	-0.230419499365049\\
3.45454545454545	-0.247336670251444\\
3.47727272727273	-0.264221562281102\\
3.5	-0.281071971878586\\
3.52272727272727	-0.297885699968618\\
3.54545454545455	-0.314660552263063\\
3.56818181818182	-0.331394339547306\\
3.59090909090909	-0.348084877965947\\
3.61363636363636	-0.364729989307815\\
3.63636363636364	-0.38132750129023\\
3.65909090909091	-0.397875247842507\\
3.68181818181818	-0.41437106938863\\
3.70454545454545	-0.430812813129099\\
3.72727272727273	-0.447198333321877\\
3.75	-0.463525491562421\\
3.77272727272727	-0.479792157062761\\
3.79545454545455	-0.495996206929577\\
3.81818181818182	-0.512135526441246\\
3.84090909090909	-0.528208009323832\\
3.86363636363636	-0.544211558025961\\
3.88636363636364	-0.560144083992567\\
3.90909090909091	-0.576003507937457\\
3.93181818181818	-0.591787760114674\\
3.95454545454545	-0.607494780588608\\
3.97727272727273	-0.623122519502829\\
4	-0.638668937347609\\
4.02272727272727	-0.654132005226082\\
4.04545454545454	-0.669509705119038\\
4.06818181818182	-0.684800030148273\\
4.09090909090909	-0.70000098483851\\
4.11363636363636	-0.715110585377812\\
4.13636363636364	-0.730126859876484\\
4.15909090909091	-0.745047848624417\\
4.18181818181818	-0.759871604346837\\
4.20454545454545	-0.774596192458442\\
4.22727272727273	-0.789219691315872\\
4.25	-0.803740192468495\\
4.27272727272727	-0.81815580090747\\
4.29545454545454	-0.83246463531306\\
4.31818181818182	-0.846664828300154\\
4.34090909090909	-0.860754526661966\\
4.36363636363636	-0.874731891611897\\
4.38636363636364	-0.888595099023504\\
4.40909090909091	-0.902342339668557\\
4.43181818181818	-0.915971819453157\\
4.45454545454545	-0.929481759651871\\
4.47727272727273	-0.942870397139873\\
4.5	-0.956135984623034\\
4.52272727272727	-0.969276790865961\\
4.54545454545454	-0.982291100917927\\
4.56818181818182	-0.995177216336687\\
4.59090909090909	-1.00793345541013\\
4.61363636363636	-1.02055815337576\\
4.63636363636364	-1.03304966263794\\
4.65909090909091	-1.04540635298294\\
4.68181818181818	-1.05762661179166\\
4.70454545454545	-1.0697088442501\\
4.72727272727273	-1.0816514735575\\
4.75	-1.09345294113212\\
4.77272727272727	-1.10511170681461\\
4.79545454545455	-1.11662624906906\\
4.81818181818182	-1.12799506518154\\
4.84090909090909	-1.13921667145622\\
4.86363636363636	-1.150289603409\\
4.88636363636364	-1.16121241595862\\
4.90909090909091	-1.1719836836153\\
4.93181818181818	-1.18260200066669\\
4.95454545454545	-1.19306598136142\\
4.97727272727273	-1.20337426008986\\
5	-1.21352549156242\\
5.02272727272727	-1.22351835098505\\
5.04545454545454	-1.23335153423218\\
5.06818181818182	-1.24302375801689\\
5.09090909090909	-1.25253376005839\\
5.11363636363636	-1.26188029924677\\
5.13636363636364	-1.27106215580496\\
5.15909090909091	-1.28007813144792\\
5.18181818181818	-1.28892704953902\\
5.20454545454545	-1.29760775524359\\
5.22727272727273	-1.30611911567967\\
5.25	-1.31446002006579\\
5.27272727272727	-1.32262937986603\\
5.29545454545454	-1.33062612893197\\
5.31818181818182	-1.33844922364191\\
5.34090909090909	-1.34609764303704\\
5.36363636363636	-1.35357038895467\\
5.38636363636364	-1.36086648615853\\
5.40909090909091	-1.36798498246601\\
5.43181818181818	-1.37492494887242\\
5.45454545454545	-1.38168547967227\\
5.47727272727273	-1.38826569257746\\
5.5	-1.39466472883238\\
5.52272727272727	-1.40088175332602\\
5.54545454545454	-1.40691595470098\\
5.56818181818182	-1.41276654545929\\
5.59090909090909	-1.41843276206524\\
5.61363636363636	-1.423913865045\\
5.63636363636364	-1.42920913908315\\
5.65909090909091	-1.43431789311598\\
5.68181818181818	-1.43923946042175\\
5.70454545454545	-1.44397319870763\\
5.72727272727273	-1.44851849019358\\
5.75	-1.45287474169295\\
5.77272727272727	-1.45704138468987\\
5.79545454545455	-1.4610178754135\\
5.81818181818182	-1.46480369490895\\
5.84090909090909	-1.46839834910501\\
5.86363636363636	-1.47180136887866\\
5.88636363636364	-1.47501231011626\\
5.90909090909091	-1.47803075377154\\
5.93181818181818	-1.48085630592024\\
5.95454545454545	-1.48348859781159\\
5.97727272727273	-1.48592728591635\\
6	-1.48817205197172\\
6.02272727272727	-1.49022260302283\\
6.04545454545454	-1.49207867146098\\
6.06818181818182	-1.49374001505858\\
6.09090909090909	-1.49520641700074\\
6.11363636363636	-1.49647768591358\\
6.13636363636364	-1.4975536558892\\
6.15909090909091	-1.49843418650735\\
6.18181818181818	-1.4991191628537\\
6.20454545454545	-1.49960849553492\\
6.22727272727273	-1.49990212069028\\
6.25	-1.5\\
6.27272727272727	-1.49990212069028\\
6.29545454545454	-1.49960849553492\\
6.31818181818182	-1.4991191628537\\
6.34090909090909	-1.49843418650735\\
6.36363636363636	-1.4975536558892\\
6.38636363636364	-1.49647768591358\\
6.40909090909091	-1.49520641700074\\
6.43181818181818	-1.49374001505858\\
6.45454545454545	-1.49207867146098\\
6.47727272727273	-1.49022260302283\\
6.5	-1.48817205197172\\
6.52272727272727	-1.48592728591635\\
6.54545454545454	-1.48348859781159\\
6.56818181818182	-1.48085630592024\\
6.59090909090909	-1.47803075377154\\
6.61363636363636	-1.47501231011626\\
6.63636363636364	-1.47180136887866\\
6.65909090909091	-1.46839834910501\\
6.68181818181818	-1.46480369490895\\
6.70454545454545	-1.4610178754135\\
6.72727272727273	-1.45704138468987\\
6.75	-1.45287474169295\\
6.77272727272727	-1.44851849019358\\
6.79545454545454	-1.44397319870763\\
6.81818181818182	-1.43923946042175\\
6.84090909090909	-1.43431789311598\\
6.86363636363636	-1.42920913908315\\
6.88636363636364	-1.423913865045\\
6.90909090909091	-1.41843276206524\\
6.93181818181818	-1.41276654545929\\
6.95454545454545	-1.40691595470098\\
6.97727272727273	-1.40088175332602\\
7	-1.39466472883238\\
7.02272727272727	-1.38826569257746\\
7.04545454545454	-1.38168547967228\\
7.06818181818182	-1.37492494887242\\
7.09090909090909	-1.36798498246601\\
7.11363636363636	-1.36086648615853\\
7.13636363636364	-1.35357038895467\\
7.15909090909091	-1.34609764303704\\
7.18181818181818	-1.33844922364191\\
7.20454545454545	-1.33062612893197\\
7.22727272727273	-1.32262937986603\\
7.25	-1.3144600200658\\
7.27272727272727	-1.30611911567967\\
7.29545454545454	-1.29760775524359\\
7.31818181818182	-1.28892704953902\\
7.34090909090909	-1.28007813144792\\
7.36363636363636	-1.27106215580496\\
7.38636363636364	-1.26188029924677\\
7.40909090909091	-1.25253376005839\\
7.43181818181818	-1.24302375801689\\
7.45454545454545	-1.23335153423218\\
7.47727272727273	-1.22351835098505\\
7.5	-1.21352549156242\\
7.52272727272727	-1.20337426008986\\
7.54545454545454	-1.19306598136142\\
7.56818181818182	-1.18260200066669\\
7.59090909090909	-1.1719836836153\\
7.61363636363636	-1.16121241595862\\
7.63636363636364	-1.150289603409\\
7.65909090909091	-1.13921667145622\\
7.68181818181818	-1.12799506518154\\
7.70454545454545	-1.11662624906906\\
7.72727272727273	-1.10511170681461\\
7.75	-1.09345294113212\\
7.77272727272727	-1.0816514735575\\
7.79545454545454	-1.0697088442501\\
7.81818181818182	-1.05762661179166\\
7.84090909090909	-1.04540635298294\\
7.86363636363636	-1.03304966263794\\
7.88636363636364	-1.02055815337576\\
7.90909090909091	-1.00793345541013\\
7.93181818181818	-0.995177216336689\\
7.95454545454545	-0.982291100917928\\
7.97727272727273	-0.969276790865962\\
8	-0.956135984623034\\
8.02272727272727	-0.942870397139875\\
8.04545454545454	-0.929481759651873\\
8.06818181818182	-0.915971819453158\\
8.09090909090909	-0.902342339668559\\
8.11363636363636	-0.888595099023505\\
8.13636363636364	-0.874731891611899\\
8.15909090909091	-0.860754526661967\\
8.18181818181818	-0.846664828300155\\
8.20454545454545	-0.832464635313061\\
8.22727272727273	-0.81815580090747\\
8.25	-0.803740192468497\\
8.27272727272727	-0.789219691315873\\
8.29545454545454	-0.774596192458443\\
8.31818181818182	-0.759871604346838\\
8.34090909090909	-0.745047848624418\\
8.36363636363636	-0.730126859876485\\
8.38636363636364	-0.715110585377813\\
8.40909090909091	-0.700000984838512\\
8.43181818181818	-0.684800030148275\\
8.45454545454546	-0.669509705119039\\
8.47727272727273	-0.654132005226083\\
8.5	-0.63866893734761\\
8.52272727272727	-0.623122519502831\\
8.54545454545454	-0.607494780588609\\
8.56818181818182	-0.591787760114675\\
8.59090909090909	-0.576003507937459\\
8.61363636363636	-0.560144083992567\\
8.63636363636364	-0.544211558025961\\
8.65909090909091	-0.528208009323834\\
8.68181818181818	-0.512135526441247\\
8.70454545454546	-0.495996206929577\\
8.72727272727273	-0.479792157062762\\
8.75	-0.463525491562421\\
8.77272727272727	-0.447198333321878\\
8.79545454545454	-0.4308128131291\\
8.81818181818182	-0.414371069388631\\
8.84090909090909	-0.397875247842507\\
8.86363636363636	-0.38132750129023\\
8.88636363636364	-0.364729989307815\\
8.90909090909091	-0.348084877965948\\
8.93181818181818	-0.331394339547307\\
8.95454545454546	-0.314660552263064\\
8.97727272727273	-0.29788569996862\\
9	-0.281071971878588\\
9.02272727272727	-0.264221562281103\\
9.04545454545454	-0.247336670251445\\
9.06818181818182	-0.23041949936505\\
9.09090909090909	-0.213472257409928\\
9.11363636363636	-0.19649715609854\\
9.13636363636363	-0.179496410779149\\
9.15909090909091	-0.162472240146713\\
9.18181818181818	-0.145426865953323\\
9.20454545454546	-0.128362512718259\\
9.22727272727273	-0.11128140743768\\
9.25	-0.0941857792939699\\
9.27272727272727	-0.0770778593648452\\
9.29545454545454	-0.0599598803321611\\
9.31818181818182	-0.042834076190545\\
9.34090909090909	-0.0257026819558501\\
9.36363636363636	-0.00856793337346331\\
9.38636363636363	0.00856793337346124\\
9.40909090909091	0.025702681955848\\
9.43181818181818	0.0428340761905443\\
9.45454545454545	0.059959880332159\\
9.47727272727273	0.0770778593648444\\
9.5	0.0941857792939692\\
9.52272727272727	0.111281407437678\\
9.54545454545454	0.128362512718259\\
9.56818181818182	0.145426865953322\\
9.59090909090909	0.162472240146712\\
9.61363636363636	0.179496410779147\\
9.63636363636363	0.196497156098538\\
9.65909090909091	0.213472257409927\\
9.68181818181818	0.230419499365048\\
9.70454545454545	0.247336670251444\\
9.72727272727273	0.264221562281101\\
9.75	0.281071971878586\\
9.77272727272727	0.297885699968617\\
9.79545454545454	0.314660552263063\\
9.81818181818182	0.331394339547306\\
9.84090909090909	0.348084877965948\\
9.86363636363636	0.364729989307813\\
9.88636363636364	0.38132750129023\\
9.90909090909091	0.397875247842507\\
9.93181818181818	0.414371069388629\\
9.95454545454545	0.430812813129099\\
9.97727272727273	0.447198333321877\\
10	0.463525491562421\\
10.0227272727273	0.47979215706276\\
10.0454545454545	0.495996206929576\\
10.0681818181818	0.512135526441245\\
10.0909090909091	0.528208009323831\\
10.1136363636364	0.54421155802596\\
10.1363636363636	0.560144083992565\\
10.1590909090909	0.576003507937456\\
10.1818181818182	0.591787760114674\\
10.2045454545455	0.607494780588608\\
10.2272727272727	0.62312251950283\\
10.25	0.638668937347608\\
10.2727272727273	0.654132005226082\\
10.2954545454545	0.669509705119038\\
10.3181818181818	0.684800030148273\\
10.3409090909091	0.700000984838509\\
10.3636363636364	0.715110585377811\\
10.3863636363636	0.730126859876484\\
10.4090909090909	0.745047848624416\\
10.4318181818182	0.759871604346837\\
10.4545454545455	0.774596192458442\\
10.4772727272727	0.789219691315872\\
10.5	0.803740192468494\\
10.5227272727273	0.818155800907469\\
10.5454545454545	0.83246463531306\\
10.5681818181818	0.846664828300153\\
10.5909090909091	0.860754526661966\\
10.6136363636364	0.874731891611898\\
10.6363636363636	0.888595099023505\\
10.6590909090909	0.902342339668557\\
10.6818181818182	0.915971819453157\\
10.7045454545455	0.929481759651871\\
10.7272727272727	0.942870397139872\\
10.75	0.956135984623034\\
10.7727272727273	0.96927679086596\\
10.7954545454545	0.982291100917927\\
10.8181818181818	0.995177216336687\\
10.8409090909091	1.00793345541013\\
10.8636363636364	1.02055815337576\\
10.8863636363636	1.03304966263794\\
10.9090909090909	1.04540635298294\\
10.9318181818182	1.05762661179166\\
10.9545454545455	1.0697088442501\\
10.9772727272727	1.0816514735575\\
11	1.09345294113212\\
11.0227272727273	1.10511170681461\\
11.0454545454545	1.11662624906906\\
11.0681818181818	1.12799506518154\\
11.0909090909091	1.13921667145622\\
11.1136363636364	1.150289603409\\
11.1363636363636	1.16121241595862\\
11.1590909090909	1.1719836836153\\
11.1818181818182	1.18260200066669\\
11.2045454545455	1.19306598136142\\
11.2272727272727	1.20337426008986\\
11.25	1.21352549156242\\
11.2727272727273	1.22351835098505\\
11.2954545454545	1.23335153423218\\
11.3181818181818	1.24302375801689\\
11.3409090909091	1.25253376005839\\
11.3636363636364	1.26188029924677\\
11.3863636363636	1.27106215580496\\
11.4090909090909	1.28007813144792\\
11.4318181818182	1.28892704953902\\
11.4545454545455	1.29760775524359\\
11.4772727272727	1.30611911567967\\
11.5	1.3144600200658\\
11.5227272727273	1.32262937986603\\
11.5454545454545	1.33062612893197\\
11.5681818181818	1.33844922364191\\
11.5909090909091	1.34609764303704\\
11.6136363636364	1.35357038895467\\
11.6363636363636	1.36086648615853\\
11.6590909090909	1.367984982466\\
11.6818181818182	1.37492494887242\\
11.7045454545455	1.38168547967227\\
11.7272727272727	1.38826569257746\\
11.75	1.39466472883238\\
11.7727272727273	1.40088175332602\\
11.7954545454545	1.40691595470098\\
11.8181818181818	1.41276654545929\\
11.8409090909091	1.41843276206524\\
11.8636363636364	1.423913865045\\
11.8863636363636	1.42920913908315\\
11.9090909090909	1.43431789311598\\
11.9318181818182	1.43923946042175\\
11.9545454545455	1.44397319870763\\
11.9772727272727	1.44851849019358\\
12	1.45287474169295\\
12.0227272727273	1.45704138468987\\
12.0454545454545	1.4610178754135\\
12.0681818181818	1.46480369490895\\
12.0909090909091	1.46839834910501\\
12.1136363636364	1.47180136887866\\
12.1363636363636	1.47501231011626\\
12.1590909090909	1.47803075377154\\
12.1818181818182	1.48085630592024\\
12.2045454545455	1.48348859781159\\
12.2272727272727	1.48592728591635\\
12.25	1.48817205197172\\
12.2727272727273	1.49022260302283\\
12.2954545454545	1.49207867146098\\
12.3181818181818	1.49374001505858\\
12.3409090909091	1.49520641700074\\
12.3636363636364	1.49647768591358\\
12.3863636363636	1.4975536558892\\
12.4090909090909	1.49843418650735\\
12.4318181818182	1.4991191628537\\
12.4545454545455	1.49960849553492\\
12.4772727272727	1.49990212069028\\
12.5	1.5\\
12.5227272727273	1.49990212069028\\
12.5454545454545	1.49960849553492\\
12.5681818181818	1.4991191628537\\
12.5909090909091	1.49843418650735\\
12.6136363636364	1.4975536558892\\
12.6363636363636	1.49647768591358\\
12.6590909090909	1.49520641700074\\
12.6818181818182	1.49374001505858\\
12.7045454545455	1.49207867146098\\
12.7272727272727	1.49022260302283\\
12.75	1.48817205197172\\
12.7727272727273	1.48592728591635\\
12.7954545454545	1.48348859781159\\
12.8181818181818	1.48085630592024\\
12.8409090909091	1.47803075377154\\
12.8636363636364	1.47501231011626\\
12.8863636363636	1.47180136887866\\
12.9090909090909	1.46839834910501\\
12.9318181818182	1.46480369490895\\
12.9545454545455	1.4610178754135\\
12.9772727272727	1.45704138468987\\
13	1.45287474169295\\
13.0227272727273	1.44851849019358\\
13.0454545454545	1.44397319870763\\
13.0681818181818	1.43923946042175\\
13.0909090909091	1.43431789311598\\
13.1136363636364	1.42920913908315\\
13.1363636363636	1.423913865045\\
13.1590909090909	1.41843276206524\\
13.1818181818182	1.41276654545929\\
13.2045454545455	1.40691595470098\\
13.2272727272727	1.40088175332602\\
13.25	1.39466472883238\\
13.2727272727273	1.38826569257746\\
13.2954545454545	1.38168547967228\\
13.3181818181818	1.37492494887242\\
13.3409090909091	1.36798498246601\\
13.3636363636364	1.36086648615853\\
13.3863636363636	1.35357038895467\\
13.4090909090909	1.34609764303704\\
13.4318181818182	1.33844922364191\\
13.4545454545455	1.33062612893197\\
13.4772727272727	1.32262937986603\\
13.5	1.3144600200658\\
13.5227272727273	1.30611911567967\\
13.5454545454545	1.29760775524359\\
13.5681818181818	1.28892704953902\\
13.5909090909091	1.28007813144792\\
13.6136363636364	1.27106215580496\\
13.6363636363636	1.26188029924677\\
13.6590909090909	1.25253376005839\\
13.6818181818182	1.24302375801689\\
13.7045454545455	1.23335153423218\\
13.7272727272727	1.22351835098505\\
13.75	1.21352549156242\\
13.7727272727273	1.20337426008986\\
13.7954545454545	1.19306598136142\\
13.8181818181818	1.18260200066669\\
13.8409090909091	1.1719836836153\\
13.8636363636364	1.16121241595863\\
13.8863636363636	1.150289603409\\
13.9090909090909	1.13921667145622\\
13.9318181818182	1.12799506518154\\
13.9545454545455	1.11662624906906\\
13.9772727272727	1.10511170681461\\
14	1.09345294113212\\
14.0227272727273	1.0816514735575\\
14.0454545454545	1.0697088442501\\
14.0681818181818	1.05762661179166\\
14.0909090909091	1.04540635298294\\
14.1136363636364	1.03304966263794\\
14.1363636363636	1.02055815337576\\
14.1590909090909	1.00793345541013\\
14.1818181818182	0.995177216336689\\
14.2045454545455	0.982291100917928\\
14.2272727272727	0.969276790865962\\
14.25	0.956135984623035\\
14.2727272727273	0.942870397139875\\
14.2954545454545	0.929481759651871\\
14.3181818181818	0.915971819453158\\
14.3409090909091	0.90234233966856\\
14.3636363636364	0.888595099023504\\
14.3863636363636	0.874731891611899\\
14.4090909090909	0.860754526661969\\
14.4318181818182	0.846664828300154\\
14.4545454545455	0.83246463531306\\
14.4772727272727	0.818155800907472\\
14.5	0.803740192468496\\
14.5227272727273	0.789219691315873\\
14.5454545454545	0.774596192458443\\
14.5681818181818	0.759871604346839\\
14.5909090909091	0.745047848624418\\
14.6136363636364	0.730126859876485\\
14.6363636363636	0.715110585377815\\
14.6590909090909	0.70000098483851\\
14.6818181818182	0.684800030148275\\
14.7045454545455	0.669509705119039\\
14.7272727272727	0.654132005226083\\
14.75	0.638668937347611\\
14.7727272727273	0.623122519502831\\
14.7954545454545	0.607494780588609\\
14.8181818181818	0.591787760114677\\
14.8409090909091	0.576003507937459\\
14.8636363636364	0.560144083992569\\
14.8863636363636	0.544211558025963\\
14.9090909090909	0.528208009323833\\
14.9318181818182	0.512135526441246\\
14.9545454545455	0.495996206929578\\
14.9772727272727	0.479792157062763\\
15	0.463525491562422\\
15.0227272727273	0.447198333321877\\
15.0454545454545	0.430812813129101\\
15.0681818181818	0.414371069388632\\
15.0909090909091	0.397875247842507\\
15.1136363636364	0.381327501290233\\
15.1363636363636	0.364729989307814\\
15.1590909090909	0.348084877965949\\
15.1818181818182	0.331394339547307\\
15.2045454545455	0.314660552263064\\
15.2272727272727	0.29788569996862\\
15.25	0.281071971878589\\
15.2727272727273	0.264221562281103\\
15.2954545454545	0.247336670251445\\
15.3181818181818	0.230419499365049\\
15.3409090909091	0.213472257409929\\
15.3636363636364	0.19649715609854\\
15.3863636363636	0.179496410779149\\
15.4090909090909	0.162472240146714\\
15.4318181818182	0.145426865953324\\
15.4545454545455	0.128362512718261\\
15.4772727272727	0.111281407437681\\
15.5	0.0941857792939701\\
15.5227272727273	0.0770778593648467\\
15.5454545454545	0.0599598803321613\\
15.5681818181818	0.0428340761905452\\
15.5909090909091	0.0257026819558516\\
15.6136363636364	0.00856793337346482\\
15.6363636363636	-0.00856793337346106\\
15.6590909090909	-0.0257026819558451\\
15.6818181818182	-0.0428340761905441\\
15.7045454545455	-0.0599598803321575\\
15.7272727272727	-0.0770778593648429\\
15.75	-0.094185779293969\\
15.7727272727273	-0.111281407437678\\
15.7954545454545	-0.128362512718257\\
15.8181818181818	-0.14542686595332\\
15.8409090909091	-0.162472240146711\\
15.8636363636364	-0.179496410779145\\
15.8863636363636	-0.196497156098537\\
15.9090909090909	-0.213472257409928\\
15.9318181818182	-0.230419499365048\\
15.9545454545455	-0.247336670251444\\
15.9772727272727	-0.2642215622811\\
16	-0.281071971878587\\
16.0227272727273	-0.297885699968617\\
16.0454545454545	-0.31466055226306\\
16.0681818181818	-0.331394339547306\\
16.0909090909091	-0.348084877965945\\
16.1136363636364	-0.364729989307813\\
16.1363636363636	-0.381327501290229\\
16.1590909090909	-0.397875247842506\\
16.1818181818182	-0.414371069388628\\
16.2045454545455	-0.430812813129098\\
16.2272727272727	-0.447198333321876\\
16.25	-0.463525491562421\\
16.2727272727273	-0.479792157062759\\
16.2954545454545	-0.495996206929575\\
16.3181818181818	-0.512135526441245\\
16.3409090909091	-0.528208009323829\\
16.3636363636364	-0.544211558025959\\
16.3863636363636	-0.560144083992568\\
16.4090909090909	-0.576003507937456\\
16.4318181818182	-0.591787760114671\\
16.4545454545455	-0.607494780588608\\
16.4772727272727	-0.623122519502828\\
16.5	-0.638668937347605\\
16.5227272727273	-0.654132005226082\\
16.5454545454545	-0.669509705119038\\
16.5681818181818	-0.684800030148274\\
16.5909090909091	-0.700000984838511\\
16.6136363636364	-0.715110585377811\\
16.6363636363636	-0.730126859876484\\
16.6590909090909	-0.745047848624417\\
16.6818181818182	-0.759871604346836\\
16.7045454545455	-0.774596192458442\\
16.7272727272727	-0.789219691315872\\
16.75	-0.803740192468493\\
16.7727272727273	-0.818155800907468\\
16.7954545454545	-0.83246463531306\\
16.8181818181818	-0.846664828300151\\
16.8409090909091	-0.860754526661966\\
16.8636363636364	-0.874731891611896\\
16.8863636363636	-0.888595099023506\\
16.9090909090909	-0.902342339668557\\
16.9318181818182	-0.915971819453155\\
16.9545454545455	-0.929481759651872\\
16.9772727272727	-0.942870397139872\\
17	-0.956135984623034\\
17.0227272727273	-0.969276790865961\\
17.0454545454545	-0.982291100917926\\
17.0681818181818	-0.995177216336686\\
17.0909090909091	-1.00793345541013\\
17.1136363636364	-1.02055815337576\\
17.1363636363636	-1.03304966263794\\
17.1590909090909	-1.04540635298294\\
17.1818181818182	-1.05762661179166\\
17.2045454545455	-1.0697088442501\\
17.2272727272727	-1.0816514735575\\
17.25	-1.09345294113212\\
17.2727272727273	-1.10511170681461\\
17.2954545454545	-1.11662624906906\\
17.3181818181818	-1.12799506518154\\
17.3409090909091	-1.13921667145622\\
17.3636363636364	-1.150289603409\\
17.3863636363636	-1.16121241595862\\
17.4090909090909	-1.1719836836153\\
17.4318181818182	-1.18260200066669\\
17.4545454545455	-1.19306598136142\\
17.4772727272727	-1.20337426008986\\
17.5	-1.21352549156242\\
17.5227272727273	-1.22351835098505\\
17.5454545454545	-1.23335153423218\\
17.5681818181818	-1.24302375801689\\
17.5909090909091	-1.25253376005839\\
17.6136363636364	-1.26188029924677\\
17.6363636363636	-1.27106215580496\\
17.6590909090909	-1.28007813144792\\
17.6818181818182	-1.28892704953902\\
17.7045454545455	-1.29760775524359\\
17.7272727272727	-1.30611911567967\\
17.75	-1.31446002006579\\
17.7727272727273	-1.32262937986603\\
17.7954545454545	-1.33062612893197\\
17.8181818181818	-1.33844922364191\\
17.8409090909091	-1.34609764303704\\
17.8636363636364	-1.35357038895467\\
17.8863636363636	-1.36086648615853\\
17.9090909090909	-1.367984982466\\
17.9318181818182	-1.37492494887242\\
17.9545454545455	-1.38168547967227\\
17.9772727272727	-1.38826569257746\\
18	-1.39466472883238\\
18.0227272727273	-1.40088175332602\\
18.0454545454545	-1.40691595470098\\
18.0681818181818	-1.41276654545929\\
18.0909090909091	-1.41843276206524\\
18.1136363636364	-1.423913865045\\
18.1363636363636	-1.42920913908315\\
18.1590909090909	-1.43431789311598\\
18.1818181818182	-1.43923946042175\\
18.2045454545455	-1.44397319870763\\
18.2272727272727	-1.44851849019358\\
18.25	-1.45287474169295\\
18.2727272727273	-1.45704138468987\\
18.2954545454545	-1.4610178754135\\
18.3181818181818	-1.46480369490895\\
18.3409090909091	-1.46839834910501\\
18.3636363636364	-1.47180136887866\\
18.3863636363636	-1.47501231011626\\
18.4090909090909	-1.47803075377154\\
18.4318181818182	-1.48085630592024\\
18.4545454545455	-1.48348859781159\\
18.4772727272727	-1.48592728591635\\
18.5	-1.48817205197172\\
18.5227272727273	-1.49022260302283\\
18.5454545454545	-1.49207867146098\\
18.5681818181818	-1.49374001505858\\
18.5909090909091	-1.49520641700074\\
18.6136363636364	-1.49647768591358\\
18.6363636363636	-1.4975536558892\\
18.6590909090909	-1.49843418650735\\
18.6818181818182	-1.4991191628537\\
18.7045454545455	-1.49960849553492\\
18.7272727272727	-1.49990212069028\\
18.75	-1.5\\
18.7727272727273	-1.49990212069028\\
18.7954545454545	-1.49960849553492\\
18.8181818181818	-1.4991191628537\\
18.8409090909091	-1.49843418650735\\
18.8636363636364	-1.4975536558892\\
18.8863636363636	-1.49647768591358\\
18.9090909090909	-1.49520641700074\\
18.9318181818182	-1.49374001505858\\
18.9545454545455	-1.49207867146098\\
18.9772727272727	-1.49022260302283\\
19	-1.48817205197172\\
19.0227272727273	-1.48592728591635\\
19.0454545454545	-1.48348859781159\\
19.0681818181818	-1.48085630592024\\
19.0909090909091	-1.47803075377154\\
19.1136363636364	-1.47501231011626\\
19.1363636363636	-1.47180136887866\\
19.1590909090909	-1.46839834910501\\
19.1818181818182	-1.46480369490895\\
19.2045454545455	-1.4610178754135\\
19.2272727272727	-1.45704138468987\\
19.25	-1.45287474169295\\
19.2727272727273	-1.44851849019358\\
19.2954545454545	-1.44397319870763\\
19.3181818181818	-1.43923946042175\\
19.3409090909091	-1.43431789311598\\
19.3636363636364	-1.42920913908315\\
19.3863636363636	-1.423913865045\\
19.4090909090909	-1.41843276206524\\
19.4318181818182	-1.41276654545929\\
19.4545454545455	-1.40691595470098\\
19.4772727272727	-1.40088175332602\\
19.5	-1.39466472883238\\
19.5227272727273	-1.38826569257746\\
19.5454545454545	-1.38168547967228\\
19.5681818181818	-1.37492494887242\\
19.5909090909091	-1.36798498246601\\
19.6136363636364	-1.36086648615853\\
19.6363636363636	-1.35357038895467\\
19.6590909090909	-1.34609764303704\\
19.6818181818182	-1.33844922364191\\
19.7045454545455	-1.33062612893197\\
19.7272727272727	-1.32262937986603\\
19.75	-1.3144600200658\\
19.7727272727273	-1.30611911567967\\
19.7954545454545	-1.29760775524359\\
19.8181818181818	-1.28892704953902\\
19.8409090909091	-1.28007813144792\\
19.8636363636364	-1.27106215580496\\
19.8863636363636	-1.26188029924677\\
19.9090909090909	-1.25253376005839\\
19.9318181818182	-1.24302375801689\\
19.9545454545455	-1.23335153423218\\
19.9772727272727	-1.22351835098505\\
20	-1.21352549156242\\
20.0227272727273	-1.20337426008986\\
20.0454545454545	-1.19306598136142\\
20.0681818181818	-1.18260200066669\\
20.0909090909091	-1.1719836836153\\
20.1136363636364	-1.16121241595863\\
20.1363636363636	-1.150289603409\\
20.1590909090909	-1.13921667145622\\
20.1818181818182	-1.12799506518154\\
20.2045454545455	-1.11662624906906\\
20.2272727272727	-1.10511170681461\\
20.25	-1.09345294113212\\
20.2727272727273	-1.08165147355751\\
20.2954545454545	-1.0697088442501\\
20.3181818181818	-1.05762661179166\\
20.3409090909091	-1.04540635298294\\
20.3636363636364	-1.03304966263794\\
20.3863636363636	-1.02055815337576\\
20.4090909090909	-1.00793345541013\\
20.4318181818182	-0.995177216336689\\
20.4545454545455	-0.982291100917927\\
20.4772727272727	-0.969276790865962\\
20.5	-0.956135984623037\\
20.5227272727273	-0.942870397139873\\
20.5454545454545	-0.929481759651873\\
20.5681818181818	-0.91597181945316\\
20.5909090909091	-0.902342339668558\\
20.6136363636364	-0.888595099023507\\
20.6363636363636	-0.874731891611899\\
20.6590909090909	-0.860754526661967\\
20.6818181818182	-0.846664828300156\\
20.7045454545455	-0.832464635313063\\
20.7272727272727	-0.818155800907472\\
20.75	-0.803740192468496\\
20.7727272727273	-0.789219691315873\\
20.7954545454545	-0.774596192458443\\
20.8181818181818	-0.759871604346839\\
20.8409090909091	-0.745047848624418\\
20.8636363636364	-0.730126859876485\\
20.8863636363636	-0.715110585377815\\
20.9090909090909	-0.700000984838512\\
20.9318181818182	-0.684800030148275\\
20.9545454545455	-0.669509705119039\\
20.9772727272727	-0.654132005226086\\
21	-0.638668937347611\\
21.0227272727273	-0.623122519502831\\
21.0454545454545	-0.607494780588612\\
21.0681818181818	-0.591787760114677\\
21.0909090909091	-0.576003507937459\\
21.1136363636364	-0.560144083992571\\
21.1363636363636	-0.544211558025963\\
21.1590909090909	-0.528208009323835\\
21.1818181818182	-0.512135526441246\\
21.2045454545455	-0.495996206929579\\
21.2272727272727	-0.47979215706276\\
21.25	-0.463525491562422\\
21.2727272727273	-0.447198333321877\\
21.2954545454545	-0.430812813129099\\
21.3181818181818	-0.414371069388632\\
21.3409090909091	-0.397875247842508\\
21.3636363636364	-0.381327501290231\\
21.3863636363636	-0.364729989307817\\
21.4090909090909	-0.348084877965949\\
21.4318181818182	-0.331394339547307\\
21.4545454545455	-0.314660552263067\\
21.4772727272727	-0.297885699968621\\
21.5	-0.281071971878589\\
21.5227272727273	-0.264221562281106\\
21.5454545454545	-0.247336670251448\\
21.5681818181818	-0.230419499365049\\
21.5909090909091	-0.21347225740993\\
21.6136363636364	-0.19649715609854\\
21.6363636363636	-0.179496410779149\\
21.6590909090909	-0.162472240146712\\
21.6818181818182	-0.145426865953324\\
21.7045454545455	-0.128362512718261\\
21.7272727272727	-0.111281407437679\\
21.75	-0.0941857792939729\\
21.7727272727273	-0.0770778593648469\\
21.7954545454545	-0.0599598803321614\\
21.8181818181818	-0.042834076190548\\
21.8409090909091	-0.0257026819558518\\
21.8636363636364	-0.00856793337346501\\
21.8863636363636	0.00856793337345821\\
21.9090909090909	0.0257026819558476\\
21.9318181818182	0.0428340761905412\\
21.9545454545455	0.0599598803321546\\
21.9772727272727	0.0770778593648427\\
22	0.0941857792939661\\
22.0227272727273	0.111281407437678\\
22.0454545454545	0.12836251271826\\
22.0681818181818	0.145426865953323\\
22.0909090909091	0.16247224014671\\
22.1136363636364	0.179496410779148\\
22.1363636363636	0.196497156098539\\
22.1590909090909	0.213472257409926\\
22.1818181818182	0.230419499365048\\
22.2045454545455	0.247336670251444\\
22.2272727272727	0.264221562281102\\
22.25	0.281071971878585\\
22.2727272727273	0.297885699968617\\
22.2954545454545	0.314660552263063\\
22.3181818181818	0.331394339547303\\
22.3409090909091	0.348084877965945\\
22.3636363636364	0.364729989307813\\
22.3863636363636	0.381327501290227\\
22.4090909090909	0.397875247842506\\
22.4318181818182	0.414371069388628\\
22.4545454545455	0.430812813129097\\
22.4772727272727	0.447198333321875\\
22.5	0.46352549156242\\
22.5227272727273	0.479792157062759\\
22.5454545454545	0.495996206929577\\
22.5681818181818	0.512135526441245\\
22.5909090909091	0.528208009323829\\
22.6136363636364	0.544211558025961\\
22.6363636363636	0.560144083992565\\
22.6590909090909	0.576003507937453\\
22.6818181818182	0.591787760114673\\
22.7045454545455	0.607494780588605\\
22.7272727272727	0.623122519502825\\
22.75	0.638668937347607\\
22.7727272727273	0.65413200522608\\
22.7954545454545	0.669509705119033\\
22.8181818181818	0.684800030148272\\
22.8409090909091	0.700000984838506\\
22.8636363636364	0.715110585377811\\
22.8863636363636	0.730126859876484\\
22.9090909090909	0.745047848624417\\
22.9318181818182	0.759871604346838\\
22.9545454545455	0.774596192458442\\
22.9772727272727	0.789219691315872\\
23	0.803740192468495\\
23.0227272727273	0.818155800907468\\
23.0454545454545	0.832464635313059\\
23.0681818181818	0.846664828300153\\
23.0909090909091	0.860754526661964\\
23.1136363636364	0.874731891611896\\
23.1363636363636	0.888595099023503\\
23.1590909090909	0.902342339668555\\
23.1818181818182	0.915971819453157\\
23.2045454545455	0.92948175965187\\
23.2272727272727	0.94287039713987\\
23.25	0.956135984623034\\
23.2727272727273	0.969276790865959\\
23.2954545454545	0.982291100917925\\
23.3181818181818	0.995177216336686\\
23.3409090909091	1.00793345541013\\
23.3636363636364	1.02055815337576\\
23.3863636363636	1.03304966263794\\
23.4090909090909	1.04540635298294\\
23.4318181818182	1.05762661179166\\
23.4545454545455	1.0697088442501\\
23.4772727272727	1.0816514735575\\
23.5	1.09345294113212\\
23.5227272727273	1.10511170681461\\
23.5454545454545	1.11662624906906\\
23.5681818181818	1.12799506518154\\
23.5909090909091	1.13921667145622\\
23.6136363636364	1.15028960340899\\
23.6363636363636	1.16121241595862\\
23.6590909090909	1.17198368361529\\
23.6818181818182	1.18260200066669\\
23.7045454545455	1.19306598136142\\
23.7272727272727	1.20337426008986\\
23.75	1.21352549156242\\
23.7727272727273	1.22351835098505\\
23.7954545454545	1.23335153423218\\
23.8181818181818	1.24302375801689\\
23.8409090909091	1.25253376005839\\
23.8636363636364	1.26188029924677\\
23.8863636363636	1.27106215580496\\
23.9090909090909	1.28007813144792\\
23.9318181818182	1.28892704953902\\
23.9545454545455	1.29760775524359\\
23.9772727272727	1.30611911567966\\
24	1.31446002006579\\
24.0227272727273	1.32262937986603\\
24.0454545454545	1.33062612893197\\
24.0681818181818	1.33844922364191\\
24.0909090909091	1.34609764303704\\
24.1136363636364	1.35357038895467\\
24.1363636363636	1.36086648615853\\
24.1590909090909	1.367984982466\\
24.1818181818182	1.37492494887242\\
24.2045454545455	1.38168547967228\\
24.2272727272727	1.38826569257746\\
24.25	1.39466472883238\\
24.2727272727273	1.40088175332602\\
24.2954545454545	1.40691595470098\\
24.3181818181818	1.41276654545929\\
24.3409090909091	1.41843276206524\\
24.3636363636364	1.423913865045\\
24.3863636363636	1.42920913908315\\
24.4090909090909	1.43431789311598\\
24.4318181818182	1.43923946042175\\
24.4545454545455	1.44397319870763\\
24.4772727272727	1.44851849019358\\
24.5	1.45287474169295\\
24.5227272727273	1.45704138468987\\
24.5454545454545	1.4610178754135\\
24.5681818181818	1.46480369490895\\
24.5909090909091	1.46839834910501\\
24.6136363636364	1.47180136887866\\
24.6363636363636	1.47501231011626\\
24.6590909090909	1.47803075377154\\
24.6818181818182	1.48085630592024\\
24.7045454545455	1.48348859781159\\
24.7272727272727	1.48592728591635\\
24.75	1.48817205197172\\
24.7727272727273	1.49022260302283\\
24.7954545454545	1.49207867146098\\
24.8181818181818	1.49374001505858\\
24.8409090909091	1.49520641700074\\
24.8636363636364	1.49647768591358\\
24.8863636363636	1.4975536558892\\
24.9090909090909	1.49843418650735\\
24.9318181818182	1.4991191628537\\
24.9545454545455	1.49960849553492\\
24.9772727272727	1.49990212069028\\
25	1.5\\
25.0227272727273	1.49990212069028\\
25.0454545454545	1.49960849553492\\
25.0681818181818	1.4991191628537\\
25.0909090909091	1.49843418650735\\
25.1136363636364	1.4975536558892\\
25.1363636363636	1.49647768591358\\
25.1590909090909	1.49520641700074\\
25.1818181818182	1.49374001505858\\
25.2045454545455	1.49207867146098\\
25.2272727272727	1.49022260302283\\
25.25	1.48817205197172\\
25.2727272727273	1.48592728591635\\
25.2954545454545	1.48348859781159\\
25.3181818181818	1.48085630592024\\
25.3409090909091	1.47803075377154\\
25.3636363636364	1.47501231011626\\
25.3863636363636	1.47180136887866\\
25.4090909090909	1.46839834910501\\
25.4318181818182	1.46480369490895\\
25.4545454545455	1.4610178754135\\
25.4772727272727	1.45704138468987\\
25.5	1.45287474169295\\
25.5227272727273	1.44851849019358\\
25.5454545454545	1.44397319870763\\
25.5681818181818	1.43923946042175\\
25.5909090909091	1.43431789311598\\
25.6136363636364	1.42920913908315\\
25.6363636363636	1.423913865045\\
25.6590909090909	1.41843276206524\\
25.6818181818182	1.41276654545929\\
25.7045454545455	1.40691595470098\\
25.7272727272727	1.40088175332602\\
25.75	1.39466472883238\\
25.7727272727273	1.38826569257746\\
25.7954545454545	1.38168547967228\\
25.8181818181818	1.37492494887242\\
25.8409090909091	1.36798498246601\\
25.8636363636364	1.36086648615853\\
25.8863636363636	1.35357038895467\\
25.9090909090909	1.34609764303704\\
25.9318181818182	1.33844922364191\\
25.9545454545455	1.33062612893197\\
25.9772727272727	1.32262937986603\\
26	1.3144600200658\\
26.0227272727273	1.30611911567967\\
26.0454545454545	1.29760775524359\\
26.0681818181818	1.28892704953902\\
26.0909090909091	1.28007813144792\\
26.1136363636364	1.27106215580497\\
26.1363636363636	1.26188029924677\\
26.1590909090909	1.25253376005839\\
26.1818181818182	1.24302375801689\\
26.2045454545455	1.23335153423218\\
26.2272727272727	1.22351835098506\\
26.25	1.21352549156242\\
26.2727272727273	1.20337426008986\\
26.2954545454545	1.19306598136142\\
26.3181818181818	1.18260200066669\\
26.3409090909091	1.1719836836153\\
26.3636363636364	1.16121241595862\\
26.3863636363636	1.150289603409\\
26.4090909090909	1.13921667145622\\
26.4318181818182	1.12799506518154\\
26.4545454545455	1.11662624906906\\
26.4772727272727	1.10511170681461\\
26.5	1.09345294113212\\
26.5227272727273	1.08165147355751\\
26.5454545454545	1.0697088442501\\
26.5681818181818	1.05762661179166\\
26.5909090909091	1.04540635298294\\
26.6136363636364	1.03304966263794\\
26.6363636363636	1.02055815337576\\
26.6590909090909	1.00793345541014\\
26.6818181818182	0.995177216336689\\
26.7045454545455	0.982291100917931\\
26.7272727272727	0.969276790865962\\
26.75	0.956135984623035\\
26.7727272727273	0.942870397139873\\
26.7954545454545	0.929481759651873\\
26.8181818181818	0.915971819453158\\
26.8409090909091	0.902342339668558\\
26.8636363636364	0.888595099023507\\
26.8863636363636	0.874731891611899\\
26.9090909090909	0.860754526661967\\
26.9318181818182	0.846664828300154\\
26.9545454545455	0.832464635313063\\
26.9772727272727	0.818155800907472\\
27	0.803740192468496\\
27.0227272727273	0.789219691315876\\
27.0454545454545	0.774596192458445\\
27.0681818181818	0.759871604346839\\
27.0909090909091	0.745047848624421\\
27.1136363636364	0.730126859876485\\
27.1363636363636	0.715110585377815\\
27.1590909090909	0.700000984838513\\
27.1818181818182	0.684800030148275\\
27.2045454545455	0.669509705119039\\
27.2272727272727	0.654132005226086\\
27.25	0.638668937347609\\
27.2727272727273	0.623122519502831\\
27.2954545454545	0.607494780588612\\
27.3181818181818	0.591787760114675\\
27.3409090909091	0.57600350793746\\
27.3636363636364	0.560144083992571\\
27.3863636363636	0.544211558025963\\
27.4090909090909	0.528208009323835\\
27.4318181818182	0.512135526441251\\
27.4545454545455	0.495996206929579\\
27.4772727272727	0.479792157062765\\
27.5	0.463525491562427\\
27.5227272727273	0.44719833332188\\
27.5454545454545	0.430812813129104\\
27.5681818181818	0.414371069388632\\
27.5909090909091	0.397875247842508\\
27.6136363636364	0.381327501290231\\
27.6363636363636	0.364729989307814\\
27.6590909090909	0.348084877965949\\
27.6818181818182	0.331394339547307\\
27.7045454545455	0.314660552263064\\
27.7272727272727	0.297885699968621\\
27.75	0.281071971878589\\
27.7727272727273	0.264221562281104\\
27.7954545454545	0.247336670251448\\
27.8181818181818	0.230419499365052\\
27.8409090909091	0.21347225740993\\
27.8636363636364	0.196497156098543\\
27.8863636363636	0.17949641077915\\
27.9090909090909	0.162472240146715\\
27.9318181818182	0.145426865953327\\
27.9545454545455	0.128362512718261\\
27.9772727272727	0.111281407437682\\
28	0.0941857792939731\\
28.0227272727273	0.0770778593648444\\
28.0454545454545	0.0599598803321616\\
28.0681818181818	0.0428340761905482\\
28.0909090909091	0.0257026819558493\\
28.1136363636364	0.00856793337346519\\
28.1363636363636	-0.00856793337345802\\
28.1590909090909	-0.0257026819558474\\
28.1818181818182	-0.042834076190541\\
28.2045454545455	-0.0599598803321598\\
28.2272727272727	-0.0770778593648426\\
28.25	-0.094185779293966\\
28.2727272727273	-0.111281407437677\\
28.2954545454545	-0.128362512718257\\
28.3181818181818	-0.145426865953317\\
28.3409090909091	-0.16247224014671\\
28.3636363636364	-0.179496410779145\\
28.3863636363636	-0.196497156098539\\
28.4090909090909	-0.213472257409925\\
28.4318181818182	-0.230419499365048\\
28.4545454545454	-0.247336670251444\\
28.4772727272727	-0.264221562281102\\
28.5	-0.281071971878584\\
28.5227272727273	-0.297885699968619\\
28.5454545454545	-0.314660552263063\\
28.5681818181818	-0.331394339547303\\
28.5909090909091	-0.348084877965947\\
28.6136363636364	-0.364729989307813\\
28.6363636363636	-0.381327501290226\\
28.6590909090909	-0.397875247842506\\
28.6818181818182	-0.414371069388628\\
28.7045454545454	-0.430812813129095\\
28.7272727272727	-0.447198333321875\\
28.75	-0.463525491562418\\
28.7727272727273	-0.479792157062756\\
28.7954545454545	-0.495996206929575\\
28.8181818181818	-0.512135526441242\\
28.8409090909091	-0.528208009323831\\
28.8636363636364	-0.544211558025959\\
28.8863636363636	-0.560144083992562\\
28.9090909090909	-0.57600350793746\\
28.9318181818182	-0.591787760114676\\
28.9545454545455	-0.607494780588608\\
28.9772727272727	-0.623122519502827\\
29	-0.638668937347605\\
29.0227272727273	-0.654132005226077\\
29.0454545454545	-0.66950970511904\\
29.0681818181818	-0.684800030148274\\
29.0909090909091	-0.700000984838509\\
29.1136363636364	-0.715110585377809\\
29.1363636363636	-0.730126859876479\\
29.1590909090909	-0.745047848624419\\
29.1818181818182	-0.759871604346838\\
29.2045454545455	-0.774596192458441\\
29.2272727272727	-0.78921969131587\\
29.25	-0.80374019246849\\
29.2727272727273	-0.818155800907468\\
29.2954545454545	-0.832464635313061\\
29.3181818181818	-0.846664828300153\\
29.3409090909091	-0.860754526661963\\
29.3636363636364	-0.874731891611898\\
29.3863636363636	-0.888595099023503\\
29.4090909090909	-0.902342339668559\\
29.4318181818182	-0.915971819453157\\
29.4545454545455	-0.929481759651869\\
29.4772727272727	-0.942870397139869\\
29.5	-0.956135984623034\\
29.5227272727273	-0.969276790865959\\
29.5454545454545	-0.982291100917927\\
29.5681818181818	-0.995177216336686\\
29.5909090909091	-1.00793345541013\\
29.6136363636364	-1.02055815337576\\
29.6363636363636	-1.03304966263794\\
29.6590909090909	-1.04540635298294\\
29.6818181818182	-1.05762661179166\\
29.7045454545455	-1.0697088442501\\
29.7272727272727	-1.0816514735575\\
29.75	-1.09345294113212\\
29.7727272727273	-1.10511170681461\\
29.7954545454545	-1.11662624906906\\
29.8181818181818	-1.12799506518154\\
29.8409090909091	-1.13921667145622\\
29.8636363636364	-1.150289603409\\
29.8863636363636	-1.16121241595862\\
29.9090909090909	-1.1719836836153\\
29.9318181818182	-1.18260200066669\\
29.9545454545455	-1.19306598136141\\
29.9772727272727	-1.20337426008986\\
30	-1.21352549156242\\
30.0227272727273	-1.22351835098505\\
30.0454545454545	-1.23335153423218\\
30.0681818181818	-1.24302375801688\\
30.0909090909091	-1.25253376005839\\
30.1136363636364	-1.26188029924677\\
30.1363636363636	-1.27106215580496\\
30.1590909090909	-1.28007813144792\\
30.1818181818182	-1.28892704953902\\
30.2045454545455	-1.29760775524359\\
30.2272727272727	-1.30611911567966\\
30.25	-1.3144600200658\\
30.2727272727273	-1.32262937986603\\
30.2954545454545	-1.33062612893197\\
30.3181818181818	-1.33844922364191\\
30.3409090909091	-1.34609764303704\\
30.3636363636364	-1.35357038895467\\
30.3863636363636	-1.36086648615853\\
30.4090909090909	-1.367984982466\\
30.4318181818182	-1.37492494887242\\
30.4545454545455	-1.38168547967227\\
30.4772727272727	-1.38826569257746\\
30.5	-1.39466472883238\\
30.5227272727273	-1.40088175332602\\
30.5454545454545	-1.40691595470098\\
30.5681818181818	-1.41276654545928\\
30.5909090909091	-1.41843276206524\\
30.6136363636364	-1.423913865045\\
30.6363636363636	-1.42920913908315\\
30.6590909090909	-1.43431789311598\\
30.6818181818182	-1.43923946042175\\
30.7045454545455	-1.44397319870763\\
30.7272727272727	-1.44851849019358\\
30.75	-1.45287474169295\\
30.7727272727273	-1.45704138468987\\
30.7954545454545	-1.4610178754135\\
30.8181818181818	-1.46480369490894\\
30.8409090909091	-1.46839834910501\\
30.8636363636364	-1.47180136887866\\
30.8863636363636	-1.47501231011626\\
30.9090909090909	-1.47803075377154\\
30.9318181818182	-1.48085630592024\\
30.9545454545455	-1.48348859781159\\
30.9772727272727	-1.48592728591635\\
31	-1.48817205197172\\
31.0227272727273	-1.49022260302282\\
31.0454545454545	-1.49207867146098\\
31.0681818181818	-1.49374001505858\\
31.0909090909091	-1.49520641700074\\
31.1136363636364	-1.49647768591358\\
31.1363636363636	-1.4975536558892\\
31.1590909090909	-1.49843418650735\\
31.1818181818182	-1.4991191628537\\
31.2045454545455	-1.49960849553492\\
31.2272727272727	-1.49990212069028\\
31.25	-1.5\\
31.2727272727273	-1.49990212069028\\
31.2954545454545	-1.49960849553492\\
31.3181818181818	-1.4991191628537\\
31.3409090909091	-1.49843418650735\\
31.3636363636364	-1.4975536558892\\
31.3863636363636	-1.49647768591358\\
31.4090909090909	-1.49520641700074\\
31.4318181818182	-1.49374001505858\\
31.4545454545454	-1.49207867146098\\
31.4772727272727	-1.49022260302283\\
31.5	-1.48817205197172\\
31.5227272727273	-1.48592728591635\\
31.5454545454545	-1.48348859781159\\
31.5681818181818	-1.48085630592024\\
31.5909090909091	-1.47803075377154\\
31.6136363636364	-1.47501231011626\\
31.6363636363636	-1.47180136887866\\
31.6590909090909	-1.46839834910501\\
31.6818181818182	-1.46480369490895\\
31.7045454545454	-1.4610178754135\\
31.7272727272727	-1.45704138468987\\
31.75	-1.45287474169295\\
31.7727272727273	-1.44851849019358\\
31.7954545454545	-1.44397319870763\\
31.8181818181818	-1.43923946042175\\
31.8409090909091	-1.43431789311598\\
31.8636363636364	-1.42920913908315\\
31.8863636363636	-1.423913865045\\
31.9090909090909	-1.41843276206524\\
31.9318181818182	-1.41276654545929\\
31.9545454545454	-1.40691595470098\\
31.9772727272727	-1.40088175332603\\
32	-1.39466472883238\\
32.0227272727273	-1.38826569257746\\
32.0454545454545	-1.38168547967228\\
32.0681818181818	-1.37492494887242\\
32.0909090909091	-1.36798498246601\\
32.1136363636364	-1.36086648615853\\
32.1363636363636	-1.35357038895467\\
32.1590909090909	-1.34609764303704\\
32.1818181818182	-1.33844922364191\\
32.2045454545455	-1.33062612893197\\
32.2272727272727	-1.32262937986603\\
32.25	-1.31446002006579\\
32.2727272727273	-1.30611911567967\\
32.2954545454545	-1.29760775524359\\
32.3181818181818	-1.28892704953902\\
32.3409090909091	-1.28007813144793\\
32.3636363636364	-1.27106215580497\\
32.3863636363636	-1.26188029924677\\
32.4090909090909	-1.25253376005839\\
32.4318181818182	-1.24302375801689\\
32.4545454545455	-1.23335153423218\\
32.4772727272727	-1.22351835098506\\
32.5	-1.21352549156242\\
32.5227272727273	-1.20337426008986\\
32.5454545454545	-1.19306598136142\\
32.5681818181818	-1.18260200066669\\
32.5909090909091	-1.1719836836153\\
32.6136363636364	-1.16121241595862\\
32.6363636363636	-1.150289603409\\
32.6590909090909	-1.13921667145622\\
32.6818181818182	-1.12799506518154\\
32.7045454545455	-1.11662624906906\\
32.7272727272727	-1.10511170681461\\
32.75	-1.09345294113212\\
32.7727272727273	-1.08165147355751\\
32.7954545454545	-1.0697088442501\\
32.8181818181818	-1.05762661179166\\
32.8409090909091	-1.04540635298295\\
32.8636363636364	-1.03304966263795\\
32.8863636363636	-1.02055815337576\\
32.9090909090909	-1.00793345541013\\
32.9318181818182	-0.995177216336691\\
32.9545454545455	-0.982291100917933\\
32.9772727272727	-0.969276790865968\\
33	-0.956135984623043\\
33.0227272727273	-0.942870397139875\\
33.0454545454545	-0.929481759651875\\
33.0681818181818	-0.915971819453163\\
33.0909090909091	-0.902342339668556\\
33.1136363636364	-0.888595099023505\\
33.1363636363636	-0.8747318916119\\
33.1590909090909	-0.860754526661969\\
33.1818181818182	-0.84666482830015\\
33.2045454545455	-0.832464635313059\\
33.2272727272727	-0.81815580090747\\
33.25	-0.803740192468496\\
33.2727272727273	-0.789219691315876\\
33.2954545454545	-0.774596192458448\\
33.3181818181818	-0.759871604346835\\
33.3409090909091	-0.745047848624416\\
33.3636363636364	-0.730126859876485\\
33.3863636363636	-0.715110585377815\\
33.4090909090909	-0.700000984838515\\
33.4318181818182	-0.68480003014828\\
33.4545454545455	-0.669509705119037\\
33.4772727272727	-0.654132005226084\\
33.5	-0.638668937347611\\
33.5227272727273	-0.623122519502834\\
33.5454545454545	-0.607494780588614\\
33.5681818181818	-0.591787760114672\\
33.5909090909091	-0.576003507937457\\
33.6136363636364	-0.560144083992569\\
33.6363636363636	-0.544211558025965\\
33.6590909090909	-0.528208009323828\\
33.6818181818182	-0.512135526441244\\
33.7045454545455	-0.495996206929576\\
33.7272727272727	-0.479792157062763\\
33.75	-0.463525491562425\\
33.7727272727273	-0.447198333321872\\
33.7954545454545	-0.430812813129097\\
33.8181818181818	-0.41437106938863\\
33.8409090909091	-0.397875247842508\\
33.8636363636364	-0.381327501290234\\
33.8863636363636	-0.36472998930782\\
33.9090909090909	-0.348084877965944\\
33.9318181818182	-0.331394339547305\\
33.9545454545455	-0.314660552263065\\
33.9772727272727	-0.297885699968621\\
34	-0.281071971878592\\
34.0227272727273	-0.264221562281099\\
34.0454545454545	-0.247336670251443\\
34.0681818181818	-0.23041949936505\\
34.0909090909091	-0.21347225740993\\
34.1136363636364	-0.196497156098543\\
34.1363636363636	-0.179496410779155\\
34.1590909090909	-0.16247224014671\\
34.1818181818182	-0.145426865953322\\
34.2045454545455	-0.128362512718261\\
34.2272727272727	-0.111281407437682\\
34.25	-0.094185779293976\\
34.2727272727273	-0.0770778593648526\\
34.2954545454545	-0.0599598803321591\\
34.3181818181818	-0.0428340761905457\\
34.3409090909091	-0.0257026819558521\\
34.3636363636364	-0.00856793337346804\\
34.3863636363636	0.00856793337345517\\
34.4090909090909	0.0257026819558499\\
34.4318181818182	0.0428340761905435\\
34.4545454545455	0.0599598803321569\\
34.4772727272727	0.0770778593648397\\
34.5	0.0941857792939631\\
34.5227272727273	0.111281407437669\\
34.5454545454545	0.128362512718259\\
34.5681818181818	0.14542686595332\\
34.5909090909091	0.162472240146707\\
34.6136363636364	0.179496410779142\\
34.6363636363636	0.196497156098531\\
34.6590909090909	0.213472257409928\\
34.6818181818182	0.230419499365047\\
34.7045454545455	0.247336670251441\\
34.7272727272727	0.264221562281097\\
34.75	0.281071971878579\\
34.7727272727273	0.297885699968619\\
34.7954545454545	0.314660552263062\\
34.8181818181818	0.331394339547303\\
34.8409090909091	0.348084877965942\\
34.8636363636364	0.364729989307818\\
34.8863636363636	0.381327501290231\\
34.9090909090909	0.397875247842506\\
34.9318181818182	0.414371069388627\\
34.9545454545455	0.430812813129095\\
34.9772727272727	0.44719833332187\\
35	0.463525491562423\\
35.0227272727273	0.479792157062761\\
35.0454545454545	0.495996206929574\\
35.0681818181818	0.512135526441242\\
35.0909090909091	0.528208009323826\\
35.1136363636364	0.544211558025953\\
35.1363636363636	0.560144083992567\\
35.1590909090909	0.576003507937455\\
35.1818181818182	0.59178776011467\\
35.2045454545455	0.607494780588602\\
35.2272727272727	0.623122519502822\\
35.25	0.638668937347609\\
35.2727272727273	0.654132005226082\\
35.2954545454545	0.669509705119035\\
35.3181818181818	0.684800030148278\\
35.3409090909091	0.700000984838513\\
35.3636363636364	0.715110585377813\\
35.3863636363636	0.730126859876483\\
35.4090909090909	0.745047848624414\\
35.4318181818182	0.759871604346842\\
35.4545454545455	0.774596192458446\\
35.4772727272727	0.789219691315874\\
35.5	0.803740192468494\\
35.5227272727273	0.818155800907468\\
35.5454545454545	0.832464635313057\\
35.5681818181818	0.846664828300157\\
35.5909090909091	0.860754526661968\\
35.6136363636364	0.874731891611898\\
35.6363636363636	0.888595099023503\\
35.6590909090909	0.902342339668555\\
35.6818181818182	0.915971819453152\\
35.7045454545455	0.929481759651873\\
35.7272727272727	0.942870397139874\\
35.75	0.956135984623033\\
35.7727272727273	0.969276790865958\\
35.7954545454545	0.982291100917923\\
35.8181818181818	0.99517721633669\\
35.8409090909091	1.00793345541013\\
35.8636363636364	1.02055815337576\\
35.8863636363636	1.03304966263794\\
35.9090909090909	1.04540635298294\\
35.9318181818182	1.05762661179166\\
35.9545454545455	1.0697088442501\\
35.9772727272727	1.0816514735575\\
36	1.09345294113212\\
36.0227272727273	1.1051117068146\\
36.0454545454545	1.11662624906905\\
36.0681818181818	1.12799506518154\\
36.0909090909091	1.13921667145622\\
36.1136363636364	1.150289603409\\
36.1363636363636	1.16121241595862\\
36.1590909090909	1.17198368361529\\
36.1818181818182	1.18260200066669\\
36.2045454545455	1.19306598136142\\
36.2272727272727	1.20337426008986\\
36.25	1.21352549156242\\
36.2727272727273	1.22351835098505\\
36.2954545454545	1.23335153423218\\
36.3181818181818	1.24302375801689\\
36.3409090909091	1.25253376005839\\
36.3636363636364	1.26188029924677\\
36.3863636363636	1.27106215580496\\
36.4090909090909	1.28007813144792\\
36.4318181818182	1.28892704953901\\
36.4545454545455	1.29760775524359\\
36.4772727272727	1.30611911567966\\
36.5	1.31446002006579\\
36.5227272727273	1.32262937986603\\
36.5454545454545	1.33062612893197\\
36.5681818181818	1.33844922364191\\
36.5909090909091	1.34609764303704\\
36.6136363636364	1.35357038895467\\
36.6363636363636	1.36086648615853\\
36.6590909090909	1.367984982466\\
36.6818181818182	1.37492494887242\\
36.7045454545455	1.38168547967227\\
36.7272727272727	1.38826569257746\\
36.75	1.39466472883237\\
36.7727272727273	1.40088175332602\\
36.7954545454545	1.40691595470097\\
36.8181818181818	1.41276654545929\\
36.8409090909091	1.41843276206524\\
36.8636363636364	1.423913865045\\
36.8863636363636	1.42920913908314\\
36.9090909090909	1.43431789311598\\
36.9318181818182	1.43923946042175\\
36.9545454545455	1.44397319870763\\
36.9772727272727	1.44851849019358\\
37	1.45287474169295\\
37.0227272727273	1.45704138468987\\
37.0454545454545	1.4610178754135\\
37.0681818181818	1.46480369490894\\
37.0909090909091	1.46839834910501\\
37.1136363636364	1.47180136887866\\
37.1363636363636	1.47501231011626\\
37.1590909090909	1.47803075377154\\
37.1818181818182	1.48085630592024\\
37.2045454545455	1.48348859781159\\
37.2272727272727	1.48592728591635\\
37.25	1.48817205197172\\
37.2727272727273	1.49022260302283\\
37.2954545454545	1.49207867146098\\
37.3181818181818	1.49374001505858\\
37.3409090909091	1.49520641700074\\
37.3636363636364	1.49647768591358\\
37.3863636363636	1.4975536558892\\
37.4090909090909	1.49843418650735\\
37.4318181818182	1.4991191628537\\
37.4545454545455	1.49960849553492\\
37.4772727272727	1.49990212069028\\
37.5	1.5\\
37.5227272727273	1.49990212069028\\
37.5454545454545	1.49960849553492\\
37.5681818181818	1.4991191628537\\
37.5909090909091	1.49843418650735\\
37.6136363636364	1.4975536558892\\
37.6363636363636	1.49647768591358\\
37.6590909090909	1.49520641700074\\
37.6818181818182	1.49374001505858\\
37.7045454545454	1.49207867146098\\
37.7272727272727	1.49022260302282\\
37.75	1.48817205197172\\
37.7727272727273	1.48592728591635\\
37.7954545454545	1.48348859781159\\
37.8181818181818	1.48085630592025\\
37.8409090909091	1.47803075377154\\
37.8636363636364	1.47501231011626\\
37.8863636363636	1.47180136887866\\
37.9090909090909	1.46839834910501\\
37.9318181818182	1.46480369490895\\
37.9545454545454	1.4610178754135\\
37.9772727272727	1.45704138468987\\
38	1.45287474169295\\
38.0227272727273	1.44851849019358\\
38.0454545454545	1.44397319870763\\
38.0681818181818	1.43923946042175\\
38.0909090909091	1.43431789311598\\
38.1136363636364	1.42920913908315\\
38.1363636363636	1.423913865045\\
38.1590909090909	1.41843276206524\\
38.1818181818182	1.41276654545929\\
38.2045454545454	1.40691595470098\\
38.2272727272727	1.40088175332602\\
38.25	1.39466472883238\\
38.2727272727273	1.38826569257746\\
38.2954545454545	1.38168547967228\\
38.3181818181818	1.37492494887242\\
38.3409090909091	1.36798498246601\\
38.3636363636364	1.36086648615853\\
38.3863636363636	1.35357038895467\\
38.4090909090909	1.34609764303704\\
38.4318181818182	1.33844922364191\\
38.4545454545454	1.33062612893197\\
38.4772727272727	1.32262937986603\\
38.5	1.3144600200658\\
38.5227272727273	1.30611911567967\\
38.5454545454545	1.29760775524359\\
38.5681818181818	1.28892704953902\\
38.5909090909091	1.28007813144793\\
38.6136363636364	1.27106215580497\\
38.6363636363636	1.26188029924677\\
38.6590909090909	1.25253376005839\\
38.6818181818182	1.24302375801689\\
38.7045454545455	1.23335153423218\\
38.7272727272727	1.22351835098505\\
38.75	1.21352549156242\\
38.7727272727273	1.20337426008987\\
38.7954545454545	1.19306598136141\\
38.8181818181818	1.18260200066669\\
38.8409090909091	1.1719836836153\\
38.8636363636364	1.16121241595863\\
38.8863636363636	1.150289603409\\
38.9090909090909	1.13921667145622\\
38.9318181818182	1.12799506518154\\
38.9545454545455	1.11662624906906\\
38.9772727272727	1.10511170681461\\
39	1.09345294113212\\
39.0227272727273	1.08165147355751\\
39.0454545454545	1.06970884425011\\
39.0681818181818	1.05762661179166\\
39.0909090909091	1.04540635298294\\
39.1136363636364	1.03304966263795\\
39.1363636363636	1.02055815337576\\
39.1590909090909	1.00793345541013\\
39.1818181818182	0.995177216336688\\
39.2045454545455	0.982291100917929\\
39.2272727272727	0.969276790865964\\
39.25	0.956135984623031\\
39.2727272727273	0.942870397139871\\
39.2954545454545	0.929481759651871\\
39.3181818181818	0.915971819453158\\
39.3409090909091	0.902342339668561\\
39.3636363636364	0.888595099023501\\
39.3863636363636	0.874731891611895\\
39.4090909090909	0.860754526661965\\
39.4318181818182	0.846664828300155\\
39.4545454545455	0.832464635313063\\
39.4772727272727	0.818155800907474\\
39.5	0.803740192468492\\
39.5227272727273	0.789219691315871\\
39.5454545454545	0.774596192458443\\
39.5681818181818	0.75987160434684\\
39.5909090909091	0.745047848624421\\
39.6136363636364	0.73012685987649\\
39.6363636363636	0.715110585377811\\
39.6590909090909	0.700000984838511\\
39.6818181818182	0.684800030148276\\
39.7045454545455	0.669509705119042\\
39.7272727272727	0.654132005226088\\
39.75	0.638668937347607\\
39.7727272727273	0.623122519502829\\
39.7954545454545	0.60749478058861\\
39.8181818181818	0.591787760114678\\
39.8409090909091	0.576003507937462\\
39.8636363636364	0.560144083992574\\
39.8863636363636	0.544211558025961\\
39.9090909090909	0.528208009323833\\
39.9318181818182	0.512135526441249\\
39.9545454545455	0.495996206929582\\
39.9772727272727	0.479792157062768\\
40	0.46352549156242\\
40.0227272727273	0.447198333321877\\
40.0454545454545	0.430812813129102\\
40.0681818181818	0.414371069388635\\
40.0909090909091	0.397875247842513\\
40.1136363636364	0.381327501290229\\
40.1363636363636	0.364729989307815\\
40.1590909090909	0.348084877965949\\
40.1818181818182	0.33139433954731\\
40.2045454545455	0.31466055226306\\
40.2272727272727	0.297885699968626\\
40.25	0.281071971878587\\
40.2727272727273	0.264221562281104\\
40.2954545454545	0.247336670251448\\
40.3181818181818	0.230419499365055\\
40.3409090909091	0.213472257409925\\
40.3636363636364	0.196497156098549\\
40.3863636363636	0.17949641077915\\
40.4090909090909	0.162472240146715\\
40.4318181818182	0.145426865953327\\
40.4545454545455	0.128362512718267\\
40.4772727272727	0.111281407437677\\
40.5	0.0941857792939814\\
40.5227272727273	0.0770778593648474\\
40.5454545454545	0.0599598803321646\\
40.5681818181818	0.0428340761905512\\
40.5909090909091	0.025702681955847\\
40.6136363636364	0.00856793337346289\\
40.6363636363636	-0.00856793337346032\\
40.6590909090909	-0.0257026819558444\\
40.6818181818182	-0.0428340761905487\\
40.7045454545455	-0.0599598803321514\\
40.7272727272727	-0.0770778593648448\\
40.75	-0.0941857792939682\\
40.7727272727273	-0.111281407437685\\
40.7954545454545	-0.128362512718254\\
40.8181818181818	-0.145426865953325\\
40.8409090909091	-0.162472240146702\\
40.8636363636364	-0.179496410779147\\
40.8863636363636	-0.196497156098536\\
40.9090909090909	-0.213472257409933\\
40.9318181818182	-0.230419499365042\\
40.9545454545455	-0.247336670251446\\
40.9772727272727	-0.264221562281091\\
41	-0.281071971878584\\
41.0227272727273	-0.297885699968613\\
41.0454545454545	-0.314660552263067\\
41.0681818181818	-0.331394339547308\\
41.0909090909091	-0.348084877965947\\
41.1136363636364	-0.364729989307812\\
41.1363636363636	-0.381327501290226\\
41.1590909090909	-0.397875247842511\\
41.1818181818182	-0.414371069388632\\
41.2045454545455	-0.430812813129099\\
41.2272727272727	-0.447198333321875\\
41.25	-0.463525491562417\\
41.2727272727273	-0.479792157062756\\
41.2954545454545	-0.495996206929579\\
41.3181818181818	-0.512135526441247\\
41.3409090909091	-0.528208009323831\\
41.3636363636364	-0.544211558025958\\
41.3863636363636	-0.560144083992562\\
41.4090909090909	-0.57600350793745\\
41.4318181818182	-0.591787760114675\\
41.4545454545455	-0.607494780588607\\
41.4772727272727	-0.623122519502827\\
41.5	-0.638668937347604\\
41.5227272727273	-0.654132005226086\\
41.5454545454545	-0.66950970511904\\
41.5681818181818	-0.684800030148273\\
41.5909090909091	-0.700000984838508\\
41.6136363636364	-0.715110585377808\\
41.6363636363636	-0.730126859876479\\
41.6590909090909	-0.745047848624419\\
41.6818181818182	-0.759871604346837\\
41.7045454545455	-0.774596192458441\\
41.7272727272727	-0.789219691315869\\
41.75	-0.80374019246849\\
41.7727272727273	-0.818155800907463\\
41.7954545454545	-0.832464635313061\\
41.8181818181818	-0.846664828300152\\
41.8409090909091	-0.860754526661963\\
41.8636363636364	-0.874731891611893\\
41.8863636363636	-0.888595099023507\\
41.9090909090909	-0.902342339668559\\
41.9318181818182	-0.915971819453156\\
41.9545454545455	-0.929481759651869\\
41.9772727272727	-0.942870397139869\\
42	-0.956135984623029\\
42.0227272727273	-0.969276790865962\\
42.0454545454545	-0.982291100917927\\
42.0681818181818	-0.995177216336686\\
42.0909090909091	-1.00793345541013\\
42.1136363636364	-1.02055815337576\\
42.1363636363636	-1.03304966263794\\
42.1590909090909	-1.04540635298294\\
42.1818181818182	-1.05762661179166\\
42.2045454545455	-1.0697088442501\\
42.2272727272727	-1.0816514735575\\
42.25	-1.09345294113211\\
42.2727272727273	-1.10511170681461\\
42.2954545454545	-1.11662624906906\\
42.3181818181818	-1.12799506518154\\
42.3409090909091	-1.13921667145622\\
42.3636363636364	-1.150289603409\\
42.3863636363636	-1.16121241595862\\
42.4090909090909	-1.1719836836153\\
42.4318181818182	-1.18260200066669\\
42.4545454545455	-1.19306598136142\\
42.4772727272727	-1.20337426008986\\
42.5	-1.21352549156242\\
42.5227272727273	-1.22351835098505\\
42.5454545454545	-1.23335153423218\\
42.5681818181818	-1.24302375801688\\
42.5909090909091	-1.25253376005839\\
42.6136363636364	-1.26188029924677\\
42.6363636363636	-1.27106215580496\\
42.6590909090909	-1.28007813144792\\
42.6818181818182	-1.28892704953902\\
42.7045454545455	-1.29760775524359\\
42.7272727272727	-1.30611911567967\\
42.75	-1.3144600200658\\
42.7727272727273	-1.32262937986603\\
42.7954545454545	-1.33062612893197\\
42.8181818181818	-1.33844922364191\\
42.8409090909091	-1.34609764303704\\
42.8636363636364	-1.35357038895467\\
42.8863636363636	-1.36086648615853\\
42.9090909090909	-1.367984982466\\
42.9318181818182	-1.37492494887242\\
42.9545454545455	-1.38168547967227\\
42.9772727272727	-1.38826569257746\\
43	-1.39466472883238\\
43.0227272727273	-1.40088175332602\\
43.0454545454545	-1.40691595470098\\
43.0681818181818	-1.41276654545929\\
43.0909090909091	-1.41843276206523\\
43.1136363636364	-1.423913865045\\
43.1363636363636	-1.42920913908315\\
43.1590909090909	-1.43431789311598\\
43.1818181818182	-1.43923946042174\\
43.2045454545455	-1.44397319870763\\
43.2272727272727	-1.44851849019358\\
43.25	-1.45287474169295\\
43.2727272727273	-1.45704138468987\\
43.2954545454545	-1.4610178754135\\
43.3181818181818	-1.46480369490895\\
43.3409090909091	-1.46839834910501\\
43.3636363636364	-1.47180136887866\\
43.3863636363636	-1.47501231011626\\
43.4090909090909	-1.47803075377154\\
43.4318181818182	-1.48085630592024\\
43.4545454545455	-1.48348859781159\\
43.4772727272727	-1.48592728591635\\
43.5	-1.48817205197172\\
43.5227272727273	-1.49022260302282\\
43.5454545454545	-1.49207867146098\\
43.5681818181818	-1.49374001505858\\
43.5909090909091	-1.49520641700074\\
43.6136363636364	-1.49647768591358\\
43.6363636363636	-1.4975536558892\\
43.6590909090909	-1.49843418650735\\
43.6818181818182	-1.4991191628537\\
43.7045454545455	-1.49960849553492\\
43.7272727272727	-1.49990212069028\\
43.75	-1.5\\
43.7727272727273	-1.49990212069028\\
43.7954545454545	-1.49960849553492\\
43.8181818181818	-1.4991191628537\\
43.8409090909091	-1.49843418650735\\
43.8636363636364	-1.4975536558892\\
43.8863636363636	-1.49647768591358\\
43.9090909090909	-1.49520641700074\\
43.9318181818182	-1.49374001505858\\
43.9545454545455	-1.49207867146098\\
43.9772727272727	-1.49022260302283\\
44	-1.48817205197172\\
44.0227272727273	-1.48592728591635\\
44.0454545454545	-1.48348859781159\\
44.0681818181818	-1.48085630592024\\
44.0909090909091	-1.47803075377154\\
44.1136363636364	-1.47501231011626\\
44.1363636363636	-1.47180136887866\\
44.1590909090909	-1.46839834910501\\
44.1818181818182	-1.46480369490895\\
44.2045454545455	-1.4610178754135\\
44.2272727272727	-1.45704138468987\\
44.25	-1.45287474169295\\
44.2727272727273	-1.44851849019358\\
44.2954545454545	-1.44397319870763\\
44.3181818181818	-1.43923946042175\\
44.3409090909091	-1.43431789311598\\
44.3636363636364	-1.42920913908315\\
44.3863636363636	-1.423913865045\\
44.4090909090909	-1.41843276206524\\
44.4318181818182	-1.41276654545929\\
44.4545454545455	-1.40691595470098\\
44.4772727272727	-1.40088175332603\\
44.5	-1.39466472883238\\
44.5227272727273	-1.38826569257746\\
44.5454545454545	-1.38168547967228\\
44.5681818181818	-1.37492494887242\\
44.5909090909091	-1.36798498246601\\
44.6136363636364	-1.36086648615853\\
44.6363636363636	-1.35357038895468\\
44.6590909090909	-1.34609764303704\\
44.6818181818182	-1.33844922364191\\
44.7045454545455	-1.33062612893197\\
44.7272727272727	-1.32262937986603\\
44.75	-1.31446002006579\\
44.7727272727273	-1.30611911567967\\
44.7954545454545	-1.29760775524359\\
44.8181818181818	-1.28892704953902\\
44.8409090909091	-1.28007813144792\\
44.8636363636364	-1.27106215580497\\
44.8863636363636	-1.26188029924677\\
44.9090909090909	-1.25253376005839\\
44.9318181818182	-1.24302375801689\\
44.9545454545454	-1.23335153423218\\
44.9772727272727	-1.22351835098505\\
45	-1.21352549156242\\
45.0227272727273	-1.20337426008986\\
45.0454545454545	-1.19306598136142\\
45.0681818181818	-1.18260200066669\\
45.0909090909091	-1.17198368361529\\
45.1136363636364	-1.16121241595862\\
45.1363636363636	-1.150289603409\\
45.1590909090909	-1.13921667145622\\
45.1818181818182	-1.12799506518154\\
45.2045454545454	-1.11662624906906\\
45.2272727272727	-1.10511170681461\\
45.25	-1.09345294113212\\
45.2727272727273	-1.08165147355751\\
45.2954545454545	-1.0697088442501\\
45.3181818181818	-1.05762661179166\\
45.3409090909091	-1.04540635298295\\
45.3636363636364	-1.03304966263794\\
45.3863636363636	-1.02055815337576\\
45.4090909090909	-1.00793345541013\\
45.4318181818182	-0.995177216336692\\
45.4545454545454	-0.982291100917933\\
45.4772727272727	-0.96927679086596\\
45.5	-0.956135984623035\\
45.5227272727273	-0.942870397139876\\
45.5454545454545	-0.929481759651875\\
45.5681818181818	-0.915971819453163\\
45.5909090909091	-0.902342339668565\\
45.6136363636364	-0.888595099023505\\
45.6363636363636	-0.8747318916119\\
45.6590909090909	-0.86075452666197\\
45.6818181818182	-0.846664828300159\\
45.7045454545454	-0.832464635313068\\
45.7272727272727	-0.81815580090747\\
45.75	-0.803740192468497\\
45.7727272727273	-0.789219691315876\\
45.7954545454545	-0.774596192458448\\
45.8181818181818	-0.759871604346835\\
45.8409090909091	-0.745047848624426\\
45.8636363636364	-0.730126859876486\\
45.8863636363636	-0.715110585377815\\
45.9090909090909	-0.700000984838515\\
45.9318181818182	-0.684800030148281\\
45.9545454545454	-0.669509705119037\\
45.9772727272727	-0.654132005226084\\
46	-0.638668937347612\\
46.0227272727273	-0.623122519502834\\
46.0454545454545	-0.607494780588615\\
46.0681818181818	-0.591787760114683\\
46.0909090909091	-0.576003507937458\\
46.1136363636364	-0.560144083992569\\
46.1363636363636	-0.544211558025966\\
46.1590909090909	-0.528208009323838\\
46.1818181818182	-0.512135526441254\\
46.2045454545454	-0.495996206929577\\
46.2272727272727	-0.479792157062763\\
46.25	-0.463525491562415\\
46.2727272727273	-0.447198333321883\\
46.2954545454545	-0.430812813129097\\
46.3181818181818	-0.41437106938864\\
46.3409090909091	-0.397875247842508\\
46.3636363636364	-0.381327501290234\\
46.3863636363636	-0.36472998930781\\
46.4090909090909	-0.348084877965955\\
46.4318181818182	-0.331394339547305\\
46.4545454545455	-0.314660552263075\\
46.4772727272727	-0.297885699968621\\
46.5	-0.281071971878592\\
46.5227272727273	-0.264221562281099\\
46.5454545454545	-0.247336670251454\\
46.5681818181818	-0.23041949936505\\
46.5909090909091	-0.21347225740993\\
46.6136363636364	-0.196497156098544\\
46.6363636363636	-0.179496410779155\\
46.6590909090909	-0.16247224014671\\
46.6818181818182	-0.145426865953322\\
46.7045454545455	-0.128362512718262\\
46.7272727272727	-0.111281407437682\\
46.75	-0.0941857792939763\\
46.7727272727273	-0.0770778593648423\\
46.7954545454545	-0.0599598803321595\\
46.8181818181818	-0.0428340761905461\\
46.8409090909091	-0.0257026819558525\\
46.8636363636364	-0.00856793337345775\\
46.8863636363636	0.00856793337345481\\
46.9090909090909	0.0257026819558496\\
46.9318181818182	0.0428340761905431\\
46.9545454545455	0.0599598803321566\\
46.9772727272727	0.0770778593648394\\
47	0.0941857792939734\\
47.0227272727273	0.111281407437669\\
47.0454545454545	0.128362512718259\\
47.0681818181818	0.145426865953319\\
47.0909090909091	0.162472240146707\\
47.1136363636364	0.179496410779142\\
47.1363636363636	0.196497156098541\\
47.1590909090909	0.213472257409927\\
47.1818181818182	0.230419499365047\\
47.2045454545455	0.247336670251441\\
47.2272727272727	0.264221562281096\\
47.25	0.281071971878589\\
47.2727272727273	0.297885699968618\\
47.2954545454545	0.314660552263062\\
47.3181818181818	0.331394339547302\\
47.3409090909091	0.348084877965941\\
47.3636363636364	0.364729989307817\\
47.3863636363636	0.381327501290231\\
47.4090909090909	0.397875247842505\\
47.4318181818182	0.414371069388627\\
47.4545454545455	0.430812813129094\\
47.4772727272727	0.44719833332187\\
47.5	0.463525491562422\\
47.5227272727273	0.479792157062761\\
47.5454545454545	0.495996206929574\\
47.5681818181818	0.512135526441242\\
47.5909090909091	0.528208009323826\\
47.6136363636364	0.544211558025953\\
47.6363636363636	0.560144083992567\\
47.6590909090909	0.576003507937455\\
47.6818181818182	0.59178776011467\\
47.7045454545455	0.607494780588602\\
47.7272727272727	0.623122519502822\\
47.75	0.638668937347599\\
47.7727272727273	0.654132005226081\\
47.7954545454545	0.669509705119035\\
47.8181818181818	0.684800030148268\\
47.8409090909091	0.700000984838513\\
47.8636363636364	0.715110585377803\\
47.8863636363636	0.730126859876483\\
47.9090909090909	0.745047848624414\\
47.9318181818182	0.759871604346842\\
47.9545454545455	0.774596192458436\\
47.9772727272727	0.789219691315874\\
48	0.803740192468485\\
48.0227272727273	0.818155800907468\\
48.0454545454545	0.832464635313056\\
48.0681818181818	0.846664828300157\\
48.0909090909091	0.860754526661959\\
48.1136363636364	0.874731891611897\\
48.1363636363636	0.888595099023503\\
48.1590909090909	0.902342339668554\\
48.1818181818182	0.915971819453152\\
48.2045454545455	0.929481759651873\\
48.2272727272727	0.942870397139865\\
48.25	0.956135984623033\\
48.2727272727273	0.969276790865958\\
48.2954545454545	0.982291100917923\\
48.3181818181818	0.995177216336689\\
48.3409090909091	1.00793345541013\\
48.3636363636364	1.02055815337576\\
48.3863636363636	1.03304966263794\\
48.4090909090909	1.04540635298294\\
48.4318181818182	1.05762661179165\\
48.4545454545455	1.0697088442501\\
48.4772727272727	1.0816514735575\\
48.5	1.09345294113212\\
48.5227272727273	1.1051117068146\\
48.5454545454545	1.11662624906906\\
48.5681818181818	1.12799506518153\\
48.5909090909091	1.13921667145622\\
48.6136363636364	1.15028960340899\\
48.6363636363636	1.16121241595862\\
48.6590909090909	1.17198368361529\\
48.6818181818182	1.18260200066669\\
48.7045454545455	1.19306598136141\\
48.7272727272727	1.20337426008986\\
48.75	1.21352549156242\\
48.7727272727273	1.22351835098506\\
48.7954545454545	1.23335153423218\\
48.8181818181818	1.24302375801689\\
48.8409090909091	1.25253376005839\\
48.8636363636364	1.26188029924677\\
48.8863636363636	1.27106215580496\\
48.9090909090909	1.28007813144792\\
48.9318181818182	1.28892704953902\\
48.9545454545455	1.29760775524359\\
48.9772727272727	1.30611911567966\\
49	1.31446002006579\\
49.0227272727273	1.32262937986603\\
49.0454545454545	1.33062612893197\\
49.0681818181818	1.33844922364191\\
49.0909090909091	1.34609764303704\\
49.1136363636364	1.35357038895467\\
49.1363636363636	1.36086648615853\\
49.1590909090909	1.36798498246601\\
49.1818181818182	1.37492494887242\\
49.2045454545455	1.38168547967227\\
49.2272727272727	1.38826569257746\\
49.25	1.39466472883237\\
49.2727272727273	1.40088175332602\\
49.2954545454545	1.40691595470098\\
49.3181818181818	1.41276654545929\\
49.3409090909091	1.41843276206524\\
49.3636363636364	1.423913865045\\
49.3863636363636	1.42920913908314\\
49.4090909090909	1.43431789311598\\
49.4318181818182	1.43923946042175\\
49.4545454545455	1.44397319870763\\
49.4772727272727	1.44851849019358\\
49.5	1.45287474169294\\
49.5227272727273	1.45704138468987\\
49.5454545454545	1.4610178754135\\
49.5681818181818	1.46480369490894\\
49.5909090909091	1.46839834910501\\
49.6136363636364	1.47180136887866\\
49.6363636363636	1.47501231011626\\
49.6590909090909	1.47803075377154\\
49.6818181818182	1.48085630592024\\
49.7045454545455	1.48348859781159\\
49.7272727272727	1.48592728591635\\
49.75	1.48817205197172\\
49.7727272727273	1.49022260302282\\
49.7954545454545	1.49207867146098\\
49.8181818181818	1.49374001505858\\
49.8409090909091	1.49520641700074\\
49.8636363636364	1.49647768591358\\
49.8863636363636	1.4975536558892\\
49.9090909090909	1.49843418650735\\
49.9318181818182	1.4991191628537\\
49.9545454545455	1.49960849553492\\
49.9772727272727	1.49990212069028\\
50	1.5\\
50.0227272727273	1.49990212069028\\
50.0454545454546	1.49960849553492\\
50.0681818181818	1.4991191628537\\
50.0909090909091	1.49843418650735\\
50.1136363636364	1.4975536558892\\
50.1363636363636	1.49647768591358\\
50.1590909090909	1.49520641700074\\
50.1818181818182	1.49374001505858\\
50.2045454545455	1.49207867146098\\
50.2272727272727	1.49022260302282\\
50.25	1.48817205197172\\
50.2727272727273	1.48592728591635\\
50.2954545454546	1.48348859781159\\
50.3181818181818	1.48085630592024\\
50.3409090909091	1.47803075377154\\
50.3636363636364	1.47501231011626\\
50.3863636363636	1.47180136887866\\
50.4090909090909	1.46839834910501\\
50.4318181818182	1.46480369490894\\
50.4545454545455	1.4610178754135\\
50.4772727272727	1.45704138468987\\
50.5	1.45287474169294\\
50.5227272727273	1.44851849019358\\
50.5454545454546	1.44397319870763\\
50.5681818181818	1.43923946042175\\
50.5909090909091	1.43431789311598\\
50.6136363636364	1.42920913908314\\
50.6363636363636	1.423913865045\\
50.6590909090909	1.41843276206524\\
50.6818181818182	1.41276654545929\\
50.7045454545455	1.40691595470098\\
50.7272727272727	1.40088175332602\\
50.75	1.39466472883237\\
50.7727272727273	1.38826569257746\\
50.7954545454546	1.38168547967227\\
50.8181818181818	1.37492494887242\\
50.8409090909091	1.367984982466\\
50.8636363636364	1.36086648615853\\
50.8863636363636	1.35357038895467\\
50.9090909090909	1.34609764303704\\
50.9318181818182	1.33844922364191\\
50.9545454545455	1.33062612893197\\
50.9772727272727	1.32262937986602\\
51	1.31446002006579\\
51.0227272727273	1.30611911567966\\
51.0454545454546	1.29760775524359\\
51.0681818181818	1.28892704953902\\
51.0909090909091	1.28007813144792\\
51.1136363636364	1.27106215580496\\
51.1363636363636	1.26188029924677\\
51.1590909090909	1.25253376005839\\
51.1818181818182	1.24302375801688\\
51.2045454545455	1.23335153423218\\
51.2272727272727	1.22351835098505\\
51.25	1.21352549156242\\
51.2727272727273	1.20337426008986\\
51.2954545454546	1.19306598136142\\
51.3181818181818	1.18260200066668\\
51.3409090909091	1.1719836836153\\
51.3636363636364	1.16121241595862\\
51.3863636363636	1.15028960340899\\
51.4090909090909	1.13921667145622\\
51.4318181818182	1.12799506518154\\
51.4545454545455	1.11662624906905\\
51.4772727272727	1.10511170681461\\
51.5	1.09345294113211\\
51.5227272727273	1.0816514735575\\
51.5454545454546	1.0697088442501\\
51.5681818181818	1.05762661179165\\
51.5909090909091	1.04540635298293\\
51.6136363636364	1.03304966263794\\
51.6363636363636	1.02055815337576\\
51.6590909090909	1.00793345541012\\
51.6818181818182	0.995177216336688\\
51.7045454545455	0.982291100917921\\
51.7272727272727	0.969276790865957\\
51.75	0.956135984623023\\
51.7727272727273	0.942870397139871\\
51.7954545454546	0.929481759651863\\
51.8181818181818	0.915971819453159\\
51.8409090909091	0.902342339668553\\
51.8636363636364	0.888595099023501\\
51.8863636363636	0.874731891611887\\
51.9090909090909	0.860754526661966\\
51.9318181818182	0.846664828300146\\
51.9545454545455	0.832464635313064\\
51.9772727272727	0.818155800907466\\
52	0.803740192468492\\
52.0227272727273	0.789219691315863\\
52.0454545454546	0.774596192458435\\
52.0681818181818	0.759871604346831\\
52.0909090909091	0.745047848624412\\
52.1136363636364	0.730126859876481\\
52.1363636363636	0.715110585377801\\
52.1590909090909	0.700000984838501\\
52.1818181818182	0.684800030148267\\
52.2045454545455	0.669509705119033\\
52.2272727272727	0.654132005226079\\
52.25	0.638668937347607\\
52.2727272727273	0.62312251950282\\
52.2954545454546	0.6074947805886\\
52.3181818181818	0.591787760114668\\
52.3409090909091	0.576003507937453\\
52.3636363636364	0.560144083992565\\
52.3863636363636	0.544211558025961\\
52.4090909090909	0.528208009323824\\
52.4318181818182	0.512135526441239\\
52.4545454545455	0.495996206929572\\
52.4772727272727	0.479792157062759\\
52.5	0.46352549156242\\
52.5227272727273	0.447198333321868\\
52.5454545454546	0.430812813129092\\
52.5681818181818	0.414371069388625\\
52.5909090909091	0.397875247842503\\
52.6136363636364	0.381327501290229\\
52.6363636363636	0.364729989307805\\
52.6590909090909	0.348084877965939\\
52.6818181818182	0.3313943395473\\
52.7045454545455	0.31466055226306\\
52.7272727272727	0.297885699968616\\
52.75	0.281071971878587\\
52.7727272727273	0.264221562281094\\
52.7954545454546	0.247336670251438\\
52.8181818181818	0.230419499365045\\
52.8409090909091	0.213472257409925\\
52.8636363636364	0.196497156098528\\
52.8863636363636	0.17949641077914\\
52.9090909090909	0.162472240146705\\
52.9318181818182	0.145426865953317\\
52.9545454545455	0.128362512718257\\
52.9772727272727	0.111281407437677\\
53	0.0941857792939605\\
53.0227272727273	0.0770778593648371\\
53.0454545454546	0.0599598803321544\\
53.0681818181818	0.0428340761905409\\
53.0909090909091	0.0257026819558473\\
53.1136363636364	0.00856793337346326\\
53.1363636363636	-0.00856793337347061\\
53.1590909090909	-0.0257026819558547\\
53.1818181818182	-0.0428340761905483\\
53.2045454545455	-0.0599598803321617\\
53.2272727272727	-0.0770778593648445\\
53.25	-0.0941857792939785\\
53.2727272727273	-0.111281407437685\\
53.2954545454546	-0.128362512718264\\
53.3181818181818	-0.145426865953325\\
53.3409090909091	-0.162472240146723\\
53.3636363636364	-0.179496410779147\\
53.3863636363636	-0.196497156098546\\
53.4090909090909	-0.213472257409933\\
53.4318181818182	-0.230419499365063\\
53.4545454545455	-0.247336670251446\\
53.4772727272727	-0.264221562281112\\
53.5	-0.281071971878584\\
53.5227272727273	-0.297885699968623\\
53.5454545454546	-0.314660552263067\\
53.5681818181818	-0.331394339547318\\
53.5909090909091	-0.348084877965946\\
53.6136363636364	-0.364729989307822\\
53.6363636363636	-0.381327501290226\\
53.6590909090909	-0.39787524784251\\
53.6818181818182	-0.414371069388632\\
53.7045454545455	-0.430812813129109\\
53.7272727272727	-0.447198333321885\\
53.75	-0.463525491562427\\
53.7727272727273	-0.479792157062766\\
53.7954545454546	-0.495996206929579\\
53.8181818181818	-0.512135526441257\\
53.8409090909091	-0.528208009323841\\
53.8636363636364	-0.544211558025968\\
53.8863636363636	-0.560144083992571\\
53.9090909090909	-0.576003507937469\\
53.9318181818182	-0.591787760114675\\
53.9545454545455	-0.607494780588617\\
53.9772727272727	-0.623122519502836\\
54	-0.638668937347614\\
54.0227272727273	-0.654132005226086\\
54.0454545454546	-0.669509705119049\\
54.0681818181818	-0.684800030148273\\
54.0909090909091	-0.700000984838517\\
54.1136363636364	-0.715110585377817\\
54.1363636363636	-0.730126859876488\\
54.1590909090909	-0.745047848624418\\
54.1818181818182	-0.759871604346846\\
54.2045454545455	-0.77459619245845\\
54.2272727272727	-0.789219691315878\\
54.25	-0.803740192468498\\
54.2727272727273	-0.818155800907472\\
54.2954545454546	-0.83246463531307\\
54.3181818181818	-0.846664828300161\\
54.3409090909091	-0.860754526661972\\
54.3636363636364	-0.874731891611902\\
54.3863636363636	-0.888595099023507\\
54.4090909090909	-0.902342339668567\\
54.4318181818182	-0.915971819453165\\
54.4545454545455	-0.929481759651877\\
54.4772727272727	-0.942870397139877\\
54.5	-0.956135984623037\\
54.5227272727273	-0.969276790865962\\
54.5454545454546	-0.982291100917935\\
54.5681818181818	-0.995177216336693\\
54.5909090909091	-1.00793345541014\\
54.6136363636364	-1.02055815337576\\
54.6363636363636	-1.03304966263794\\
54.6590909090909	-1.04540635298294\\
54.6818181818182	-1.05762661179166\\
54.7045454545455	-1.06970884425011\\
54.7272727272727	-1.08165147355751\\
54.75	-1.09345294113212\\
54.7727272727273	-1.10511170681461\\
54.7954545454546	-1.11662624906906\\
54.8181818181818	-1.12799506518154\\
54.8409090909091	-1.13921667145622\\
54.8636363636364	-1.150289603409\\
54.8863636363636	-1.16121241595862\\
54.9090909090909	-1.1719836836153\\
54.9318181818182	-1.1826020006667\\
54.9545454545455	-1.19306598136142\\
54.9772727272727	-1.20337426008986\\
55	-1.21352549156242\\
55.0227272727273	-1.22351835098506\\
55.0454545454546	-1.23335153423219\\
55.0681818181818	-1.24302375801689\\
55.0909090909091	-1.25253376005839\\
55.1136363636364	-1.26188029924678\\
55.1363636363636	-1.27106215580496\\
55.1590909090909	-1.28007813144793\\
55.1818181818182	-1.28892704953902\\
55.2045454545455	-1.29760775524359\\
55.2272727272727	-1.30611911567967\\
55.25	-1.3144600200658\\
55.2727272727273	-1.32262937986603\\
55.2954545454546	-1.33062612893197\\
55.3181818181818	-1.33844922364191\\
55.3409090909091	-1.34609764303704\\
55.3636363636364	-1.35357038895467\\
55.3863636363636	-1.36086648615854\\
55.4090909090909	-1.36798498246601\\
55.4318181818182	-1.37492494887242\\
55.4545454545455	-1.38168547967228\\
55.4772727272727	-1.38826569257746\\
55.5	-1.39466472883238\\
55.5227272727273	-1.40088175332603\\
55.5454545454546	-1.40691595470098\\
55.5681818181818	-1.41276654545929\\
55.5909090909091	-1.41843276206524\\
55.6136363636364	-1.423913865045\\
55.6363636363636	-1.42920913908315\\
55.6590909090909	-1.43431789311598\\
55.6818181818182	-1.43923946042175\\
55.7045454545455	-1.44397319870763\\
55.7272727272727	-1.44851849019358\\
55.75	-1.45287474169295\\
55.7727272727273	-1.45704138468987\\
55.7954545454546	-1.4610178754135\\
55.8181818181818	-1.46480369490895\\
55.8409090909091	-1.46839834910501\\
55.8636363636364	-1.47180136887866\\
55.8863636363636	-1.47501231011627\\
55.9090909090909	-1.47803075377154\\
55.9318181818182	-1.48085630592024\\
55.9545454545455	-1.48348859781159\\
55.9772727272727	-1.48592728591635\\
56	-1.48817205197172\\
56.0227272727273	-1.49022260302283\\
56.0454545454546	-1.49207867146098\\
56.0681818181818	-1.49374001505858\\
56.0909090909091	-1.49520641700074\\
56.1136363636364	-1.49647768591358\\
56.1363636363636	-1.4975536558892\\
56.1590909090909	-1.49843418650735\\
56.1818181818182	-1.4991191628537\\
56.2045454545455	-1.49960849553492\\
56.2272727272727	-1.49990212069028\\
56.25	-1.5\\
56.2727272727273	-1.49990212069028\\
56.2954545454546	-1.49960849553492\\
56.3181818181818	-1.4991191628537\\
56.3409090909091	-1.49843418650735\\
56.3636363636364	-1.4975536558892\\
56.3863636363636	-1.49647768591358\\
56.4090909090909	-1.49520641700074\\
56.4318181818182	-1.49374001505858\\
56.4545454545455	-1.49207867146098\\
56.4772727272727	-1.49022260302282\\
56.5	-1.48817205197172\\
56.5227272727273	-1.48592728591635\\
56.5454545454546	-1.48348859781159\\
56.5681818181818	-1.48085630592024\\
56.5909090909091	-1.47803075377154\\
56.6136363636364	-1.47501231011626\\
56.6363636363636	-1.47180136887866\\
56.6590909090909	-1.46839834910501\\
56.6818181818182	-1.46480369490895\\
56.7045454545455	-1.4610178754135\\
56.7272727272727	-1.45704138468987\\
56.75	-1.45287474169295\\
56.7727272727273	-1.44851849019358\\
56.7954545454546	-1.44397319870763\\
56.8181818181818	-1.43923946042175\\
56.8409090909091	-1.43431789311598\\
56.8636363636364	-1.42920913908315\\
56.8863636363636	-1.423913865045\\
56.9090909090909	-1.41843276206524\\
56.9318181818182	-1.41276654545928\\
56.9545454545455	-1.40691595470098\\
56.9772727272727	-1.40088175332602\\
57	-1.39466472883238\\
57.0227272727273	-1.38826569257746\\
57.0454545454546	-1.38168547967228\\
57.0681818181818	-1.37492494887241\\
57.0909090909091	-1.367984982466\\
57.1136363636364	-1.36086648615853\\
57.1363636363636	-1.35357038895467\\
57.1590909090909	-1.34609764303704\\
57.1818181818182	-1.33844922364191\\
57.2045454545455	-1.33062612893197\\
57.2272727272727	-1.32262937986602\\
57.25	-1.31446002006579\\
57.2727272727273	-1.30611911567966\\
57.2954545454546	-1.29760775524359\\
57.3181818181818	-1.28892704953902\\
57.3409090909091	-1.28007813144792\\
57.3636363636364	-1.27106215580496\\
57.3863636363636	-1.26188029924677\\
57.4090909090909	-1.25253376005838\\
57.4318181818182	-1.24302375801689\\
57.4545454545455	-1.23335153423218\\
57.4772727272727	-1.22351835098505\\
57.5	-1.21352549156241\\
57.5227272727273	-1.20337426008986\\
57.5454545454546	-1.19306598136141\\
57.5681818181818	-1.18260200066669\\
57.5909090909091	-1.17198368361529\\
57.6136363636364	-1.16121241595862\\
57.6363636363636	-1.15028960340899\\
57.6590909090909	-1.13921667145621\\
57.6818181818182	-1.12799506518154\\
57.7045454545455	-1.11662624906906\\
57.7272727272727	-1.10511170681461\\
57.75	-1.09345294113211\\
57.7727272727273	-1.0816514735575\\
57.7954545454546	-1.0697088442501\\
57.8181818181818	-1.05762661179166\\
57.8409090909091	-1.04540635298294\\
57.8636363636364	-1.03304966263794\\
57.8863636363636	-1.02055815337575\\
57.9090909090909	-1.00793345541013\\
57.9318181818182	-0.995177216336684\\
57.9545454545455	-0.982291100917925\\
57.9772727272727	-0.969276790865953\\
58	-0.956135984623036\\
58.0227272727273	-0.942870397139867\\
58.0454545454546	-0.929481759651867\\
58.0681818181818	-0.915971819453155\\
58.0909090909091	-0.902342339668557\\
58.1136363636364	-0.888595099023497\\
58.1363636363636	-0.874731891611891\\
58.1590909090909	-0.860754526661961\\
58.1818181818182	-0.846664828300151\\
58.2045454545455	-0.832464635313059\\
58.2272727272727	-0.818155800907462\\
58.25	-0.803740192468488\\
58.2727272727273	-0.789219691315867\\
58.2954545454546	-0.774596192458439\\
58.3181818181818	-0.759871604346836\\
58.3409090909091	-0.745047848624417\\
58.3636363636364	-0.730126859876477\\
58.3863636363636	-0.715110585377806\\
58.4090909090909	-0.700000984838506\\
58.4318181818182	-0.684800030148271\\
58.4545454545455	-0.669509705119038\\
58.4772727272727	-0.654132005226075\\
58.5	-0.638668937347602\\
58.5227272727273	-0.623122519502825\\
58.5454545454546	-0.607494780588605\\
58.5681818181818	-0.591787760114673\\
58.5909090909091	-0.576003507937458\\
58.6136363636364	-0.56014408399256\\
58.6363636363636	-0.544211558025956\\
58.6590909090909	-0.528208009323829\\
58.6818181818182	-0.512135526441245\\
58.7045454545455	-0.495996206929577\\
58.7272727272727	-0.479792157062764\\
58.75	-0.463525491562415\\
58.7727272727273	-0.447198333321873\\
58.7954545454546	-0.430812813129097\\
58.8181818181818	-0.41437106938863\\
58.8409090909091	-0.397875247842509\\
58.8636363636364	-0.381327501290224\\
58.8863636363636	-0.36472998930781\\
58.9090909090909	-0.348084877965934\\
58.9318181818182	-0.331394339547306\\
58.9545454545455	-0.314660552263055\\
58.9772727272727	-0.297885699968622\\
59	-0.281071971878582\\
59.0227272727273	-0.264221562281099\\
59.0454545454546	-0.247336670251433\\
59.0681818181818	-0.23041949936505\\
59.0909090909091	-0.21347225740992\\
59.1136363636364	-0.196497156098544\\
59.1363636363636	-0.179496410779145\\
59.1590909090909	-0.16247224014671\\
59.1818181818182	-0.145426865953312\\
59.2045454545455	-0.128362512718262\\
59.2272727272727	-0.111281407437672\\
59.25	-0.094185779293966\\
59.2727272727273	-0.0770778593648426\\
59.2954545454546	-0.0599598803321599\\
59.3181818181818	-0.0428340761905358\\
59.3409090909091	-0.0257026819558422\\
59.3636363636364	-0.00856793337345811\\
59.3863636363636	0.00856793337347576\\
59.4090909090909	0.0257026819558492\\
59.4318181818182	0.0428340761905534\\
59.4545454545455	0.0599598803321668\\
59.4772727272727	0.0770778593648496\\
59.5	0.094185779293973\\
59.5227272727273	0.11128140743769\\
59.5454545454546	0.128362512718258\\
59.5681818181818	0.14542686595333\\
59.5909090909091	0.162472240146717\\
59.6136363636364	0.179496410779152\\
59.6363636363636	0.196497156098541\\
59.6590909090909	0.213472257409938\\
59.6818181818182	0.230419499365047\\
59.7045454545455	0.247336670251451\\
59.7272727272727	0.264221562281106\\
59.75	0.281071971878599\\
59.7727272727273	0.297885699968618\\
59.7954545454546	0.314660552263072\\
59.8181818181818	0.331394339547312\\
59.8409090909091	0.348084877965951\\
59.8636363636364	0.364729989307817\\
59.8863636363636	0.381327501290241\\
59.9090909090909	0.397875247842515\\
59.9318181818182	0.414371069388637\\
59.9545454545455	0.430812813129104\\
59.9772727272727	0.447198333321879\\
60	0.463525491562422\\
60.0227272727273	0.47979215706277\\
60.0454545454546	0.495996206929584\\
60.0681818181818	0.512135526441251\\
60.0909090909091	0.528208009323835\\
60.1136363636364	0.544211558025963\\
60.1363636363636	0.560144083992576\\
60.1590909090909	0.576003507937464\\
60.1818181818182	0.59178776011468\\
60.2045454545455	0.607494780588612\\
60.2272727272727	0.623122519502831\\
60.25	0.638668937347609\\
60.2727272727273	0.65413200522609\\
60.2954545454546	0.669509705119044\\
60.3181818181818	0.684800030148278\\
60.3409090909091	0.700000984838513\\
60.3636363636364	0.715110585377812\\
60.3863636363636	0.730126859876492\\
60.4090909090909	0.745047848624423\\
60.4318181818182	0.759871604346842\\
60.4545454545455	0.774596192458445\\
60.4772727272727	0.789219691315873\\
60.5	0.803740192468503\\
60.5227272727273	0.818155800907476\\
60.5454545454546	0.832464635313065\\
60.5681818181818	0.846664828300156\\
60.5909090909091	0.860754526661976\\
60.6136363636364	0.874731891611897\\
60.6363636363636	0.888595099023511\\
60.6590909090909	0.902342339668563\\
60.6818181818182	0.91597181945316\\
60.7045454545455	0.929481759651873\\
60.7272727272727	0.942870397139881\\
60.75	0.956135984623033\\
60.7727272727273	0.969276790865966\\
60.7954545454546	0.982291100917931\\
60.8181818181818	0.995177216336689\\
60.8409090909091	1.00793345541013\\
60.8636363636364	1.02055815337577\\
60.8863636363636	1.03304966263794\\
60.9090909090909	1.04540635298294\\
60.9318181818182	1.05762661179166\\
60.9545454545455	1.0697088442501\\
60.9772727272727	1.08165147355751\\
61	1.09345294113212\\
61.0227272727273	1.10511170681461\\
61.0454545454546	1.11662624906906\\
61.0681818181818	1.12799506518155\\
61.0909090909091	1.13921667145622\\
61.1136363636364	1.150289603409\\
61.1363636363636	1.16121241595863\\
61.1590909090909	1.17198368361531\\
61.1818181818182	1.18260200066669\\
61.2045454545455	1.19306598136142\\
61.2272727272727	1.20337426008986\\
61.25	1.21352549156242\\
61.2727272727273	1.22351835098506\\
61.2954545454546	1.23335153423219\\
61.3181818181818	1.24302375801689\\
61.3409090909091	1.25253376005839\\
61.3636363636364	1.26188029924677\\
61.3863636363636	1.27106215580497\\
61.4090909090909	1.28007813144792\\
61.4318181818182	1.28892704953902\\
61.4545454545455	1.2976077552436\\
61.4772727272727	1.30611911567967\\
61.5	1.3144600200658\\
61.5227272727273	1.32262937986603\\
61.5454545454546	1.33062612893197\\
61.5681818181818	1.33844922364191\\
61.5909090909091	1.34609764303704\\
61.6136363636364	1.35357038895468\\
61.6363636363636	1.36086648615853\\
61.6590909090909	1.36798498246601\\
61.6818181818182	1.37492494887242\\
61.7045454545455	1.38168547967228\\
61.7272727272727	1.38826569257746\\
61.75	1.39466472883238\\
61.7727272727273	1.40088175332602\\
61.7954545454546	1.40691595470098\\
61.8181818181818	1.41276654545929\\
61.8409090909091	1.41843276206524\\
61.8636363636364	1.423913865045\\
61.8863636363636	1.42920913908315\\
61.9090909090909	1.43431789311598\\
61.9318181818182	1.43923946042175\\
61.9545454545455	1.44397319870763\\
61.9772727272727	1.44851849019358\\
62	1.45287474169295\\
62.0227272727273	1.45704138468987\\
62.0454545454546	1.4610178754135\\
62.0681818181818	1.46480369490895\\
62.0909090909091	1.46839834910501\\
62.1136363636364	1.47180136887866\\
62.1363636363636	1.47501231011626\\
62.1590909090909	1.47803075377154\\
62.1818181818182	1.48085630592025\\
62.2045454545455	1.48348859781159\\
62.2272727272727	1.48592728591635\\
62.25	1.48817205197172\\
62.2727272727273	1.49022260302283\\
62.2954545454546	1.49207867146098\\
62.3181818181818	1.49374001505858\\
62.3409090909091	1.49520641700074\\
62.3636363636364	1.49647768591358\\
62.3863636363636	1.4975536558892\\
62.4090909090909	1.49843418650735\\
62.4318181818182	1.4991191628537\\
62.4545454545455	1.49960849553492\\
62.4772727272727	1.49990212069028\\
62.5	1.5\\
62.5227272727273	1.49990212069028\\
62.5454545454545	1.49960849553492\\
62.5681818181818	1.4991191628537\\
62.5909090909091	1.49843418650735\\
62.6136363636364	1.4975536558892\\
62.6363636363636	1.49647768591358\\
62.6590909090909	1.49520641700074\\
62.6818181818182	1.49374001505858\\
62.7045454545455	1.49207867146098\\
62.7272727272727	1.49022260302282\\
62.75	1.48817205197172\\
62.7727272727273	1.48592728591635\\
62.7954545454545	1.48348859781159\\
62.8181818181818	1.48085630592024\\
62.8409090909091	1.47803075377154\\
62.8636363636364	1.47501231011626\\
62.8863636363636	1.47180136887866\\
62.9090909090909	1.46839834910501\\
62.9318181818182	1.46480369490894\\
62.9545454545455	1.4610178754135\\
62.9772727272727	1.45704138468987\\
63	1.45287474169295\\
63.0227272727273	1.44851849019358\\
63.0454545454545	1.44397319870763\\
63.0681818181818	1.43923946042175\\
63.0909090909091	1.43431789311598\\
63.1136363636364	1.42920913908314\\
63.1363636363636	1.423913865045\\
63.1590909090909	1.41843276206524\\
63.1818181818182	1.41276654545929\\
63.2045454545455	1.40691595470098\\
63.2272727272727	1.40088175332602\\
63.25	1.39466472883238\\
63.2727272727273	1.38826569257746\\
63.2954545454545	1.38168547967228\\
63.3181818181818	1.37492494887241\\
63.3409090909091	1.367984982466\\
63.3636363636364	1.36086648615853\\
63.3863636363636	1.35357038895467\\
63.4090909090909	1.34609764303704\\
63.4318181818182	1.33844922364191\\
63.4545454545455	1.33062612893197\\
63.4772727272727	1.32262937986602\\
63.5	1.3144600200658\\
63.5227272727273	1.30611911567966\\
63.5454545454545	1.2976077552436\\
63.5681818181818	1.28892704953902\\
63.5909090909091	1.28007813144792\\
63.6136363636364	1.27106215580496\\
63.6363636363636	1.26188029924677\\
63.6590909090909	1.25253376005839\\
63.6818181818182	1.24302375801689\\
63.7045454545455	1.23335153423218\\
63.7272727272727	1.22351835098505\\
63.75	1.21352549156242\\
63.7727272727273	1.20337426008985\\
63.7954545454545	1.19306598136142\\
63.8181818181818	1.18260200066668\\
63.8409090909091	1.17198368361529\\
63.8636363636364	1.16121241595862\\
63.8863636363636	1.15028960340899\\
63.9090909090909	1.13921667145622\\
63.9318181818182	1.12799506518154\\
63.9545454545455	1.11662624906905\\
63.9772727272727	1.1051117068146\\
64	1.09345294113211\\
64.0227272727273	1.08165147355749\\
64.0454545454545	1.0697088442501\\
64.0681818181818	1.05762661179165\\
64.0909090909091	1.04540635298293\\
64.1136363636364	1.03304966263794\\
64.1363636363636	1.02055815337575\\
64.1590909090909	1.00793345541013\\
64.1818181818182	0.99517721633668\\
64.2045454545455	0.98229110091793\\
64.2272727272727	0.969276790865949\\
64.25	0.95613598462304\\
64.2727272727273	0.942870397139863\\
64.2954545454545	0.929481759651872\\
64.3181818181818	0.915971819453151\\
64.3409090909091	0.902342339668553\\
64.3636363636364	0.888595099023501\\
64.3863636363636	0.874731891611896\\
64.4090909090909	0.860754526661957\\
64.4318181818182	0.846664828300146\\
64.4545454545455	0.832464635313055\\
64.4772727272727	0.818155800907457\\
64.5	0.803740192468492\\
64.5227272727273	0.789219691315863\\
64.5454545454545	0.774596192458444\\
64.5681818181818	0.759871604346831\\
64.5909090909091	0.745047848624412\\
64.6136363636364	0.730126859876472\\
64.6363636363636	0.715110585377811\\
64.6590909090909	0.700000984838502\\
64.6818181818182	0.684800030148267\\
64.7045454545455	0.669509705119033\\
64.7272727272727	0.65413200522607\\
64.75	0.638668937347607\\
64.7727272727273	0.62312251950282\\
64.7954545454545	0.60749478058861\\
64.8181818181818	0.591787760114668\\
64.8409090909091	0.576003507937453\\
64.8636363636364	0.560144083992565\\
64.8863636363636	0.544211558025951\\
64.9090909090909	0.528208009323824\\
64.9318181818182	0.51213552644124\\
64.9545454545455	0.495996206929572\\
64.9772727272727	0.479792157062749\\
65	0.463525491562421\\
65.0227272727273	0.447198333321868\\
65.0454545454545	0.430812813129103\\
65.0681818181818	0.414371069388625\\
65.0909090909091	0.397875247842493\\
65.1136363636364	0.381327501290229\\
65.1363636363636	0.364729989307805\\
65.1590909090909	0.34808487796594\\
65.1818181818182	0.331394339547301\\
65.2045454545455	0.31466055226306\\
65.2272727272727	0.297885699968606\\
65.25	0.281071971878577\\
65.2727272727273	0.264221562281084\\
65.2954545454545	0.247336670251439\\
65.3181818181818	0.230419499365035\\
65.3409090909091	0.213472257409926\\
65.3636363636364	0.196497156098529\\
65.3863636363636	0.17949641077914\\
65.4090909090909	0.162472240146705\\
65.4318181818182	0.145426865953318\\
65.4545454545455	0.128362512718257\\
65.4772727272727	0.111281407437667\\
65.5	0.0941857792939715\\
65.5227272727273	0.0770778593648269\\
65.5454545454545	0.0599598803321547\\
65.5681818181818	0.0428340761905307\\
65.5909090909091	0.0257026819558477\\
65.6136363636364	0.00856793337345297\\
65.6363636363636	-0.00856793337347024\\
65.6590909090909	-0.0257026819558543\\
65.6818181818182	-0.0428340761905479\\
65.7045454545455	-0.0599598803321613\\
65.7272727272727	-0.0770778593648548\\
65.75	-0.0941857792939675\\
65.7727272727273	-0.111281407437684\\
65.7954545454545	-0.128362512718264\\
65.8181818181818	-0.145426865953335\\
65.8409090909091	-0.162472240146712\\
65.8636363636364	-0.179496410779157\\
65.8863636363636	-0.196497156098546\\
65.9090909090909	-0.213472257409932\\
65.9318181818182	-0.230419499365052\\
65.9545454545455	-0.247336670251445\\
65.9772727272727	-0.264221562281101\\
66	-0.281071971878594\\
66.0227272727273	-0.297885699968634\\
66.0454545454545	-0.314660552263067\\
66.0681818181818	-0.331394339547317\\
66.0909090909091	-0.348084877965946\\
66.1136363636364	-0.364729989307822\\
66.1363636363636	-0.381327501290236\\
66.1590909090909	-0.39787524784252\\
66.1818181818182	-0.414371069388632\\
66.2045454545455	-0.430812813129109\\
66.2272727272727	-0.447198333321874\\
66.25	-0.463525491562427\\
66.2727272727273	-0.479792157062775\\
66.2954545454545	-0.495996206929579\\
66.3181818181818	-0.512135526441256\\
66.3409090909091	-0.52820800932383\\
66.3636363636364	-0.544211558025967\\
66.3863636363636	-0.560144083992581\\
66.4090909090909	-0.576003507937459\\
66.4318181818182	-0.591787760114684\\
66.4545454545455	-0.607494780588607\\
66.4772727272727	-0.623122519502836\\
66.5	-0.638668937347613\\
66.5227272727273	-0.654132005226095\\
66.5454545454545	-0.669509705119039\\
66.5681818181818	-0.684800030148282\\
66.5909090909091	-0.700000984838508\\
66.6136363636364	-0.715110585377817\\
66.6363636363636	-0.730126859876497\\
66.6590909090909	-0.745047848624427\\
66.6818181818182	-0.759871604346846\\
66.7045454545455	-0.77459619245845\\
66.7272727272727	-0.789219691315878\\
66.75	-0.803740192468498\\
66.7727272727273	-0.818155800907481\\
66.7954545454545	-0.83246463531306\\
66.8181818181818	-0.846664828300161\\
66.8409090909091	-0.860754526661971\\
66.8636363636364	-0.87473189161191\\
66.8863636363636	-0.888595099023507\\
66.9090909090909	-0.902342339668567\\
66.9318181818182	-0.915971819453164\\
66.9545454545455	-0.929481759651877\\
66.9772727272727	-0.942870397139877\\
67	-0.956135984623037\\
67.0227272727273	-0.96927679086597\\
67.0454545454545	-0.982291100917927\\
67.0681818181818	-0.995177216336693\\
67.0909090909091	-1.00793345541013\\
67.1136363636364	-1.02055815337577\\
67.1363636363636	-1.03304966263795\\
67.1590909090909	-1.04540635298295\\
67.1818181818182	-1.05762661179166\\
67.2045454545455	-1.0697088442501\\
67.2272727272727	-1.08165147355751\\
67.25	-1.09345294113212\\
67.2727272727273	-1.10511170681461\\
67.2954545454545	-1.11662624906906\\
67.3181818181818	-1.12799506518154\\
67.3409090909091	-1.13921667145621\\
67.3636363636364	-1.150289603409\\
67.3863636363636	-1.16121241595863\\
67.4090909090909	-1.1719836836153\\
67.4318181818182	-1.1826020006667\\
67.4545454545455	-1.19306598136142\\
67.4772727272727	-1.20337426008986\\
67.5	-1.21352549156243\\
67.5227272727273	-1.22351835098506\\
67.5454545454545	-1.23335153423218\\
67.5681818181818	-1.2430237580169\\
67.5909090909091	-1.25253376005839\\
67.6136363636364	-1.26188029924678\\
67.6363636363636	-1.27106215580497\\
67.6590909090909	-1.28007813144793\\
67.6818181818182	-1.28892704953902\\
67.7045454545455	-1.29760775524359\\
67.7272727272727	-1.30611911567967\\
67.75	-1.3144600200658\\
67.7727272727273	-1.32262937986604\\
67.7954545454545	-1.33062612893197\\
67.8181818181818	-1.33844922364192\\
67.8409090909091	-1.34609764303704\\
67.8636363636364	-1.35357038895468\\
67.8863636363636	-1.36086648615853\\
67.9090909090909	-1.36798498246601\\
67.9318181818182	-1.37492494887242\\
67.9545454545455	-1.38168547967228\\
67.9772727272727	-1.38826569257746\\
68	-1.39466472883238\\
68.0227272727273	-1.40088175332603\\
68.0454545454545	-1.40691595470098\\
68.0681818181818	-1.41276654545929\\
68.0909090909091	-1.41843276206524\\
68.1136363636364	-1.42391386504501\\
68.1363636363636	-1.42920913908315\\
68.1590909090909	-1.43431789311598\\
68.1818181818182	-1.43923946042175\\
68.2045454545455	-1.44397319870763\\
68.2272727272727	-1.44851849019358\\
68.25	-1.45287474169295\\
68.2727272727273	-1.45704138468987\\
68.2954545454545	-1.4610178754135\\
68.3181818181818	-1.46480369490895\\
68.3409090909091	-1.46839834910501\\
68.3636363636364	-1.47180136887866\\
68.3863636363636	-1.47501231011626\\
68.4090909090909	-1.47803075377154\\
68.4318181818182	-1.48085630592024\\
68.4545454545455	-1.48348859781159\\
68.4772727272727	-1.48592728591635\\
68.5	-1.48817205197172\\
68.5227272727273	-1.49022260302283\\
68.5454545454545	-1.49207867146098\\
68.5681818181818	-1.49374001505858\\
68.5909090909091	-1.49520641700074\\
68.6136363636364	-1.49647768591358\\
68.6363636363636	-1.4975536558892\\
68.6590909090909	-1.49843418650735\\
68.6818181818182	-1.4991191628537\\
68.7045454545455	-1.49960849553492\\
68.7272727272727	-1.49990212069028\\
68.75	-1.5\\
68.7727272727273	-1.49990212069028\\
68.7954545454545	-1.49960849553492\\
68.8181818181818	-1.4991191628537\\
68.8409090909091	-1.49843418650735\\
68.8636363636364	-1.4975536558892\\
68.8863636363637	-1.49647768591358\\
68.9090909090909	-1.49520641700074\\
68.9318181818182	-1.49374001505858\\
68.9545454545455	-1.49207867146098\\
68.9772727272727	-1.49022260302282\\
69	-1.48817205197172\\
69.0227272727273	-1.48592728591635\\
69.0454545454545	-1.48348859781159\\
69.0681818181818	-1.48085630592024\\
69.0909090909091	-1.47803075377154\\
69.1136363636364	-1.47501231011626\\
69.1363636363637	-1.47180136887866\\
69.1590909090909	-1.46839834910501\\
69.1818181818182	-1.46480369490894\\
69.2045454545455	-1.4610178754135\\
69.2272727272727	-1.45704138468987\\
69.25	-1.45287474169295\\
69.2727272727273	-1.44851849019358\\
69.2954545454545	-1.44397319870763\\
69.3181818181818	-1.43923946042175\\
69.3409090909091	-1.43431789311598\\
69.3636363636364	-1.42920913908314\\
69.3863636363637	-1.423913865045\\
69.4090909090909	-1.41843276206523\\
69.4318181818182	-1.41276654545928\\
69.4545454545455	-1.40691595470098\\
69.4772727272727	-1.40088175332602\\
69.5	-1.39466472883238\\
69.5227272727273	-1.38826569257746\\
69.5454545454545	-1.38168547967228\\
69.5681818181818	-1.37492494887241\\
69.5909090909091	-1.36798498246601\\
69.6136363636364	-1.36086648615853\\
69.6363636363636	-1.35357038895467\\
69.6590909090909	-1.34609764303703\\
69.6818181818182	-1.33844922364191\\
69.7045454545455	-1.33062612893197\\
69.7272727272727	-1.32262937986603\\
69.75	-1.31446002006579\\
69.7727272727273	-1.30611911567967\\
69.7954545454545	-1.29760775524359\\
69.8181818181818	-1.28892704953902\\
69.8409090909091	-1.28007813144792\\
69.8636363636364	-1.27106215580496\\
69.8863636363636	-1.26188029924677\\
69.9090909090909	-1.25253376005838\\
69.9318181818182	-1.24302375801688\\
69.9545454545455	-1.23335153423218\\
69.9772727272727	-1.22351835098505\\
70	-1.21352549156241\\
70.0227272727273	-1.20337426008986\\
70.0454545454545	-1.19306598136141\\
70.0681818181818	-1.18260200066669\\
70.0909090909091	-1.17198368361529\\
70.1136363636364	-1.16121241595862\\
70.1363636363636	-1.150289603409\\
70.1590909090909	-1.13921667145621\\
70.1818181818182	-1.12799506518154\\
70.2045454545455	-1.11662624906906\\
70.2272727272727	-1.1051117068146\\
70.25	-1.09345294113212\\
70.2727272727273	-1.0816514735575\\
70.2954545454545	-1.0697088442501\\
70.3181818181818	-1.05762661179166\\
70.3409090909091	-1.04540635298294\\
70.3636363636364	-1.03304966263793\\
70.3863636363636	-1.02055815337576\\
70.4090909090909	-1.00793345541013\\
70.4318181818182	-0.995177216336684\\
70.4545454545455	-0.982291100917926\\
70.4772727272727	-0.969276790865953\\
70.5	-0.956135984623028\\
70.5227272727273	-0.942870397139868\\
70.5454545454545	-0.929481759651868\\
70.5681818181818	-0.915971819453155\\
70.5909090909091	-0.902342339668557\\
70.6136363636364	-0.888595099023489\\
70.6363636363636	-0.874731891611892\\
70.6590909090909	-0.860754526661953\\
70.6818181818182	-0.846664828300151\\
70.7045454545455	-0.832464635313051\\
70.7272727272727	-0.818155800907471\\
70.75	-0.803740192468488\\
70.7727272727273	-0.789219691315868\\
70.7954545454545	-0.77459619245844\\
70.8181818181818	-0.759871604346836\\
70.8409090909091	-0.745047848624417\\
70.8636363636364	-0.730126859876477\\
70.8863636363636	-0.715110585377816\\
70.9090909090909	-0.700000984838507\\
70.9318181818182	-0.684800030148262\\
70.9545454545455	-0.669509705119028\\
70.9772727272727	-0.654132005226075\\
71	-0.638668937347603\\
71.0227272727273	-0.623122519502825\\
71.0454545454545	-0.607494780588606\\
71.0681818181818	-0.591787760114674\\
71.0909090909091	-0.576003507937458\\
71.1136363636364	-0.56014408399256\\
71.1363636363636	-0.544211558025957\\
71.1590909090909	-0.528208009323819\\
71.1818181818182	-0.512135526441235\\
71.2045454545455	-0.495996206929567\\
71.2272727272727	-0.479792157062754\\
71.25	-0.463525491562416\\
71.2727272727273	-0.447198333321873\\
71.2954545454545	-0.430812813129098\\
71.3181818181818	-0.414371069388631\\
71.3409090909091	-0.397875247842499\\
71.3636363636364	-0.381327501290214\\
71.3863636363636	-0.36472998930781\\
71.4090909090909	-0.348084877965935\\
71.4318181818182	-0.331394339547296\\
71.4545454545455	-0.314660552263055\\
71.4772727272727	-0.297885699968612\\
71.5	-0.281071971878582\\
71.5227272727273	-0.2642215622811\\
71.5454545454545	-0.247336670251444\\
71.5681818181818	-0.230419499365051\\
71.5909090909091	-0.213472257409921\\
71.6136363636364	-0.196497156098523\\
71.6363636363636	-0.179496410779146\\
71.6590909090909	-0.1624722401467\\
71.6818181818182	-0.145426865953312\\
71.7045454545455	-0.128362512718252\\
71.7272727272727	-0.111281407437672\\
71.75	-0.0941857792939664\\
71.7727272727273	-0.0770778593648324\\
71.7954545454545	-0.0599598803321602\\
71.8181818181818	-0.0428340761905362\\
71.8409090909091	-0.0257026819558532\\
71.8636363636364	-0.00856793337345848\\
71.8863636363636	0.00856793337346473\\
71.9090909090909	0.0257026819558595\\
71.9318181818182	0.0428340761905531\\
71.9545454545455	0.0599598803321665\\
71.9772727272727	0.0770778593648493\\
72	0.0941857792939833\\
72.0227272727273	0.111281407437679\\
72.0454545454545	0.128362512718269\\
72.0681818181818	0.145426865953329\\
72.0909090909091	0.162472240146717\\
72.1136363636364	0.179496410779152\\
72.1363636363636	0.19649715609854\\
72.1590909090909	0.213472257409937\\
72.1818181818182	0.230419499365057\\
72.2045454545455	0.24733667025145\\
72.2272727272727	0.264221562281106\\
72.25	0.281071971878599\\
72.2727272727273	0.297885699968628\\
72.2954545454545	0.314660552263072\\
72.3181818181818	0.331394339547312\\
72.3409090909091	0.348084877965951\\
72.3636363636364	0.364729989307816\\
72.3863636363636	0.38132750129023\\
72.4090909090909	0.397875247842515\\
72.4318181818182	0.414371069388626\\
72.4545454545455	0.430812813129104\\
72.4772727272727	0.447198333321889\\
72.5	0.463525491562422\\
72.5227272727273	0.47979215706277\\
72.5454545454545	0.495996206929583\\
72.5681818181818	0.512135526441251\\
72.5909090909091	0.528208009323835\\
72.6136363636364	0.544211558025962\\
72.6363636363636	0.560144083992566\\
72.6590909090909	0.576003507937464\\
72.6818181818182	0.591787760114669\\
72.7045454545455	0.607494780588611\\
72.7272727272727	0.623122519502841\\
72.75	0.638668937347618\\
72.7727272727273	0.65413200522609\\
72.7954545454545	0.669509705119044\\
72.8181818181818	0.684800030148277\\
72.8409090909091	0.700000984838512\\
72.8636363636364	0.715110585377812\\
72.8863636363636	0.730126859876492\\
72.9090909090909	0.745047848624432\\
72.9318181818182	0.759871604346841\\
72.9545454545455	0.774596192458454\\
72.9772727272727	0.789219691315873\\
73	0.803740192468502\\
73.0227272727273	0.818155800907476\\
73.0454545454545	0.832464635313065\\
73.0681818181818	0.846664828300156\\
73.0909090909091	0.860754526661967\\
73.1136363636364	0.874731891611905\\
73.1363636363636	0.888595099023502\\
73.1590909090909	0.902342339668571\\
73.1818181818182	0.915971819453151\\
73.2045454545455	0.929481759651881\\
73.2272727272727	0.942870397139881\\
73.25	0.956135984623041\\
73.2727272727273	0.969276790865966\\
73.2954545454545	0.98229110091793\\
73.3181818181818	0.995177216336689\\
73.3409090909091	1.00793345541013\\
73.3636363636364	1.02055815337577\\
73.3863636363636	1.03304966263795\\
73.4090909090909	1.04540635298295\\
73.4318181818182	1.05762661179166\\
73.4545454545455	1.06970884425011\\
73.4772727272727	1.08165147355751\\
73.5	1.09345294113212\\
73.5227272727273	1.10511170681461\\
73.5454545454545	1.11662624906906\\
73.5681818181818	1.12799506518154\\
73.5909090909091	1.13921667145623\\
73.6136363636364	1.15028960340899\\
73.6363636363636	1.16121241595863\\
73.6590909090909	1.17198368361531\\
73.6818181818182	1.18260200066669\\
73.7045454545455	1.19306598136142\\
73.7272727272727	1.20337426008987\\
73.75	1.21352549156242\\
73.7727272727273	1.22351835098505\\
73.7954545454545	1.23335153423218\\
73.8181818181818	1.24302375801689\\
73.8409090909091	1.25253376005839\\
73.8636363636364	1.26188029924678\\
73.8863636363636	1.27106215580497\\
73.9090909090909	1.28007813144793\\
73.9318181818182	1.28892704953902\\
73.9545454545455	1.2976077552436\\
73.9772727272727	1.30611911567967\\
74	1.3144600200658\\
74.0227272727273	1.32262937986603\\
74.0454545454545	1.33062612893197\\
74.0681818181818	1.33844922364191\\
74.0909090909091	1.34609764303704\\
74.1136363636364	1.35357038895468\\
74.1363636363636	1.36086648615853\\
74.1590909090909	1.36798498246601\\
74.1818181818182	1.37492494887242\\
74.2045454545455	1.38168547967228\\
74.2272727272727	1.38826569257746\\
74.25	1.39466472883238\\
74.2727272727273	1.40088175332602\\
74.2954545454545	1.40691595470098\\
74.3181818181818	1.41276654545929\\
74.3409090909091	1.41843276206524\\
74.3636363636364	1.423913865045\\
74.3863636363636	1.42920913908315\\
74.4090909090909	1.43431789311598\\
74.4318181818182	1.43923946042175\\
74.4545454545455	1.44397319870763\\
74.4772727272727	1.44851849019358\\
74.5	1.45287474169295\\
74.5227272727273	1.45704138468987\\
74.5454545454545	1.4610178754135\\
74.5681818181818	1.46480369490895\\
74.5909090909091	1.46839834910501\\
74.6136363636364	1.47180136887866\\
74.6363636363636	1.47501231011626\\
74.6590909090909	1.47803075377154\\
74.6818181818182	1.48085630592024\\
74.7045454545455	1.48348859781159\\
74.7272727272727	1.48592728591635\\
74.75	1.48817205197172\\
74.7727272727273	1.49022260302283\\
74.7954545454546	1.49207867146098\\
74.8181818181818	1.49374001505858\\
74.8409090909091	1.49520641700074\\
74.8636363636364	1.49647768591358\\
74.8863636363636	1.4975536558892\\
74.9090909090909	1.49843418650735\\
74.9318181818182	1.4991191628537\\
74.9545454545455	1.49960849553492\\
74.9772727272727	1.49990212069028\\
75	1.5\\
75.0227272727273	1.49990212069028\\
75.0454545454546	1.49960849553492\\
75.0681818181818	1.4991191628537\\
75.0909090909091	1.49843418650735\\
75.1136363636364	1.4975536558892\\
75.1363636363636	1.49647768591358\\
75.1590909090909	1.49520641700074\\
75.1818181818182	1.49374001505858\\
75.2045454545455	1.49207867146098\\
75.2272727272727	1.49022260302283\\
75.25	1.48817205197172\\
75.2727272727273	1.48592728591635\\
75.2954545454546	1.48348859781159\\
75.3181818181818	1.48085630592024\\
75.3409090909091	1.47803075377154\\
75.3636363636364	1.47501231011626\\
75.3863636363636	1.47180136887866\\
75.4090909090909	1.46839834910501\\
75.4318181818182	1.46480369490894\\
75.4545454545455	1.4610178754135\\
75.4772727272727	1.45704138468987\\
75.5	1.45287474169294\\
75.5227272727273	1.44851849019358\\
75.5454545454546	1.44397319870763\\
75.5681818181818	1.43923946042174\\
75.5909090909091	1.43431789311598\\
75.6136363636364	1.42920913908314\\
75.6363636363636	1.423913865045\\
75.6590909090909	1.41843276206524\\
75.6818181818182	1.41276654545929\\
75.7045454545455	1.40691595470097\\
75.7272727272727	1.40088175332602\\
75.75	1.39466472883237\\
75.7727272727273	1.38826569257746\\
75.7954545454546	1.38168547967227\\
75.8181818181818	1.37492494887242\\
75.8409090909091	1.367984982466\\
75.8636363636364	1.36086648615853\\
75.8863636363636	1.35357038895467\\
75.9090909090909	1.34609764303704\\
75.9318181818182	1.33844922364191\\
75.9545454545455	1.33062612893196\\
75.9772727272727	1.32262937986603\\
76	1.31446002006579\\
76.0227272727273	1.30611911567966\\
76.0454545454546	1.29760775524358\\
76.0681818181818	1.28892704953902\\
76.0909090909091	1.28007813144792\\
76.1136363636364	1.27106215580496\\
76.1363636363636	1.26188029924677\\
76.1590909090909	1.25253376005839\\
76.1818181818182	1.24302375801689\\
76.2045454545455	1.23335153423218\\
76.2272727272727	1.22351835098505\\
76.25	1.21352549156242\\
76.2727272727273	1.20337426008985\\
76.2954545454546	1.19306598136142\\
76.3181818181818	1.18260200066669\\
76.3409090909091	1.1719836836153\\
76.3636363636364	1.16121241595862\\
76.3863636363636	1.15028960340899\\
76.4090909090909	1.13921667145622\\
76.4318181818182	1.12799506518154\\
76.4545454545455	1.11662624906905\\
76.4772727272727	1.1051117068146\\
76.5	1.09345294113211\\
76.5227272727273	1.08165147355749\\
76.5454545454546	1.06970884425009\\
76.5681818181818	1.05762661179165\\
76.5909090909091	1.04540635298294\\
76.6136363636364	1.03304966263794\\
76.6363636363636	1.02055815337576\\
76.6590909090909	1.00793345541013\\
76.6818181818182	0.995177216336689\\
76.7045454545455	0.982291100917914\\
76.7272727272727	0.969276790865957\\
76.75	0.956135984623024\\
76.7727272727273	0.942870397139864\\
76.7954545454546	0.929481759651864\\
76.8181818181818	0.915971819453151\\
76.8409090909091	0.902342339668553\\
76.8636363636364	0.888595099023502\\
76.8863636363636	0.874731891611896\\
76.9090909090909	0.860754526661966\\
76.9318181818182	0.846664828300155\\
76.9545454545455	0.832464635313055\\
76.9772727272727	0.818155800907475\\
77	0.803740192468484\\
77.0227272727273	0.789219691315863\\
77.0454545454546	0.774596192458435\\
77.0681818181818	0.759871604346831\\
77.0909090909091	0.745047848624413\\
77.1136363636364	0.730126859876482\\
77.1363636363636	0.715110585377812\\
77.1590909090909	0.700000984838512\\
77.1818181818182	0.684800030148277\\
77.2045454545455	0.669509705119033\\
77.2272727272727	0.65413200522608\\
77.25	0.638668937347598\\
77.2727272727273	0.623122519502821\\
77.2954545454546	0.607494780588601\\
77.3181818181818	0.591787760114669\\
77.3409090909091	0.576003507937444\\
77.3636363636364	0.560144083992565\\
77.3863636363636	0.544211558025952\\
77.4090909090909	0.528208009323834\\
77.4318181818182	0.51213552644124\\
77.4545454545455	0.495996206929583\\
77.4772727272727	0.479792157062759\\
77.5	0.463525491562411\\
77.5227272727273	0.447198333321868\\
77.5454545454546	0.430812813129093\\
77.5681818181818	0.414371069388626\\
77.5909090909091	0.397875247842504\\
77.6136363636364	0.381327501290219\\
77.6363636363636	0.364729989307816\\
77.6590909090909	0.34808487796594\\
77.6818181818182	0.331394339547301\\
77.7045454545455	0.314660552263061\\
77.7272727272727	0.297885699968617\\
77.75	0.281071971878577\\
77.7727272727273	0.264221562281105\\
77.7954545454546	0.247336670251439\\
77.8181818181818	0.230419499365035\\
77.8409090909091	0.213472257409926\\
77.8636363636364	0.196497156098529\\
77.8863636363636	0.17949641077914\\
77.9090909090909	0.162472240146706\\
77.9318181818182	0.145426865953318\\
77.9545454545455	0.128362512718257\\
77.9772727272727	0.111281407437678\\
78	0.0941857792939613\\
78.0227272727273	0.0770778593648485\\
78.0454545454546	0.0599598803321551\\
78.0681818181818	0.0428340761905417\\
78.0909090909091	0.0257026819558374\\
78.1136363636364	0.00856793337345334\\
78.1363636363636	-0.00856793337346988\\
78.1590909090909	-0.025702681955854\\
78.1818181818182	-0.0428340761905476\\
78.2045454545455	-0.059959880332161\\
78.2272727272727	-0.0770778593648438\\
78.25	-0.0941857792939778\\
78.2727272727273	-0.111281407437673\\
78.2954545454546	-0.128362512718274\\
78.3181818181818	-0.145426865953324\\
78.3409090909091	-0.162472240146722\\
78.3636363636364	-0.179496410779157\\
78.3863636363636	-0.196497156098545\\
78.4090909090909	-0.213472257409932\\
78.4318181818182	-0.230419499365051\\
78.4545454545455	-0.247336670251455\\
78.4772727272727	-0.264221562281101\\
78.5	-0.281071971878593\\
78.5227272727273	-0.297885699968612\\
78.5454545454546	-0.314660552263066\\
78.5681818181818	-0.331394339547317\\
78.5909090909091	-0.348084877965956\\
78.6136363636364	-0.364729989307821\\
78.6363636363636	-0.381327501290235\\
78.6590909090909	-0.39787524784251\\
78.6818181818182	-0.414371069388631\\
78.7045454545455	-0.430812813129098\\
78.7272727272727	-0.447198333321884\\
78.75	-0.463525491562437\\
78.7727272727273	-0.479792157062765\\
78.7954545454546	-0.495996206929588\\
78.8181818181818	-0.512135526441256\\
78.8409090909091	-0.52820800932384\\
78.8636363636364	-0.544211558025967\\
78.8863636363636	-0.560144083992571\\
78.9090909090909	-0.576003507937459\\
78.9318181818182	-0.591787760114684\\
78.9545454545455	-0.607494780588606\\
78.9772727272727	-0.623122519502836\\
79	-0.638668937347623\\
79.0227272727273	-0.654132005226085\\
79.0454545454546	-0.669509705119048\\
79.0681818181818	-0.684800030148282\\
79.0909090909091	-0.700000984838517\\
79.1136363636364	-0.715110585377817\\
79.1363636363636	-0.730126859876487\\
79.1590909090909	-0.745047848624418\\
79.1818181818182	-0.759871604346836\\
79.2045454545455	-0.774596192458449\\
79.2272727272727	-0.789219691315868\\
79.25	-0.803740192468507\\
79.2727272727273	-0.818155800907471\\
79.2954545454546	-0.832464635313069\\
79.3181818181818	-0.84666482830016\\
79.3409090909091	-0.860754526661971\\
79.3636363636364	-0.874731891611901\\
79.3863636363636	-0.888595099023506\\
79.4090909090909	-0.902342339668566\\
79.4318181818182	-0.915971819453155\\
79.4545454545455	-0.929481759651877\\
79.4772727272727	-0.942870397139877\\
79.5	-0.956135984623045\\
79.5227272727273	-0.969276790865962\\
79.5454545454546	-0.982291100917934\\
79.5681818181818	-0.995177216336693\\
79.5909090909091	-1.00793345541014\\
79.6136363636364	-1.02055815337576\\
79.6363636363636	-1.03304966263794\\
79.6590909090909	-1.04540635298295\\
79.6818181818182	-1.05762661179166\\
79.7045454545455	-1.0697088442501\\
79.7272727272727	-1.08165147355751\\
79.75	-1.09345294113213\\
79.7727272727273	-1.10511170681461\\
79.7954545454546	-1.11662624906906\\
79.8181818181818	-1.12799506518154\\
79.8409090909091	-1.13921667145623\\
79.8636363636364	-1.150289603409\\
79.8863636363636	-1.16121241595862\\
79.9090909090909	-1.1719836836153\\
79.9318181818182	-1.1826020006667\\
79.9545454545455	-1.19306598136142\\
79.9772727272727	-1.20337426008986\\
80	-1.21352549156242\\
80.0227272727273	-1.22351835098505\\
80.0454545454546	-1.23335153423218\\
80.0681818181818	-1.24302375801689\\
80.0909090909091	-1.2525337600584\\
80.1136363636364	-1.26188029924678\\
80.1363636363636	-1.27106215580497\\
80.1590909090909	-1.28007813144792\\
80.1818181818182	-1.28892704953902\\
80.2045454545455	-1.29760775524359\\
80.2272727272727	-1.30611911567967\\
80.25	-1.3144600200658\\
80.2727272727273	-1.32262937986603\\
80.2954545454546	-1.33062612893197\\
80.3181818181818	-1.33844922364191\\
80.3409090909091	-1.34609764303704\\
80.3636363636364	-1.35357038895468\\
80.3863636363636	-1.36086648615854\\
80.4090909090909	-1.36798498246601\\
80.4318181818182	-1.37492494887242\\
80.4545454545455	-1.38168547967228\\
80.4772727272727	-1.38826569257746\\
80.5	-1.39466472883238\\
80.5227272727273	-1.40088175332603\\
80.5454545454546	-1.40691595470098\\
80.5681818181818	-1.41276654545928\\
80.5909090909091	-1.41843276206524\\
80.6136363636364	-1.42391386504501\\
80.6363636363636	-1.42920913908315\\
80.6590909090909	-1.43431789311598\\
80.6818181818182	-1.43923946042175\\
80.7045454545455	-1.44397319870763\\
80.7272727272727	-1.44851849019358\\
80.75	-1.45287474169295\\
80.7727272727273	-1.45704138468987\\
80.7954545454546	-1.4610178754135\\
80.8181818181818	-1.46480369490894\\
80.8409090909091	-1.46839834910501\\
80.8636363636364	-1.47180136887866\\
80.8863636363636	-1.47501231011627\\
80.9090909090909	-1.47803075377154\\
80.9318181818182	-1.48085630592024\\
80.9545454545455	-1.48348859781159\\
80.9772727272727	-1.48592728591635\\
81	-1.48817205197172\\
81.0227272727273	-1.49022260302283\\
81.0454545454546	-1.49207867146098\\
81.0681818181818	-1.49374001505858\\
81.0909090909091	-1.49520641700074\\
81.1136363636364	-1.49647768591358\\
81.1363636363636	-1.4975536558892\\
81.1590909090909	-1.49843418650735\\
81.1818181818182	-1.4991191628537\\
81.2045454545455	-1.49960849553492\\
81.2272727272727	-1.49990212069028\\
81.25	-1.5\\
81.2727272727273	-1.49990212069028\\
81.2954545454546	-1.49960849553492\\
81.3181818181818	-1.4991191628537\\
81.3409090909091	-1.49843418650735\\
81.3636363636364	-1.4975536558892\\
81.3863636363636	-1.49647768591358\\
81.4090909090909	-1.49520641700074\\
81.4318181818182	-1.49374001505858\\
81.4545454545455	-1.49207867146098\\
81.4772727272727	-1.49022260302282\\
81.5	-1.48817205197172\\
81.5227272727273	-1.48592728591635\\
81.5454545454546	-1.48348859781158\\
81.5681818181818	-1.48085630592024\\
81.5909090909091	-1.47803075377154\\
81.6136363636364	-1.47501231011626\\
81.6363636363636	-1.47180136887866\\
81.6590909090909	-1.46839834910501\\
81.6818181818182	-1.46480369490895\\
81.7045454545455	-1.4610178754135\\
81.7272727272727	-1.45704138468987\\
81.75	-1.45287474169295\\
81.7727272727273	-1.44851849019358\\
81.7954545454546	-1.44397319870763\\
81.8181818181818	-1.43923946042174\\
81.8409090909091	-1.43431789311598\\
81.8636363636364	-1.42920913908315\\
81.8863636363636	-1.423913865045\\
81.9090909090909	-1.41843276206523\\
81.9318181818182	-1.41276654545928\\
81.9545454545455	-1.40691595470098\\
81.9772727272727	-1.40088175332602\\
82	-1.39466472883238\\
82.0227272727273	-1.38826569257746\\
82.0454545454546	-1.38168547967227\\
82.0681818181818	-1.37492494887241\\
82.0909090909091	-1.367984982466\\
82.1136363636364	-1.36086648615853\\
82.1363636363636	-1.35357038895467\\
82.1590909090909	-1.34609764303703\\
82.1818181818182	-1.33844922364191\\
82.2045454545455	-1.33062612893197\\
82.2272727272727	-1.32262937986603\\
82.25	-1.31446002006579\\
82.2727272727273	-1.30611911567967\\
82.2954545454546	-1.29760775524359\\
82.3181818181818	-1.28892704953902\\
82.3409090909091	-1.28007813144792\\
82.3636363636364	-1.27106215580496\\
82.3863636363636	-1.26188029924677\\
82.4090909090909	-1.25253376005839\\
82.4318181818182	-1.24302375801688\\
82.4545454545455	-1.23335153423218\\
82.4772727272727	-1.22351835098505\\
82.5	-1.21352549156241\\
82.5227272727273	-1.20337426008986\\
82.5454545454546	-1.19306598136141\\
82.5681818181818	-1.18260200066669\\
82.5909090909091	-1.17198368361529\\
82.6136363636364	-1.16121241595862\\
82.6363636363636	-1.15028960340899\\
82.6590909090909	-1.13921667145621\\
82.6818181818182	-1.12799506518153\\
82.7045454545455	-1.11662624906906\\
82.7272727272727	-1.10511170681461\\
82.75	-1.09345294113211\\
82.7727272727273	-1.0816514735575\\
82.7954545454545	-1.0697088442501\\
82.8181818181818	-1.05762661179166\\
82.8409090909091	-1.04540635298294\\
82.8636363636364	-1.03304966263794\\
82.8863636363636	-1.02055815337575\\
82.9090909090909	-1.00793345541013\\
82.9318181818182	-0.995177216336677\\
82.9545454545455	-0.982291100917918\\
82.9772727272727	-0.969276790865961\\
83	-0.956135984623036\\
83.0227272727273	-0.942870397139868\\
83.0454545454545	-0.929481759651868\\
83.0681818181818	-0.915971819453164\\
83.0909090909091	-0.902342339668557\\
83.1136363636364	-0.888595099023506\\
83.1363636363636	-0.874731891611892\\
83.1590909090909	-0.860754526661962\\
83.1818181818182	-0.846664828300151\\
83.2045454545455	-0.832464635313051\\
83.2272727272727	-0.818155800907462\\
83.25	-0.803740192468497\\
83.2727272727273	-0.789219691315868\\
83.2954545454545	-0.77459619245844\\
83.3181818181818	-0.759871604346836\\
83.3409090909091	-0.745047848624408\\
83.3636363636364	-0.730126859876487\\
83.3863636363637	-0.715110585377807\\
83.4090909090909	-0.700000984838507\\
83.4318181818182	-0.684800030148272\\
83.4545454545455	-0.669509705119029\\
83.4772727272727	-0.654132005226075\\
83.5	-0.638668937347603\\
83.5227272727273	-0.623122519502835\\
83.5454545454545	-0.607494780588606\\
83.5681818181818	-0.591787760114674\\
83.5909090909091	-0.576003507937449\\
83.6136363636364	-0.56014408399256\\
83.6363636363637	-0.544211558025947\\
83.6590909090909	-0.52820800932383\\
83.6818181818182	-0.512135526441245\\
83.7045454545455	-0.495996206929568\\
83.7272727272727	-0.479792157062754\\
83.75	-0.463525491562416\\
83.7727272727273	-0.447198333321874\\
83.7954545454545	-0.430812813129088\\
83.8181818181818	-0.414371069388631\\
83.8409090909091	-0.397875247842499\\
83.8636363636364	-0.381327501290235\\
83.8863636363637	-0.3647299893078\\
83.9090909090909	-0.348084877965935\\
83.9318181818182	-0.331394339547306\\
83.9545454545455	-0.314660552263066\\
83.9772727272727	-0.297885699968612\\
84	-0.281071971878583\\
84.0227272727273	-0.2642215622811\\
84.0454545454545	-0.247336670251434\\
84.0681818181818	-0.23041949936504\\
84.0909090909091	-0.21347225740991\\
84.1136363636364	-0.196497156098545\\
84.1363636363637	-0.179496410779146\\
84.1590909090909	-0.162472240146701\\
84.1818181818182	-0.145426865953313\\
84.2045454545455	-0.128362512718263\\
84.2272727272727	-0.111281407437673\\
84.25	-0.0941857792939668\\
84.2727272727273	-0.0770778593648434\\
84.2954545454545	-0.0599598803321606\\
84.3181818181818	-0.0428340761905365\\
84.3409090909091	-0.0257026819558323\\
84.3636363636364	-0.00856793337346951\\
84.3863636363637	0.00856793337346436\\
84.4090909090909	0.0257026819558591\\
84.4318181818182	0.0428340761905527\\
84.4545454545455	0.0599598803321661\\
84.4772727272727	0.0770778593648382\\
84.5	0.0941857792939723\\
84.5227272727273	0.111281407437678\\
84.5454545454545	0.128362512718258\\
84.5681818181818	0.145426865953329\\
84.5909090909091	0.162472240146717\\
84.6136363636364	0.179496410779151\\
84.6363636363637	0.19649715609855\\
84.6590909090909	0.213472257409947\\
84.6818181818182	0.230419499365056\\
84.7045454545455	0.24733667025145\\
84.7272727272727	0.264221562281106\\
84.75	0.281071971878599\\
84.7727272727273	0.297885699968617\\
84.7954545454545	0.314660552263061\\
84.8181818181818	0.331394339547312\\
84.8409090909091	0.348084877965951\\
84.8636363636364	0.364729989307816\\
84.8863636363637	0.38132750129024\\
84.9090909090909	0.397875247842515\\
84.9318181818182	0.414371069388647\\
84.9545454545455	0.430812813129103\\
84.9772727272727	0.447198333321879\\
85	0.463525491562431\\
85.0227272727273	0.47979215706277\\
85.0454545454545	0.495996206929573\\
85.0681818181818	0.512135526441241\\
85.0909090909091	0.528208009323835\\
85.1136363636364	0.544211558025962\\
85.1363636363637	0.560144083992576\\
85.1590909090909	0.576003507937464\\
85.1818181818182	0.591787760114689\\
85.2045454545455	0.607494780588621\\
85.2272727272727	0.623122519502831\\
85.25	0.638668937347608\\
85.2727272727273	0.65413200522609\\
85.2954545454545	0.669509705119043\\
85.3181818181818	0.684800030148267\\
85.3409090909091	0.700000984838512\\
85.3636363636364	0.715110585377812\\
85.3863636363637	0.730126859876492\\
85.4090909090909	0.745047848624422\\
85.4318181818182	0.759871604346841\\
85.4545454545455	0.774596192458454\\
85.4772727272727	0.789219691315882\\
85.5	0.803740192468493\\
85.5227272727273	0.818155800907476\\
85.5454545454545	0.832464635313064\\
85.5681818181818	0.846664828300156\\
85.5909090909091	0.860754526661958\\
85.6136363636364	0.874731891611897\\
85.6363636363637	0.888595099023511\\
85.6590909090909	0.902342339668562\\
85.6818181818182	0.915971819453159\\
85.7045454545455	0.929481759651881\\
85.7272727272727	0.942870397139881\\
85.75	0.95613598462304\\
85.7727272727273	0.969276790865974\\
85.7954545454545	0.98229110091793\\
85.8181818181818	0.995177216336689\\
85.8409090909091	1.00793345541013\\
85.8636363636364	1.02055815337577\\
85.8863636363637	1.03304966263795\\
85.9090909090909	1.04540635298294\\
85.9318181818182	1.05762661179166\\
85.9545454545455	1.06970884425011\\
85.9772727272727	1.08165147355751\\
86	1.09345294113212\\
86.0227272727273	1.10511170681461\\
86.0454545454545	1.11662624906907\\
86.0681818181818	1.12799506518154\\
86.0909090909091	1.13921667145622\\
86.1136363636364	1.150289603409\\
86.1363636363637	1.16121241595863\\
86.1590909090909	1.17198368361531\\
86.1818181818182	1.18260200066669\\
86.2045454545455	1.19306598136142\\
86.2272727272727	1.20337426008987\\
86.25	1.21352549156242\\
86.2727272727273	1.22351835098505\\
86.2954545454545	1.23335153423219\\
86.3181818181818	1.24302375801689\\
86.3409090909091	1.25253376005839\\
86.3636363636364	1.26188029924677\\
86.3863636363637	1.27106215580497\\
86.4090909090909	1.28007813144793\\
86.4318181818182	1.28892704953902\\
86.4545454545455	1.29760775524359\\
86.4772727272727	1.30611911567967\\
86.5	1.3144600200658\\
86.5227272727273	1.32262937986603\\
86.5454545454545	1.33062612893197\\
86.5681818181818	1.33844922364191\\
86.5909090909091	1.34609764303704\\
86.6136363636364	1.35357038895467\\
86.6363636363637	1.36086648615853\\
86.6590909090909	1.36798498246601\\
86.6818181818182	1.37492494887242\\
86.7045454545455	1.38168547967228\\
86.7272727272727	1.38826569257747\\
86.75	1.39466472883238\\
86.7727272727273	1.40088175332602\\
86.7954545454545	1.40691595470098\\
86.8181818181818	1.41276654545929\\
86.8409090909091	1.41843276206524\\
86.8636363636364	1.423913865045\\
86.8863636363637	1.42920913908315\\
86.9090909090909	1.43431789311598\\
86.9318181818182	1.43923946042175\\
86.9545454545455	1.44397319870763\\
86.9772727272727	1.44851849019358\\
87	1.45287474169295\\
87.0227272727273	1.45704138468987\\
87.0454545454545	1.4610178754135\\
87.0681818181818	1.46480369490895\\
87.0909090909091	1.46839834910501\\
87.1136363636364	1.47180136887866\\
87.1363636363636	1.47501231011626\\
87.1590909090909	1.47803075377154\\
87.1818181818182	1.48085630592025\\
87.2045454545455	1.48348859781159\\
87.2272727272727	1.48592728591635\\
87.25	1.48817205197172\\
87.2727272727273	1.49022260302283\\
87.2954545454545	1.49207867146098\\
87.3181818181818	1.49374001505858\\
87.3409090909091	1.49520641700074\\
87.3636363636364	1.49647768591358\\
87.3863636363636	1.4975536558892\\
87.4090909090909	1.49843418650735\\
87.4318181818182	1.4991191628537\\
87.4545454545455	1.49960849553492\\
87.4772727272727	1.49990212069028\\
87.5	1.5\\
87.5227272727273	1.49990212069028\\
87.5454545454545	1.49960849553492\\
87.5681818181818	1.4991191628537\\
87.5909090909091	1.49843418650735\\
87.6136363636364	1.4975536558892\\
87.6363636363636	1.49647768591358\\
87.6590909090909	1.49520641700074\\
87.6818181818182	1.49374001505858\\
87.7045454545455	1.49207867146098\\
87.7272727272727	1.49022260302282\\
87.75	1.48817205197172\\
87.7727272727273	1.48592728591635\\
87.7954545454545	1.48348859781159\\
87.8181818181818	1.48085630592024\\
87.8409090909091	1.47803075377154\\
87.8636363636364	1.47501231011626\\
87.8863636363636	1.47180136887866\\
87.9090909090909	1.46839834910501\\
87.9318181818182	1.46480369490894\\
87.9545454545455	1.4610178754135\\
87.9772727272727	1.45704138468987\\
88	1.45287474169294\\
88.0227272727273	1.44851849019358\\
88.0454545454545	1.44397319870763\\
88.0681818181818	1.43923946042175\\
88.0909090909091	1.43431789311598\\
88.1136363636364	1.42920913908314\\
88.1363636363636	1.423913865045\\
88.1590909090909	1.41843276206524\\
88.1818181818182	1.41276654545929\\
88.2045454545455	1.40691595470097\\
88.2272727272727	1.40088175332602\\
88.25	1.39466472883237\\
88.2727272727273	1.38826569257746\\
88.2954545454545	1.38168547967227\\
88.3181818181818	1.37492494887242\\
88.3409090909091	1.36798498246601\\
88.3636363636364	1.36086648615853\\
88.3863636363636	1.35357038895467\\
88.4090909090909	1.34609764303704\\
88.4318181818182	1.33844922364191\\
88.4545454545455	1.33062612893197\\
88.4772727272727	1.32262937986602\\
88.5	1.31446002006579\\
88.5227272727273	1.30611911567966\\
88.5454545454545	1.29760775524359\\
88.5681818181818	1.28892704953901\\
88.5909090909091	1.28007813144792\\
88.6136363636364	1.27106215580497\\
88.6363636363636	1.26188029924677\\
88.6590909090909	1.25253376005839\\
88.6818181818182	1.24302375801688\\
88.7045454545455	1.23335153423218\\
88.7272727272727	1.22351835098505\\
88.75	1.21352549156242\\
88.7727272727273	1.20337426008986\\
88.7954545454545	1.19306598136142\\
88.8181818181818	1.18260200066669\\
88.8409090909091	1.17198368361529\\
88.8636363636364	1.16121241595863\\
88.8863636363636	1.15028960340899\\
88.9090909090909	1.13921667145622\\
88.9318181818182	1.12799506518153\\
88.9545454545455	1.11662624906905\\
88.9772727272727	1.1051117068146\\
89	1.09345294113211\\
89.0227272727273	1.0816514735575\\
89.0454545454545	1.0697088442501\\
89.0681818181818	1.05762661179165\\
89.0909090909091	1.04540635298294\\
89.1136363636364	1.03304966263794\\
89.1363636363636	1.02055815337575\\
89.1590909090909	1.00793345541013\\
89.1818181818182	0.995177216336681\\
89.2045454545455	0.98229110091793\\
89.2272727272727	0.969276790865957\\
89.25	0.956135984623024\\
89.2727272727273	0.942870397139872\\
89.2954545454545	0.929481759651872\\
89.3181818181818	0.915971819453151\\
89.3409090909091	0.902342339668553\\
89.3636363636364	0.888595099023502\\
89.3863636363636	0.874731891611897\\
89.4090909090909	0.860754526661958\\
89.4318181818182	0.846664828300138\\
89.4545454545455	0.832464635313064\\
89.4772727272727	0.818155800907467\\
89.5	0.803740192468484\\
89.5227272727273	0.789219691315864\\
89.5454545454545	0.774596192458445\\
89.5681818181818	0.759871604346841\\
89.5909090909091	0.745047848624413\\
89.6136363636364	0.730126859876482\\
89.6363636363636	0.715110585377812\\
89.6590909090909	0.700000984838502\\
89.6818181818182	0.684800030148267\\
89.7045454545455	0.669509705119043\\
89.7272727272727	0.65413200522608\\
89.75	0.638668937347598\\
89.7727272727273	0.623122519502821\\
89.7954545454545	0.607494780588601\\
89.8181818181818	0.591787760114679\\
89.8409090909091	0.576003507937454\\
89.8636363636364	0.560144083992566\\
89.8863636363636	0.544211558025962\\
89.9090909090909	0.528208009323835\\
89.9318181818182	0.512135526441241\\
89.9545454545455	0.495996206929573\\
89.9772727272727	0.479792157062749\\
90	0.463525491562421\\
90.0227272727273	0.447198333321869\\
90.0454545454545	0.430812813129093\\
90.0681818181818	0.414371069388626\\
90.0909090909091	0.397875247842494\\
90.1136363636364	0.38132750129023\\
90.1363636363636	0.364729989307816\\
90.1590909090909	0.348084877965951\\
90.1818181818182	0.331394339547301\\
90.2045454545455	0.314660552263061\\
90.2272727272727	0.297885699968607\\
90.25	0.281071971878578\\
90.2727272727273	0.264221562281085\\
90.2954545454545	0.247336670251439\\
90.3181818181818	0.230419499365046\\
90.3409090909091	0.213472257409926\\
90.3636363636364	0.196497156098529\\
90.3863636363636	0.179496410779151\\
90.4090909090909	0.162472240146717\\
90.4318181818182	0.145426865953318\\
90.4545454545455	0.128362512718258\\
90.4772727272727	0.111281407437668\\
90.5	0.0941857792939617\\
90.5227272727273	0.0770778593648382\\
90.5454545454546	0.0599598803321448\\
90.5681818181818	0.0428340761905421\\
90.5909090909091	0.0257026819558484\\
90.6136363636364	0.0085679333734537\\
90.6363636363636	-0.00856793337346951\\
90.6590909090909	-0.0257026819558429\\
90.6818181818182	-0.0428340761905365\\
90.7045454545455	-0.0599598803321606\\
90.7272727272727	-0.077077859364854\\
90.75	-0.0941857792939774\\
90.7727272727273	-0.111281407437683\\
90.7954545454546	-0.128362512718274\\
90.8181818181818	-0.145426865953334\\
90.8409090909091	-0.162472240146711\\
90.8636363636364	-0.179496410779146\\
90.8863636363636	-0.196497156098545\\
90.9090909090909	-0.213472257409931\\
};
\addplot [color=mycolor2, line width=2.0pt]
  table[row sep=crcr]{%
90.9090909090909	-0.213472257409931\\
90.9318181818182	-0.23041949936504\\
90.9545454545455	-0.247336670251445\\
90.9772727272727	-0.264221562281111\\
91	-0.281071971878593\\
91.0227272727273	-0.297885699968622\\
91.0454545454546	-0.314660552263076\\
91.0681818181818	-0.331394339547317\\
91.0909090909091	-0.348084877965956\\
91.1136363636364	-0.364729989307811\\
91.1363636363636	-0.381327501290235\\
91.1590909090909	-0.397875247842509\\
91.1818181818182	-0.414371069388631\\
91.2045454545455	-0.430812813129108\\
91.2272727272727	-0.447198333321894\\
91.25	-0.463525491562426\\
91.2727272727273	-0.479792157062764\\
91.2954545454546	-0.495996206929578\\
91.3181818181818	-0.512135526441255\\
91.3409090909091	-0.52820800932384\\
91.3636363636364	-0.544211558025967\\
91.3863636363636	-0.56014408399258\\
91.4090909090909	-0.576003507937459\\
91.4318181818182	-0.591787760114674\\
91.4545454545455	-0.607494780588606\\
91.4772727272727	-0.623122519502835\\
91.5	-0.638668937347622\\
91.5227272727273	-0.654132005226085\\
91.5454545454546	-0.669509705119038\\
91.5681818181818	-0.684800030148282\\
91.5909090909091	-0.700000984838516\\
91.6136363636364	-0.715110585377816\\
91.6363636363636	-0.730126859876487\\
91.6590909090909	-0.745047848624427\\
91.6818181818182	-0.759871604346836\\
91.7045454545455	-0.77459619245844\\
91.7272727272727	-0.789219691315877\\
91.75	-0.803740192468506\\
91.7727272727273	-0.81815580090748\\
91.7954545454546	-0.83246463531306\\
91.8181818181818	-0.846664828300151\\
91.8409090909091	-0.860754526661971\\
91.8636363636364	-0.874731891611901\\
91.8863636363636	-0.888595099023506\\
91.9090909090909	-0.902342339668566\\
91.9318181818182	-0.915971819453164\\
91.9545454545455	-0.929481759651868\\
91.9772727272727	-0.942870397139876\\
92	-0.956135984623036\\
92.0227272727273	-0.969276790865969\\
92.0454545454546	-0.982291100917934\\
92.0681818181818	-0.995177216336685\\
92.0909090909091	-1.00793345541014\\
92.1136363636364	-1.02055815337576\\
92.1363636363636	-1.03304966263794\\
92.1590909090909	-1.04540635298295\\
92.1818181818182	-1.05762661179166\\
92.2045454545455	-1.0697088442501\\
92.2272727272727	-1.0816514735575\\
92.25	-1.09345294113212\\
92.2727272727273	-1.10511170681461\\
92.2954545454546	-1.11662624906906\\
92.3181818181818	-1.12799506518154\\
92.3409090909091	-1.13921667145623\\
92.3636363636364	-1.150289603409\\
92.3863636363636	-1.16121241595862\\
92.4090909090909	-1.1719836836153\\
92.4318181818182	-1.18260200066669\\
92.4545454545455	-1.19306598136142\\
92.4772727272727	-1.20337426008986\\
92.5	-1.21352549156243\\
92.5227272727273	-1.22351835098506\\
92.5454545454546	-1.23335153423218\\
92.5681818181818	-1.24302375801689\\
92.5909090909091	-1.25253376005839\\
92.6136363636364	-1.26188029924678\\
92.6363636363636	-1.27106215580496\\
92.6590909090909	-1.28007813144792\\
92.6818181818182	-1.28892704953902\\
92.7045454545455	-1.29760775524359\\
92.7272727272727	-1.30611911567967\\
92.75	-1.31446002006579\\
92.7727272727273	-1.32262937986603\\
92.7954545454546	-1.33062612893198\\
92.8181818181818	-1.33844922364191\\
92.8409090909091	-1.34609764303704\\
92.8636363636364	-1.35357038895468\\
92.8863636363636	-1.36086648615854\\
92.9090909090909	-1.367984982466\\
92.9318181818182	-1.37492494887241\\
92.9545454545455	-1.38168547967228\\
92.9772727272727	-1.38826569257746\\
93	-1.39466472883238\\
93.0227272727273	-1.40088175332603\\
93.0454545454546	-1.40691595470098\\
93.0681818181818	-1.41276654545929\\
93.0909090909091	-1.41843276206524\\
93.1136363636364	-1.42391386504501\\
93.1363636363636	-1.42920913908315\\
93.1590909090909	-1.43431789311598\\
93.1818181818182	-1.43923946042174\\
93.2045454545455	-1.44397319870763\\
93.2272727272727	-1.44851849019358\\
93.25	-1.45287474169295\\
93.2727272727273	-1.45704138468987\\
93.2954545454546	-1.4610178754135\\
93.3181818181818	-1.46480369490895\\
93.3409090909091	-1.46839834910501\\
93.3636363636364	-1.47180136887866\\
93.3863636363636	-1.47501231011626\\
93.4090909090909	-1.47803075377154\\
93.4318181818182	-1.48085630592024\\
93.4545454545455	-1.48348859781159\\
93.4772727272727	-1.48592728591635\\
93.5	-1.48817205197172\\
93.5227272727273	-1.49022260302283\\
93.5454545454546	-1.49207867146098\\
93.5681818181818	-1.49374001505858\\
93.5909090909091	-1.49520641700074\\
93.6136363636364	-1.49647768591358\\
93.6363636363636	-1.4975536558892\\
93.6590909090909	-1.49843418650735\\
93.6818181818182	-1.4991191628537\\
93.7045454545455	-1.49960849553492\\
93.7272727272727	-1.49990212069028\\
93.75	-1.5\\
93.7727272727273	-1.49990212069028\\
93.7954545454546	-1.49960849553492\\
93.8181818181818	-1.4991191628537\\
93.8409090909091	-1.49843418650735\\
93.8636363636364	-1.4975536558892\\
93.8863636363636	-1.49647768591358\\
93.9090909090909	-1.49520641700074\\
93.9318181818182	-1.49374001505858\\
93.9545454545455	-1.49207867146098\\
93.9772727272727	-1.49022260302282\\
94	-1.48817205197172\\
94.0227272727273	-1.48592728591635\\
94.0454545454546	-1.48348859781159\\
94.0681818181818	-1.48085630592024\\
94.0909090909091	-1.47803075377154\\
94.1136363636364	-1.47501231011626\\
94.1363636363636	-1.47180136887866\\
94.1590909090909	-1.46839834910501\\
94.1818181818182	-1.46480369490894\\
94.2045454545455	-1.4610178754135\\
94.2272727272727	-1.45704138468987\\
94.25	-1.45287474169295\\
94.2727272727273	-1.44851849019358\\
94.2954545454546	-1.44397319870763\\
94.3181818181818	-1.43923946042175\\
94.3409090909091	-1.43431789311598\\
94.3636363636364	-1.42920913908315\\
94.3863636363636	-1.423913865045\\
94.4090909090909	-1.41843276206523\\
94.4318181818182	-1.41276654545928\\
94.4545454545455	-1.40691595470098\\
94.4772727272727	-1.40088175332603\\
94.5	-1.39466472883238\\
94.5227272727273	-1.38826569257746\\
94.5454545454546	-1.38168547967227\\
94.5681818181818	-1.37492494887241\\
94.5909090909091	-1.367984982466\\
94.6136363636364	-1.36086648615853\\
94.6363636363636	-1.35357038895467\\
94.6590909090909	-1.34609764303704\\
94.6818181818182	-1.33844922364191\\
94.7045454545455	-1.33062612893197\\
94.7272727272727	-1.32262937986603\\
94.75	-1.31446002006579\\
94.7727272727273	-1.30611911567967\\
94.7954545454546	-1.29760775524359\\
94.8181818181818	-1.28892704953902\\
94.8409090909091	-1.28007813144792\\
94.8636363636364	-1.27106215580496\\
94.8863636363636	-1.26188029924677\\
94.9090909090909	-1.25253376005839\\
94.9318181818182	-1.24302375801688\\
94.9545454545455	-1.23335153423218\\
94.9772727272727	-1.22351835098505\\
95	-1.21352549156242\\
95.0227272727273	-1.20337426008986\\
95.0454545454545	-1.19306598136142\\
95.0681818181818	-1.18260200066669\\
95.0909090909091	-1.1719836836153\\
95.1136363636364	-1.16121241595862\\
95.1363636363636	-1.15028960340899\\
95.1590909090909	-1.13921667145622\\
95.1818181818182	-1.12799506518154\\
95.2045454545455	-1.11662624906906\\
95.2272727272727	-1.10511170681461\\
95.25	-1.09345294113212\\
95.2727272727273	-1.0816514735575\\
95.2954545454545	-1.0697088442501\\
95.3181818181818	-1.05762661179165\\
95.3409090909091	-1.04540635298294\\
95.3636363636364	-1.03304966263794\\
95.3863636363636	-1.02055815337575\\
95.4090909090909	-1.00793345541013\\
95.4318181818182	-0.995177216336693\\
95.4545454545455	-0.982291100917926\\
95.4772727272727	-0.969276790865962\\
95.5	-0.956135984623036\\
95.5227272727273	-0.942870397139868\\
95.5454545454545	-0.929481759651868\\
95.5681818181818	-0.915971819453147\\
95.5909090909091	-0.902342339668549\\
95.6136363636364	-0.888595099023506\\
95.6363636363636	-0.874731891611892\\
95.6590909090909	-0.860754526661962\\
95.6818181818182	-0.846664828300152\\
95.7045454545455	-0.832464635313051\\
95.7272727272727	-0.818155800907471\\
95.75	-0.803740192468498\\
95.7727272727273	-0.789219691315877\\
95.7954545454545	-0.77459619245844\\
95.8181818181818	-0.759871604346827\\
95.8409090909091	-0.745047848624408\\
95.8636363636364	-0.730126859876478\\
95.8863636363636	-0.715110585377798\\
95.9090909090909	-0.700000984838507\\
95.9318181818182	-0.684800030148272\\
95.9545454545455	-0.669509705119039\\
95.9772727272727	-0.654132005226076\\
96	-0.638668937347613\\
96.0227272727273	-0.623122519502836\\
96.0454545454545	-0.607494780588606\\
96.0681818181818	-0.591787760114664\\
96.0909090909091	-0.576003507937449\\
96.1136363636364	-0.560144083992561\\
96.1363636363636	-0.544211558025947\\
96.1590909090909	-0.52820800932382\\
96.1818181818182	-0.512135526441246\\
96.2045454545455	-0.495996206929578\\
96.2272727272727	-0.479792157062755\\
96.25	-0.463525491562416\\
96.2727272727273	-0.447198333321884\\
96.2954545454545	-0.430812813129109\\
96.3181818181818	-0.414371069388621\\
96.3409090909091	-0.397875247842499\\
96.3636363636364	-0.381327501290225\\
96.3863636363636	-0.364729989307811\\
96.4090909090909	-0.348084877965935\\
96.4318181818182	-0.331394339547296\\
96.4545454545455	-0.314660552263066\\
96.4772727272727	-0.297885699968612\\
96.5	-0.281071971878583\\
96.5227272727273	-0.264221562281101\\
96.5454545454545	-0.247336670251455\\
96.5681818181818	-0.230419499365051\\
96.5909090909091	-0.213472257409921\\
96.6136363636364	-0.196497156098535\\
96.6363636363636	-0.179496410779146\\
96.6590909090909	-0.162472240146701\\
96.6818181818182	-0.145426865953313\\
96.7045454545455	-0.128362512718253\\
96.7272727272727	-0.111281407437684\\
96.75	-0.0941857792939672\\
96.7727272727273	-0.0770778593648438\\
96.7954545454545	-0.059959880332161\\
96.8181818181818	-0.0428340761905369\\
96.8409090909091	-0.0257026819558326\\
96.8636363636364	-0.00856793337345922\\
96.8863636363636	0.008567933373464\\
96.9090909090909	0.0257026819558481\\
96.9318181818182	0.0428340761905523\\
96.9545454545455	0.0599598803321657\\
96.9772727272727	0.0770778593648485\\
97	0.0941857792939826\\
97.0227272727273	0.111281407437678\\
97.0454545454545	0.128362512718257\\
97.0681818181818	0.145426865953318\\
97.0909090909091	0.162472240146727\\
97.1136363636364	0.179496410779162\\
97.1363636363636	0.196497156098539\\
97.1590909090909	0.213472257409926\\
97.1818181818182	0.230419499365056\\
97.2045454545455	0.24733667025145\\
97.2272727272727	0.264221562281105\\
97.25	0.281071971878588\\
97.2727272727273	0.297885699968627\\
97.2954545454545	0.314660552263061\\
97.3181818181818	0.331394339547301\\
97.3409090909091	0.34808487796595\\
97.3636363636364	0.364729989307826\\
97.3863636363636	0.38132750129024\\
97.4090909090909	0.397875247842504\\
97.4318181818182	0.414371069388636\\
97.4545454545455	0.430812813129103\\
97.4772727272727	0.447198333321878\\
97.5	0.463525491562421\\
97.5227272727273	0.479792157062769\\
97.5454545454545	0.495996206929583\\
97.5681818181818	0.51213552644124\\
97.5909090909091	0.528208009323834\\
97.6136363636364	0.544211558025972\\
97.6363636363636	0.560144083992575\\
97.6590909090909	0.576003507937463\\
97.6818181818182	0.591787760114669\\
97.7045454545455	0.607494780588611\\
97.7272727272727	0.62312251950283\\
97.75	0.638668937347608\\
97.7727272727273	0.654132005226089\\
97.7954545454545	0.669509705119043\\
97.8181818181818	0.684800030148277\\
97.8409090909091	0.700000984838502\\
97.8636363636364	0.715110585377812\\
97.8863636363636	0.730126859876491\\
97.9090909090909	0.745047848624422\\
97.9318181818182	0.759871604346841\\
97.9545454545455	0.774596192458453\\
97.9772727272727	0.789219691315872\\
98	0.803740192468493\\
98.0227272727273	0.818155800907466\\
98.0454545454545	0.832464635313064\\
98.0681818181818	0.846664828300155\\
98.0909090909091	0.860754526661966\\
98.1136363636364	0.874731891611905\\
98.1363636363636	0.888595099023519\\
98.1590909090909	0.902342339668562\\
98.1818181818182	0.915971819453159\\
98.2045454545455	0.929481759651881\\
98.2272727272727	0.94287039713988\\
98.25	0.956135984623032\\
98.2727272727273	0.969276790865957\\
98.2954545454545	0.98229110091793\\
98.3181818181818	0.995177216336689\\
98.3409090909091	1.00793345541013\\
98.3636363636364	1.02055815337577\\
98.3863636363636	1.03304966263795\\
98.4090909090909	1.04540635298295\\
98.4318181818182	1.05762661179166\\
98.4545454545455	1.0697088442501\\
98.4772727272727	1.08165147355751\\
98.5	1.09345294113212\\
98.5227272727273	1.1051117068146\\
98.5454545454545	1.11662624906906\\
98.5681818181818	1.12799506518154\\
98.5909090909091	1.13921667145622\\
98.6136363636364	1.150289603409\\
98.6363636363636	1.16121241595863\\
98.6590909090909	1.1719836836153\\
98.6818181818182	1.1826020006667\\
98.7045454545455	1.19306598136142\\
98.7272727272727	1.20337426008987\\
98.75	1.21352549156242\\
98.7727272727273	1.22351835098505\\
98.7954545454545	1.23335153423218\\
98.8181818181818	1.24302375801689\\
98.8409090909091	1.25253376005839\\
98.8636363636364	1.26188029924677\\
98.8863636363636	1.27106215580497\\
98.9090909090909	1.28007813144793\\
98.9318181818182	1.28892704953902\\
98.9545454545455	1.2976077552436\\
98.9772727272727	1.30611911567966\\
99	1.3144600200658\\
99.0227272727273	1.32262937986603\\
99.0454545454545	1.33062612893197\\
99.0681818181818	1.33844922364191\\
99.0909090909091	1.34609764303704\\
99.1136363636364	1.35357038895467\\
99.1363636363637	1.36086648615853\\
99.1590909090909	1.36798498246601\\
99.1818181818182	1.37492494887242\\
99.2045454545455	1.38168547967228\\
99.2272727272727	1.38826569257746\\
99.25	1.39466472883238\\
99.2727272727273	1.40088175332602\\
99.2954545454545	1.40691595470098\\
99.3181818181818	1.41276654545929\\
99.3409090909091	1.41843276206524\\
99.3636363636364	1.42391386504501\\
99.3863636363637	1.42920913908315\\
99.4090909090909	1.43431789311598\\
99.4318181818182	1.43923946042175\\
99.4545454545455	1.44397319870763\\
99.4772727272727	1.44851849019358\\
99.5	1.45287474169295\\
99.5227272727273	1.45704138468987\\
99.5454545454545	1.4610178754135\\
99.5681818181818	1.46480369490894\\
99.5909090909091	1.46839834910501\\
99.6136363636364	1.47180136887866\\
99.6363636363637	1.47501231011627\\
99.6590909090909	1.47803075377154\\
99.6818181818182	1.48085630592025\\
99.7045454545455	1.48348859781159\\
99.7272727272727	1.48592728591635\\
99.75	1.48817205197172\\
99.7727272727273	1.49022260302283\\
99.7954545454545	1.49207867146098\\
99.8181818181818	1.49374001505858\\
99.8409090909091	1.49520641700074\\
99.8636363636364	1.49647768591358\\
99.8863636363637	1.4975536558892\\
99.9090909090909	1.49843418650735\\
99.9318181818182	1.4991191628537\\
99.9545454545455	1.49960849553492\\
99.9772727272727	1.49990212069028\\
100	1.5\\
};
\addlegendentry{$f$ = 80 Hz}

\end{axis}
\end{tikzpicture}%}
    \caption{This is my figure with 2 sine waves.}\label{fig:sine_waves}
\end{figure}

\section{Conclusion}
So, we can conclude that frequency, initial phase and amplitude influence a sine wave's behavior.

\bibliographystyle{IEEEtran}
\bibliography{./references}

\end{document}
